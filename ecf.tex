\chapter{异常控制流与系统 I/O}\thispagestyle{empty}
    \begin{summary}
        \begin{compactitem}
            \item 知道控制流和异常控制流的定义,知道异常及其处理是由软硬件结合实现的。
            \item 熟悉异常处理的整个过程,包括异常表等数据结构,会以某个异常为例,复述整个异常处理的过程。会区分异常和过程调用的区别。
            \item 熟练掌握异常的四种类别及其特征、常见例子,熟悉异常结束后的返回机制。了解 x86-64 系统调用的基本范式。
            \item 熟练掌握进程的概念,理解上下文、上下文切换、调度、抢占等概念及其内容,知道内核模式和用户模式的区别。熟悉并行和并发的相同和不同。
            \item 会使用 Linux 系统下提供的进程管理函数,记忆其参数填法、返回值和行为(特别是 \verb|waitpid|),熟练掌握 \verb|fork| 函数,会通过画进程图和拓扑排序的方式进行分析。
            \item 知道 Linux 的常见信号,熟悉信号处理的过程及其细节(例如 \verb|SIGKILL, SIGSTOP| 的默认行为不能修改、待处理信号至多只有一个、进程可以发信号给自己等),知道发送信号的机制和多种 API。
            \item 理解信号处理例程和主“程序”的并发性质,熟悉编写信号处理程序中需要注意的五个要点并理解其基本原理(软中断和调度问题)。信号不能用来计数。
            \item 了解非本地跳转的基本机制,知道 \verb|setjmp| 的返回值不能赋值等。
            \item 知道 Linux 系统的文件模型和文件类型,会用各种文件读写系统调用编写程序。熟悉文件元数据的结构和读写方法。了解 RIO 包的用法和目的。
            \item 熟练掌握描述符表(理解与打开文件的关系)、打开文件表、v-node 表、引用计数的概念,会结合异常控制流(\verb|fork|)和重定向(\verb|dup2, dup|)的知识分析程序。
            \item 知道标准输出和底层系统调用的区别,熟练掌握输出缓冲的机制和 \verb|fflush|。
        \end{compactitem}
    \end{summary}

    \begin{problems}
        \pro 区分异常的四种类型:填写下表(打勾)。
        \begin{table}[H]
            \centering
            \begin{tabular}{|c|c|ccc|}
                \hline
                \multirow{2}{*}{异常的种类} & \multirow{2}{*}{是否同步} & \multicolumn{3}{c|}{可能的返回行为} \\ \cline{3-5} 
                &  & \multicolumn{1}{c|}{重复当前指令} & \multicolumn{1}{c|}{执行下一条指令} & 结束进程运行 \\ \hline
                中断(interrupt) & \rule{0pt}{3ex} & \multicolumn{1}{c|}{} & \multicolumn{1}{c|}{} &  \\ \hline
                陷阱(trap) & \rule{0pt}{3ex} & \multicolumn{1}{c|}{} & \multicolumn{1}{c|}{} &  \\ \hline
                故障(fault) & \rule{0pt}{3ex} & \multicolumn{1}{c|}{} & \multicolumn{1}{c|}{} &  \\ \hline
                终止(abort) & \rule{0pt}{3ex} & \multicolumn{1}{c|}{} & \multicolumn{1}{c|}{} &  \\ \hline
            \end{tabular}
        \end{table}
        \pro 下列行为分别触发什么类型的异常?
            \qn 执行指令 \verb|mov $57, %eax; syscall|
            \qn 程序执行过程中,发现它所使用的物理内存损坏了
            \qn 程序执行过程中,试图往 \verb|main| 函数的内存中写入数据
            \qn 按下键盘
            \qn 磁盘读出了一块数据
            \qn 用 \verb|read| 函数发起磁盘读
            \qn 用户程序执行了指令 \verb|lgdt|,但是这个指令只能在内核模式下执行
        \pro 在下面的程序中,父进程和子进程的输出分别是什么?
        \begin{minted}[frame=single, fontsize=\small, linenos]{c}
    int main() {
        int a = 9;
        if (Fork() == 0)
            printf("p1: a=%d\n", a--);
        printf("p2: a=%d\n", a++);
        exit(0);
    }
        \end{minted}
        \pro 阅读下述程序。
        \begin{minted}[frame=single, fontsize=\small, linenos]{c}
    int main() {
        char c = 'A';
        printf("%c", c); fflush(stdout);
        if (fork() == 0) {
            c++;
            printf("%c", c); fflush(stdout);
        } else {
            printf("%c", c); fflush(stdout);
            fork();
        }
        c++;
        printf("%c", c); fflush(stdout);
        return 0;
    }
        \end{minted}
        假设系统调用成功,所有子进程都正常运行。判断下列哪些输出是可能的:\verb|AABBBC|、\verb|ABCABB|、\verb|ABBABC|、\verb|AACBBC|、\verb|ABABCB|、\verb|ABCBAB|。
        \pro 阅读下述程序。
        \begin{minted}[frame=single, fontsize=\small, linenos]{c}
    int main() {
        int child_status;
        char c = 'A';
        printf("%c", c); fflush(stdout);
        c++;
        if (fork() == 0) {
            printf("%c", c); fflush(stdout);
            c++;
            fork();
        } else {
            printf("%c", c); fflush(stdout);
            c += 2;
            wait(&child_status);
        } 
        printf("%c", c); fflush(stdout);
        exit(0);
    }
        \end{minted}
        假设系统调用成功,所有子进程都正常运行。判断下列哪些输出是可能的:\verb|ABBCCD|、\verb|ABBCDC|、\verb|ABBDCC|、\verb|ABDBCC|、\verb|ABCDBC|、\verb|ABCDCB|。
        \pro 阅读下述程序。
        \begin{minted}[frame=single, fontsize=\small, linenos]{c}
    void handler() {
        printf("D\n");
        return;
    }
    int main() {
        signal(SIGCHLD, handler);
        if (fork() > 0) {
            printf("A\n");
        } else {
            printf("B\n");
        } 
        printf("C\n");
        exit(0);
    }
        \end{minted}
        假设系统调用成功,所有子进程都正常运行。判断下列哪些输出是可能的:\verb|ACBC|、\verb|ABCCD|、\verb|ACBDC|、\verb|ABDCC|、\verb|BCDAC|、\verb|ABCC|。
        \pro 在某年的 ICS 课堂上,老师给同学布置了一个作业,在 Linux 上写出一份代码,使得运行它以后可输出系统能创建的进程的最大数目。下面是几位同学的答案。
            \qn Alice 同学的答案是:
            \begin{minted}[frame=single, fontsize=\small, linenos]{c}
    int main() {
        int pid;
        int count = 1;
        while ((pid = fork()) != 0) {
            // parent process
            count++;
        } 
        if (pid == 0) {
            // child process
            exit(0);
        } 
        printf("max = %d", count);
    }
            \end{minted}
            这段代码不能够正确运行,原因在于对 \verb|fork| 的返回值处理得不正确。请修改至多一处代码,使得程序正确运行。
            \qn Bob 同学对 Alice 同学修改过后的正确代码发出了疑问。Bob 同学认为,由于进程的调度时间和顺序都是不确定的,因此有的时候会调度到子进程,子进程执行 \verb|exit(0)| 以后就结束了,因此父进程可以创建更多的进程,所以 Alice 的代码输出的答案大于真实上限。请问,Bob 的说法正确吗?如果正确,请指出 Alice 应当如何修改代码,以避免 Bob 提到的问题。如果 Bob 的说法错误,请指出他错在何处。
            \qn Carol同学的答案是:
            \begin{minted}[frame=single, fontsize=\small, linenos]{c}
    int main() {
        int pid;
        int count = 1;
        while ((pid = fork()) > 0) {
            // parent process
            count++;
        } 
        if (pid == 0) {
            // child process
            while(1)
                sleep(1);
        } 
        printf("max = %d", count);
    }
            \end{minted}
            连续运行 Carol 同学的答案两次,发现结果分别如下:
            \begin{minted}[frame=single, fontsize=\small]{shell-session}
    $ ./test
    max = 1795
    $ ./test
    max = 1
            \end{minted}
                \subqn 解释为什么会发生这种情况。
                \subqn 为了解决第一次运行后的遗留问题,可以不修改代码,而直接在 Linux 终端中使用指令来解决。假设在第一次程序运行完以后,使用 \verb|ps| 指令,得到的列表前几项如下:
                \begin{minted}[frame=single, fontsize=\small]{shell-session}
    $ ./test
    max = 1795
    $ ps
    22698 pts/0     00:00:00 bash
    22725 pts/0     00:00:00 test
    22726 pts/0     00:00:00 test
    22727 pts/0     00:00:00 test
    ............
                \end{minted}
                假设 \verb|test| 程序开始运行后,没有任何新的进程被创建,并且所有进程号均按照顺序递增、逐个地分配。于是,输入下列的指令,就可以让第二次运行得到正确的结果。其中 \verb|-9| 表示 \verb|SIGKILL|。请在横线上填入正确的值。
                \begin{minted}[frame=single, fontsize=\small]{shell-session}
    $ kill -9 _________
                \end{minted}
            \qn Dave 同学修改了 Carol 同学的答案。他将 Carol 的最后一句 \verb|printf| 改为如下代码:
            \begin{minted}[frame=single, fontsize=\small, linenos]{c}
    if (pid < 0) {
        printf("max = %d", count);
        kill(0, SIGKILL);
    }
            \end{minted}
            这段代码有时无法输出任何答案。Dave 想了一想,将 \verb|printf| 中的字符串做了些修改,这样这段代码就能正确运行了。他修改了什么?
        \pro 以下问题中我们均假设缓冲区足够大,且 \verb|stdout| 只有在关闭文件、换行与 \verb|fflush| 的情况下才会刷新缓冲区;程序运行过程中的所有系统调用均成功。
        \qn 考虑下面的程序。
        \begin{minted}[frame=single, fontsize=\small, linenos]{c}
    int main() {
        printf("a");
        fork();
        printf("b");
        fork();
        printf("c");
        return 0;
    }
        \end{minted}
        写出它的一个可能的输出 \rule{3.5cm}{0.25mm},这个输出是否是唯一可能的?说明理由。
        \qn 考虑下面的程序。
        \begin{minted}[frame=single, fontsize=\small, linenos]{c}
    int main() {
        write(1, "a", 1);
        fork();
        write(1, "b", 1);
        fork();
        write(1, "c", 1);
        return 0;
    }
        \end{minted}
        以上程序的输出中有 \rule{1cm}{0.25mm} 个 \verb|a|,\rule{1cm}{0.25mm} 个 \verb|b|,\rule{1cm}{0.25mm} 个 \verb|c|;其第一个输出的字符一定是 \rule{1cm}{0.25mm}。
        \qn 考虑下面的程序。
        \begin{minted}[frame=single, fontsize=\small, linenos]{c}
    int main() {
        printf("a");
        fork();
        write(1, "b", 1);
        fork();
        write(1, "c", 1);
        return 0;
    }
        \end{minted}
        以上程序的输出中有 \rule{1cm}{0.25mm} 个 \verb|a|,\rule{1cm}{0.25mm} 个 \verb|b|,\rule{1cm}{0.25mm} 个 \verb|c|;其第一个输出的字符一定是 \rule{1cm}{0.25mm}。
        \pro 假设磁盘上有空文件 \verb|file.txt|,以下程序运行过程中的所有系统调用均成功。
        \begin{minted}[frame=single, fontsize=\small, linenos]{c}
    int main() {
        int fd1 = open("file.txt", O_RDWR | O_CREAT, S_IRUSR | S_IWUSR); 
        int fd2 = open("file.txt", O_RDWR | O_CREAT, S_IRUSR | S_IWUSR); 
        printf("%d %d\n", fd1, fd2);
        write(fd1, "123", 3); write(fd2, "45", 2);
        close(fd1); close(fd2);
        return 0;
    }
        \end{minted}
        \qn 程序关闭 \verb|fd1| 前,补全下面的 Linux 三级表结构;填写打开文件表中的 \verb|refcnt| 值。
        \begin{table}[H]
            \tt
            \centering
            \begin{tabular}{ccccc}
                文件描述符表 & {\qquad \qquad \qquad} & 打开文件表 & {\qquad \qquad \qquad} & v-node 表 \\ \cline{1-1} \cline{3-3} \cline{5-5} 
                \multicolumn{1}{|c|}{0} & \multicolumn{1}{c|}{} & \multicolumn{1}{c|}{\multirow{2}{*}{refcount = \_\_\_\_}} & \multicolumn{1}{c|}{} & \multicolumn{1}{c|}{\multirow{6}{*}{file.txt}} \\ \cline{1-1}
                \multicolumn{1}{|c|}{1} & \multicolumn{1}{c|}{} & \multicolumn{1}{c|}{} & \multicolumn{1}{c|}{} & \multicolumn{1}{c|}{} \\ \cline{1-1} \cline{3-3}
                \multicolumn{1}{|c|}{2} &  &  & \multicolumn{1}{c|}{} & \multicolumn{1}{c|}{} \\ \cline{1-1}
                \multicolumn{1}{|c|}{3} &  &  & \multicolumn{1}{c|}{} & \multicolumn{1}{c|}{} \\ \cline{1-1} \cline{3-3}
                \multicolumn{1}{|c|}{4} & \multicolumn{1}{c|}{} & \multicolumn{1}{c|}{\multirow{2}{*}{refcount = \_\_\_\_}} & \multicolumn{1}{c|}{} & \multicolumn{1}{c|}{} \\ \cline{1-1}
                \multicolumn{1}{|c|}{5} & \multicolumn{1}{c|}{} & \multicolumn{1}{c|}{} & \multicolumn{1}{c|}{} & \multicolumn{1}{c|}{} \\ \cline{1-1} \cline{3-3} \cline{5-5} 
            \end{tabular}
        \end{table}
        \qn 程序结束时,标准输出上的内容是 \rule{3.5cm}{0.25mm},\verb|file.txt| 中的内容是 \rule{3.5cm}{0.25mm}。
        \pro 假设磁盘上有空文件 \verb|file.txt|,以下程序运行过程中的所有系统调用均成功。
        \begin{minted}[frame=single, fontsize=\small, linenos]{c}
    int main() {
        int fd1 = open("file.txt", O_RDWR | O_CREAT, S_IRUSR | S_IWUSR); 
        int fd2 = open("file.txt", O_RDWR | O_CREAT, S_IRUSR | S_IWUSR); 
        dup2(fd2, fd1);
        printf("%d %d\n", fd1, fd2);
        write(fd1, "123", 3); write(fd2, "45", 2);
        close(fd1); close(fd2);
        return 0;
    }
        \end{minted}
        \qn 程序关闭 \verb|fd1| 前,补全下面的 Linux 三级表结构;填写打开文件表中的 \verb|refcnt| 值。
        \begin{table}[H]
            \tt
            \centering
            \begin{tabular}{ccccc}
                文件描述符表 & {\qquad \qquad \qquad} & 打开文件表 & {\qquad \qquad \qquad} & v-node 表 \\ \cline{1-1} \cline{3-3} \cline{5-5} 
                \multicolumn{1}{|c|}{0} & \multicolumn{1}{c|}{} & \multicolumn{1}{c|}{\multirow{2}{*}{refcount = \_\_\_\_}} & \multicolumn{1}{c|}{} & \multicolumn{1}{c|}{\multirow{6}{*}{file.txt}} \\ \cline{1-1}
                \multicolumn{1}{|c|}{1} & \multicolumn{1}{c|}{} & \multicolumn{1}{c|}{} & \multicolumn{1}{c|}{} & \multicolumn{1}{c|}{} \\ \cline{1-1} \cline{3-3}
                \multicolumn{1}{|c|}{2} &  &  & \multicolumn{1}{c|}{} & \multicolumn{1}{c|}{} \\ \cline{1-1}
                \multicolumn{1}{|c|}{3} &  &  & \multicolumn{1}{c|}{} & \multicolumn{1}{c|}{} \\ \cline{1-1} \cline{3-3}
                \multicolumn{1}{|c|}{4} & \multicolumn{1}{c|}{} & \multicolumn{1}{c|}{\multirow{2}{*}{refcount = \_\_\_\_}} & \multicolumn{1}{c|}{} & \multicolumn{1}{c|}{} \\ \cline{1-1}
                \multicolumn{1}{|c|}{5} & \multicolumn{1}{c|}{} & \multicolumn{1}{c|}{} & \multicolumn{1}{c|}{} & \multicolumn{1}{c|}{} \\ \cline{1-1} \cline{3-3} \cline{5-5} 
            \end{tabular}
        \end{table}
        \qn 程序结束时,标准输出上的内容是 \rule{3.5cm}{0.25mm},\verb|file.txt| 中的内容是 \rule{3.5cm}{0.25mm}。
        \pro 假设磁盘上有空文件 \verb|file.txt|,以下程序运行过程中的所有系统调用均成功;缓冲区足够大,且 \verb|stdout| 只有在关闭文件、换行与 \verb|fflush| 的情况下才会刷新缓冲区。
        \begin{minted}[frame=single, fontsize=\small, linenos]{c}
    int main() {
        pid_t pid;
        int child_status;
        int fd1 = open("file.txt", O_RDWR | O_CREAT, S_IRUSR | S_IWUSR); 
        if ((pid = fork()) > 0) {
            printf("P:%d ", fd1);
            write(fd1, "123", 3);
            waitpid(pid, &child_status, 0);
        } else {
            printf("C:%d ", fd1);
            write(fd1, "45", 2);
        } 
        close(fd1);
        return 0;
    }
        \end{minted}
        \qn 子进程关闭 \verb|fd1| 前,补全下面的 Linux 三级表结构;填写打开文件表中的 \verb|refcnt| 值。
        \begin{table}[H]
            \tt
            \centering
            \begin{tabular}{ccccc}
                文件描述符表 & {\qquad \qquad \qquad} & 打开文件表 & {\qquad \qquad \qquad} & v-node 表 \\ \cline{1-1} \cline{3-3} \cline{5-5} 
                \multicolumn{1}{|c|}{Parent 3} & \multicolumn{1}{c|}{} & \multicolumn{1}{c|}{\multirow{2}{*}{refcount = \_\_\_\_}} & \multicolumn{1}{c|}{} & \multicolumn{1}{c|}{\multirow{6}{*}{file.txt}} \\ \cline{1-1}
                \multicolumn{1}{|c|}{Parent 4} & \multicolumn{1}{c|}{} & \multicolumn{1}{c|}{} & \multicolumn{1}{c|}{} & \multicolumn{1}{c|}{} \\ \cline{1-1} \cline{3-3}
                &  &  & \multicolumn{1}{c|}{} & \multicolumn{1}{c|}{} \\
                &  &  & \multicolumn{1}{c|}{} & \multicolumn{1}{c|}{} \\ \cline{1-1} \cline{3-3}
                \multicolumn{1}{|c|}{Child 3} & \multicolumn{1}{c|}{} & \multicolumn{1}{c|}{\multirow{2}{*}{refcount = \_\_\_\_}} & \multicolumn{1}{c|}{} & \multicolumn{1}{c|}{} \\ \cline{1-1}
                \multicolumn{1}{|c|}{Child 4} & \multicolumn{1}{c|}{} & \multicolumn{1}{c|}{} & \multicolumn{1}{c|}{} & \multicolumn{1}{c|}{} \\ \cline{1-1} \cline{3-3} \cline{5-5} 
            \end{tabular}
        \end{table}
        \qn 程序结束时,标准输出上的内容是 \rule{3.5cm}{0.25mm},\verb|file.txt| 中的内容是 \rule{3.5cm}{0.25mm}。(都写出所有可能答案)
    \end{problems}

\chapter{异常控制流与系统 I/O{---}往年考题}\thispagestyle{empty}
    \begin{problems}
        \proy{2018} 关于进程,以下说法正确的是:
        \begin{choices}
            \item 没有设置模式位时,进程运行在用户模式中,允许执行特权指令,例如发起 I/O 操作。
            \item 调用 \verb|waitpid(-1, NULL, WNOHANG & WUNTRACED)| 会立即返回:如果调用进程的所有子进程都没有被停止或终止,则返回 \verb|0|;如果有停止或终止的子进程,则返回其中一个的 \verb|PID|。
            \item \verb|execve| 函数的第三个参数 \verb|envp| 指向一个以 \verb|null| 结尾的指针数组,其中每一 个指针指向一个形如“\verb|name=value|”的环境变量字符串。
            \item 进程可以通过使用 \verb|signal| 函数修改和信号相关联的默认行为,唯一的例外是 \verb|SIGKILL|,它的默认行为是不能修改的。
        \end{choices}
        \proy{2018} 假设某进程恰有五个已打开的文件描述符 0\textasciitilde4,分别引用五个不同文件,尝试运行以下代码:\verb|dup2(3, 2); dup2(0, 3); dup2(1, 10); dup2(10, 4); dup2(4, 0);|。关于得到的结果,说法正确的是:
        \begin{choices}
            \item 运行正常完成,现在有四个描述符引用同一个文件。
            \item 运行正常完成,现在进程共引用四个不同的文件。
            \item 由于试图从一个未打开的描述符进行复制,发生错误。
            \item 由于试图向一个未打开的描述符进行复制,发生错误。
        \end{choices}
        \proy{2018} Bob 是一名刚刚学完异常的同学,他希望通过配合 \verb|kill| 和 \verb|signal| 的使用,能让两个进程向同一个文件中交替地打印出字符。可惜他的 tshlab 做得不过关,导致他写的这个程序有各种问题。
        \begin{minted}[frame=single, fontsize=\small, linenos]{c}
    #include "csapp.h"
    #define MAXN 6
    int parentPID = 0;
    int childPID = 0;
    int count = 1;
    int fd1 = 1;
    void handler1() {
        if (count > MAXN)
            return;
        for (int i = 0; i < count; i++)
            write(fd1, "+", 1);
        __________X__________
        kill(parentPID, SIGUSR2);
    }
    void handler2() {
        if (count > MAXN)
            return;
        for (int i = 0; i < count; i++)
            write(fd1, "-", 1);
        __________Y__________
        kill(childPID, SIGUSR1);
    }

    int main() {
        signal(SIGUSR1, handler1);
        signal(SIGUSR2, handler2);
        parentPID = getpid();
        childPID = fork();
        fd1 = open("file.txt", O_RDWR);
        if (childPID) {
            __________Z__________
            kill(childPID, SIGUSR1);
        }
        exit(0);
    }
        \end{minted}
        假设程序能在任意时刻被系统打断、调度,并且调度的时间切片大小是不确定的,可以充分长。在每次程序执行前,\verb|file.txt| 是一个已经存在的空文件。
        \qn 此时,\verb|X| 处语句和 \verb|Y| 处语句都是 \verb|count++;|,\verb|Z| 处语句是空语句。Alice 测试该代码,发现有时 \verb|file.txt| 中没有任何输出。请解释原因。
        
        \begin{hint}
            考虑第 28 行 \verb|fork| 以后,下一次被调度的进程,并从这个角度回答本题。同时要求简述解决方案。
        \end{hint}
        \qn Bob 根据 Alice 的反馈,在某两行之间加了若干代码,修复了前述的问题。当 \verb|X| 处代码和 \verb|Y| 处代码都是 \verb|count++;|,\verb|Z| 处为空时,Bob 期望 \verb|file.txt| 中的输出是:
        \begin{center}
            \verb|+-++--+++---++++----+++++-----++++++------|
        \end{center}
        可 Alice 测评 Bob 的程序的时候,却发现有时 Bob 的程序在 \verb|file.txt| 中的输出是:
        \begin{center}
            \verb|+-+--+---+----+------|
        \end{center}
        而与此同时,终端上出现了如下的输出:\verb|+|。现在要找出程序的问题,所以要对上面的输出过程进行分析,请按以下要求填写后面的表格:
        \begin{compactenum}[(i)]
            \item 在描述符表一栏中,勾选该进程当前 \verb|fd1| 的值。
            \item 在打开文件表一栏中,填写该项的 \verb|refcnt|。如果某一项不存在,请在括号中写 \verb|0|。
            \item 画出描述符表到打开文件表的表项指向关系。不需要画关于标准输入/标准输出/标准错误的箭头。评分时不对箭头评分,请务必保证(i)、(ii)两步的解答与箭头连接的情况相一致。
        \end{compactenum}
        \subqn 当程序第一次在终端上输出 \verb|+| 的瞬间,Linux 三级表的结构如下,请补全:
        \begin{table}[H]
            \tt
            \centering
            \begin{tabular}{cccccc}
                \multicolumn{2}{c}{文件描述符表} & {\qquad} & 打开文件表 & {\qquad} & v-node 表 \\ \cline{1-2} \cline{4-4} \cline{6-6} 
                \multicolumn{1}{|c|}{\multirow{4}{*}{父进程}} & \multicolumn{1}{c|}{(     ) 0} & \multicolumn{1}{c|}{} & \multicolumn{1}{c|}{\multirow{4}{*}{refcount = \_\_\_\_}} & \multicolumn{1}{c|}{} & \multicolumn{1}{c|}{\multirow{9}{*}{file.txt}} \\ \cline{2-2}
                \multicolumn{1}{|c|}{} & \multicolumn{1}{c|}{(     ) 1} & \multicolumn{1}{c|}{} & \multicolumn{1}{c|}{} & \multicolumn{1}{c|}{} & \multicolumn{1}{c|}{} \\ \cline{2-2}
                \multicolumn{1}{|c|}{} & \multicolumn{1}{c|}{(     ) 2} & \multicolumn{1}{c|}{} & \multicolumn{1}{c|}{} & \multicolumn{1}{c|}{} & \multicolumn{1}{c|}{} \\ \cline{2-2}
                \multicolumn{1}{|c|}{} & \multicolumn{1}{c|}{(     ) 3} & \multicolumn{1}{c|}{} & \multicolumn{1}{c|}{} & \multicolumn{1}{c|}{} & \multicolumn{1}{c|}{} \\ \cline{1-2} \cline{4-4}
                &  &  &  & \multicolumn{1}{c|}{} & \multicolumn{1}{c|}{} \\ \cline{1-2} \cline{4-4}
                \multicolumn{1}{|c|}{\multirow{4}{*}{子进程}} & \multicolumn{1}{c|}{(     ) 0} & \multicolumn{1}{c|}{} & \multicolumn{1}{c|}{\multirow{4}{*}{refcount = \_\_\_\_}} & \multicolumn{1}{c|}{} & \multicolumn{1}{c|}{} \\ \cline{2-2}
                \multicolumn{1}{|c|}{} & \multicolumn{1}{c|}{(     ) 1} & \multicolumn{1}{c|}{} & \multicolumn{1}{c|}{} & \multicolumn{1}{c|}{} & \multicolumn{1}{c|}{} \\ \cline{2-2}
                \multicolumn{1}{|c|}{} & \multicolumn{1}{c|}{(     ) 2} & \multicolumn{1}{c|}{} & \multicolumn{1}{c|}{} & \multicolumn{1}{c|}{} & \multicolumn{1}{c|}{} \\ \cline{2-2}
                \multicolumn{1}{|c|}{} & \multicolumn{1}{c|}{(     ) 3} & \multicolumn{1}{c|}{} & \multicolumn{1}{c|}{} & \multicolumn{1}{c|}{} & \multicolumn{1}{c|}{} \\ \cline{1-2} \cline{4-4} \cline{6-6} 
            \end{tabular}
        \end{table}
        \subqn 当程序第一次在 \verb|file.txt| 中输出 \verb|+| 的瞬间,Linux 三级表的结构如下,请补全:
        \begin{table}[H]
            \tt
            \centering
            \begin{tabular}{cccccc}
                \multicolumn{2}{c}{文件描述符表} & {\qquad} & 打开文件表 & {\qquad} & v-node 表 \\ \cline{1-2} \cline{4-4} \cline{6-6} 
                \multicolumn{1}{|c|}{\multirow{4}{*}{父进程}} & \multicolumn{1}{c|}{(     ) 0} & \multicolumn{1}{c|}{} & \multicolumn{1}{c|}{\multirow{4}{*}{refcount = \_\_\_\_}} & \multicolumn{1}{c|}{} & \multicolumn{1}{c|}{\multirow{9}{*}{file.txt}} \\ \cline{2-2}
                \multicolumn{1}{|c|}{} & \multicolumn{1}{c|}{(     ) 1} & \multicolumn{1}{c|}{} & \multicolumn{1}{c|}{} & \multicolumn{1}{c|}{} & \multicolumn{1}{c|}{} \\ \cline{2-2}
                \multicolumn{1}{|c|}{} & \multicolumn{1}{c|}{(     ) 2} & \multicolumn{1}{c|}{} & \multicolumn{1}{c|}{} & \multicolumn{1}{c|}{} & \multicolumn{1}{c|}{} \\ \cline{2-2}
                \multicolumn{1}{|c|}{} & \multicolumn{1}{c|}{(     ) 3} & \multicolumn{1}{c|}{} & \multicolumn{1}{c|}{} & \multicolumn{1}{c|}{} & \multicolumn{1}{c|}{} \\ \cline{1-2} \cline{4-4}
                &  &  &  & \multicolumn{1}{c|}{} & \multicolumn{1}{c|}{} \\ \cline{1-2} \cline{4-4}
                \multicolumn{1}{|c|}{\multirow{4}{*}{子进程}} & \multicolumn{1}{c|}{(     ) 0} & \multicolumn{1}{c|}{} & \multicolumn{1}{c|}{\multirow{4}{*}{refcount = \_\_\_\_}} & \multicolumn{1}{c|}{} & \multicolumn{1}{c|}{} \\ \cline{2-2}
                \multicolumn{1}{|c|}{} & \multicolumn{1}{c|}{(     ) 1} & \multicolumn{1}{c|}{} & \multicolumn{1}{c|}{} & \multicolumn{1}{c|}{} & \multicolumn{1}{c|}{} \\ \cline{2-2}
                \multicolumn{1}{|c|}{} & \multicolumn{1}{c|}{(     ) 2} & \multicolumn{1}{c|}{} & \multicolumn{1}{c|}{} & \multicolumn{1}{c|}{} & \multicolumn{1}{c|}{} \\ \cline{2-2}
                \multicolumn{1}{|c|}{} & \multicolumn{1}{c|}{(     ) 3} & \multicolumn{1}{c|}{} & \multicolumn{1}{c|}{} & \multicolumn{1}{c|}{} & \multicolumn{1}{c|}{} \\ \cline{1-2} \cline{4-4} \cline{6-6} 
            \end{tabular}
        \end{table}
        \subqn 如果要产生 Bob 预期的输出,Linux 三级表的关系应当如下,请补全:
        \begin{table}[H]
            \tt
            \centering
            \begin{tabular}{cccccc}
                \multicolumn{2}{c}{文件描述符表} & {\qquad} & 打开文件表 & {\qquad} & v-node 表 \\ \cline{1-2} \cline{4-4} \cline{6-6} 
                \multicolumn{1}{|c|}{\multirow{4}{*}{父进程}} & \multicolumn{1}{c|}{(     ) 0} & \multicolumn{1}{c|}{} & \multicolumn{1}{c|}{\multirow{4}{*}{refcount = \_\_\_\_}} & \multicolumn{1}{c|}{} & \multicolumn{1}{c|}{\multirow{9}{*}{file.txt}} \\ \cline{2-2}
                \multicolumn{1}{|c|}{} & \multicolumn{1}{c|}{(     ) 1} & \multicolumn{1}{c|}{} & \multicolumn{1}{c|}{} & \multicolumn{1}{c|}{} & \multicolumn{1}{c|}{} \\ \cline{2-2}
                \multicolumn{1}{|c|}{} & \multicolumn{1}{c|}{(     ) 2} & \multicolumn{1}{c|}{} & \multicolumn{1}{c|}{} & \multicolumn{1}{c|}{} & \multicolumn{1}{c|}{} \\ \cline{2-2}
                \multicolumn{1}{|c|}{} & \multicolumn{1}{c|}{(     ) 3} & \multicolumn{1}{c|}{} & \multicolumn{1}{c|}{} & \multicolumn{1}{c|}{} & \multicolumn{1}{c|}{} \\ \cline{1-2} \cline{4-4}
                &  &  &  & \multicolumn{1}{c|}{} & \multicolumn{1}{c|}{} \\ \cline{1-2} \cline{4-4}
                \multicolumn{1}{|c|}{\multirow{4}{*}{子进程}} & \multicolumn{1}{c|}{(     ) 0} & \multicolumn{1}{c|}{} & \multicolumn{1}{c|}{\multirow{4}{*}{refcount = \_\_\_\_}} & \multicolumn{1}{c|}{} & \multicolumn{1}{c|}{} \\ \cline{2-2}
                \multicolumn{1}{|c|}{} & \multicolumn{1}{c|}{(     ) 1} & \multicolumn{1}{c|}{} & \multicolumn{1}{c|}{} & \multicolumn{1}{c|}{} & \multicolumn{1}{c|}{} \\ \cline{2-2}
                \multicolumn{1}{|c|}{} & \multicolumn{1}{c|}{(     ) 2} & \multicolumn{1}{c|}{} & \multicolumn{1}{c|}{} & \multicolumn{1}{c|}{} & \multicolumn{1}{c|}{} \\ \cline{2-2}
                \multicolumn{1}{|c|}{} & \multicolumn{1}{c|}{(     ) 3} & \multicolumn{1}{c|}{} & \multicolumn{1}{c|}{} & \multicolumn{1}{c|}{} & \multicolumn{1}{c|}{} \\ \cline{1-2} \cline{4-4} \cline{6-6} 
            \end{tabular}
        \end{table}
        \qn 对于上一问的错误代码,如果终端上输出的是 \verb|+++|,那么 \verb|file.txt| 中的内容是什么?答:\rule{6cm}{0.25mm}。
        \qn Bob 修复了(2)的问题,使得代码能够产生预期的输出。现在,Bob 又希望自己的代码最终输出的是 \verb|+--+++----+++++------|,为此,他对 \verb|X, Y, Z| 处作了如下的修改。\verb|X, Y| 处语句已做如下填写,请补上 \verb|Z| 处语句。
        \begin{compactitem}
            \item \verb|X| 处填写为 \verb|count += 2;|。
            \item \verb|Y| 处填写为 \verb|count += 2;|。
            \item \verb|Z| 处填写为 \rule{6cm}{0.25mm}。
        \end{compactitem}
        \proy{2016} 下列哪一事件不会导致信号被发送到进程?
        \begin{choices}
            \item 新连接到达监听端口
            \item 进程访问非法地址
            \item 除零
            \item 上述情况都不对
        \end{choices}
        \proy{2016} 考虑以下代码,假设 \verb|result.txt| 中的初始内容为 \verb|666666|。
        \begin{minted}[frame=single, fontsize=\small, linenos]{c}
    char *str1 = "6666";
    char *str2 = "2333";
    char *str3 = "hhhh";
    int fd1, fd2, fd3, i;
    fd1 = open("result.txt", O_RDWR);
    fd2 = open("result.txt", O_RDWR);
    dup2(fd1, fd2);
    for (i = 0; i < 5; ++i) {
        fd3 = open("result.txt", O_RDWR);
        write(fd1, str1, 4);
        write(fd2, str2, 4);
        write(fd3, str3, 4);
        close(fd3);
    }
    close(fd1); close(fd2);
        \end{minted}
        假设所有系统调用均成功,则这段代码执行结束后,\verb|result.txt| 的内容中有(\qquad)个 \verb|6|。
        \begin{choices}
            \item 6
            \item 16
            \item 20
            \item 22
        \end{choices}
        \proy{2016} 关于 I/O 操作,以下说法中正确的是:
        \begin{choices}
            \item 由于 RIO 包的健壮性,所以 RIO 中的函数都可以交叉调用。
            \item 成功调用 \verb|open| 函数后,一定返回一个不小于 3 的文件描述符。
            \item 调用 Unix I/O 开销较大,标准 I/O 库使用缓冲区来加快 I/O 的速度。
            \item 和描述符表一样,每个进程拥有独立的打开文件表。
        \end{choices}
        \proy{2016} 请阅读以下程序,然后回答问题。假设程序中的函数调用都可以正确执行,并默认 \verb|printf| 执行完会调用 \verb|fflush|。
        \begin{minted}[frame=single, fontsize=\small, linenos]{c}
    int main() {
        int cnt = 1;
        int pid_1, pid_2;
        pid_1 = fork();
        if (pid_1 == 0) {
            pid_2 = fork();
            if(pid_2 != 0) {
                wait(pid_2, NULL, 0);
                printf("B");
            }
            printf("F");
            exit(0);
        } else {
            __________A__________
            __________B__________
            wait(pid_1, NULL, 0);
            pid_2 = fork();
            if (pid_2 == 0) {
                printf("D");
                cnt -= 1;
            }
            if(cnt == 0)
                printf("E");
            else
                printf("G");
            exit(0);
        }
    }
        \end{minted}
        \qn 如果程序中的 \verb|A, B| 位置的代码为空,列出所有可能的输出结果:\rule{3cm}{0.25mm}。
        \qn 如果程序中的 \verb|A, B| 位置的代码分别为 \verb|printf("C");|、\verb|exit(0);|,列出所有可能的输出结果:\rule{6cm}{0.25mm}。
        \proy{2016} 请阅读以下程序,然后回答问题(假设程序中的函数调用都可以正确执行,且每条语句都是原子动作):
        \begin{minted}[frame=single, fontsize=\small, linenos]{c}
    pid_t pid;
    int even = 0;
    int counter1 = 0;
    int counter2 = 1;
    void handler1(int sig) {
        if (even % 2 == 0) {
            printf("%d\n", counter1);
            counter1 = ______________________;
        } else {
            printf("%d\n", counter2);
            counter2 = ______________________;
            even = even + ______________________;
        }
    }
    void handler2(int sig) {
        if (______________________) {
            counter1 = even * even;
        } else {
            counter2 = even * even;
        }
    }
    int main() {
        signal(SIGUSR1, handler1);
        signal(SIGUSR2, handler2);

        if ((pid = fork()) == 0) {
            while (1) {};
        }
        while (even < 30) {
            kill(pid, ______________________);
            sleep(1);
            kill(pid, ______________________);
            sleep(1);
            even = even + ______________________;
        }
        kill(pid, SIGKILL);
        exit(0);
    }
        \end{minted}
        完成程序,使得程序在输出的数字为以下 $Q$ 队列的前 30 项,$Q$ 队列定义为
        \[ Q_0=0, \quad Q_1=1, \quad Q_{n+1} = \begin{cases}
            Q_n+1 \quad (2 \mid n), \\
            2Q_n \quad (2 \nmid n),
        \end{cases} \quad (n \in \mathbb N). \]
        注意:若某个位置中的程序内容对本次程序执行结果没有影响,请在相应位置填写 “无关”。
        \proy{2015} 设一段程序中阻塞了 \verb|SIGCHLD| 和 \verb|SIGUSR1| 信号。接下来,向它按顺序发送 \verb|SIGCHLD, SIGUSR1, SIGCHLD| 信号,当程序取消阻塞继续执行时,将处理这三个信号中的哪几个?
        \begin{choices}
            \item 都不处理
            \item 处理一次 \verb|SIGCHLD|
            \item 处理一次 \verb|SIGCHLD|,一次 \verb|SIGUSR1|
            \item 处理所有三个信号
        \end{choices}
        \proy{2015} 学完本课程后,几位同学聚在一起讨论有关异常的话题,请问下面哪一个说法有误?
        \begin{choices}
            \item 发生异常和异常处理意味着控制流的突变。
            \item 与异常相关的处理是由硬件和操作系统共同完成的。
            \item 异常是由于计算机系统发生了不可恢复的错误导致的。
            \item 异常的发生可能是异步的,也可能是同步的。
        \end{choices}
        \proy{2015} 下列说法正确的是:
        \begin{choices}
            \item \verb|SIGTSTP| 信号既不能被捕获,也不能被忽略。
            \item 存在信号的默认处理行为是进程停止直到被 \verb|SIGCONT| 信号重启。
            \item 系统调用不能被中断,因为那是操作系统的工作。
            \item 子进程能给父进程发送信号,但不能发送给兄弟进程。
        \end{choices}
        \proy{2015} 在系统调用成功的情况下,下面哪个是以下程序可能的输出?
        \begin{minted}[frame=single, fontsize=\small, linenos]{c}
    int main() {
        int pid = fork();
        if (pid == 0) {
            printf("A");
        } else {
            pid = fork();
            if (pid == 0) {
                printf("A");
            } else {
                printf("B");
            }
        }
        exit(0);
    }
        \end{minted}
        \begin{choices}
            \item \verb|AAB|
            \item \verb|AAA|
            \item \verb|AABB|
            \item \verb|AA|
        \end{choices}
        \proy{2015} 以下关于 Unix I/O 的说法正确的是:
        \begin{choices}
            \item 从网络套接字(socket)读取内容时,可以通过反复读的方式处理不足值问题,直到读完所需要的数量或遇到 \verb|EOF| 为止。
            \item 以 \verb|O_RDWR| 方式打开文件后,文件会有两个指针,分别记录读文件的当前位置和写文件的当前位置。
            \item 用 \verb|read| 函数直接读取控制台输入的文本行,会自动在行末追加 \verb|\0| 字符。
            \item 使用 \verb|dup2(4, 1)| 成功进行重定向后执行 \verb|close(4)|,会导致 1 号文件描述符也不可用。
        \end{choices}
        \begin{hint}
            \verb|O_RDWR| 表示文件可读可写。\verb|dup2(oldfd, newfd)| 表示将 \verb|oldfd| 重定向给 \verb|newfd|。
        \end{hint}
        \proy{2015} 考虑如下程序代码:
        \begin{minted}[frame=single, fontsize=\small, linenos]{c}
    #include <stdio.h>
    #include "csapp.h"
    int main() {
        printf("2");
        if (Fork()) {
            printf("33");
            Write(STDOUT_FILENO, "lol", 3);
        } else {
            Sleep(1);
            printf("233");
            Write(STDOUT_FILENO, "hhhh", 4);
        }
        fflush(stdout);
        return 0;
    }
        \end{minted}
        编译后运行程序,程序正常退出。那么程序的输出最可能是:
        \begin{choices}
            \item \verb|233lol233hhhh|
            \item \verb|lol233hhhh2233|
            \item \verb|233lol2233hhhh|
            \item \verb|2lol33hhhh233|
        \end{choices}
        \proy{2015} 以下程序运行时系统调用全部正确执行,且每个信号都被处理。请给出代码运行后所有可能的输出结果。
        \begin{minted}[frame=single, fontsize=\small, linenos]{c}
    #include <stdio.h>
    #include <stdlib.h>
    #include <unistd.h>
    #include <signal.h>

    int c = 1;
    void handler1(int sig) {
        c++;
        printf("%d", c);
    }

    int main() {
        signal(SIGUSR1, handler1);
        sigset_t s;
        sigemptyset(&s);
        sigaddset(&s, SIGUSR1);
        sigprocmask(SIG_BLOCK, &s, 0);

        int pid = fork() ? fork() : fork();
        if (pid == 0) {
            kill(getppid(), SIGUSR1);
            printf("S");
            sigprocmask(SIG_UNBLOCK, &s, 0);
            exit(0);
        } else {
            while (waitpid(-1, NULL, 0) != -1);
            sigprocmask(SIG_UNBLOCK, &s, 0);
            printf("P");
        }
        return 0;
    }
        \end{minted}
        \proy{2014} 在系统调用成功的情况下,下列代码会输出几个 \verb|hello|?
        \begin{minted}[frame=single, fontsize=\small, linenos]{c}
    void doit() {
        if (fork() == 0) {
            printf("hello\n");
            fork();
        }
        return;
    }
    int main() {
        doit();
        printf("hello\n");
        exit(0);
    }
        \end{minted}
        \begin{choices}
            \item 3
            \item 4
            \item 5
            \item 6
        \end{choices}
        \proy{2014} 下列说法中哪一个是错误的?
        \begin{choices}
            \item 中断一定是异步发生的。
            \item 异常处理程序一定运行在内核模式下。
            \item 故障处理一定返回到当前指令。
            \item 陷阱一定是同步发生的。
        \end{choices}
        \proy{2014} 下列这段代码的输出不可能是:
        \begin{minted}[frame=single, fontsize=\small, linenos]{c}
    void handler() {
        printf("h");
    }
    int main() {
        signal(SIGCHLD, handler) ;
        if (fork() == 0) {
            printf("a") ;
        } else {
            printf("b") ;
        }
        printf("c") ;
        exit(0);
    }
        \end{minted}
        \begin{choices}
            \item \verb|abcc|
            \item \verb|abch|
            \item \verb|bcach|
            \item \verb|bchac|
        \end{choices}
        \proy{2014} 设文本文件 \verb|ICS.txt| 中包含 3000 个字符,考虑如下代码段:
        \begin{minted}[frame=single, fontsize=\small, linenos]{c}
    int main(int argc, char** argv) {
        int fd = open("ICS.txt", O_CREAT | O_RDWR, S_IRUSR | S_IWUSR);
        write(fd, "ICS", 3);
        char buf[128];
        int i;
        for (i = 0; i < 10; i++) {
            int fd1 = open("ICS.txt", O_RDWR);
            int fd2 = dup(fd1);
            int cnt = read(fd1, buf, 128);
            write(fd2, buf, cnt);
        }
        return 0; 
    }
        \end{minted}
        上述代码执行完后,\verb|ICS.txt| 中包含多少个字符(假设所有系统调用都成功)?
        \begin{choices}
            \item 3
            \item 256
            \item 3000
            \item 3072
        \end{choices}
        \proy{2014} 下列系统 I/O 的说法中,正确的是:
        \begin{choices}
            \item C 语言中的标准 I/O 函数在不同操作系统中的实现代码一样。
            \item 对于同一个文件描述符,混用 RIO 包中的 \verb|rio_readnb| 和 \verb|rio_readn| 两个函数不会造成问题。
            \item C 语言中的标准 I/O 函数是异步线程安全的。
            \item 使用 I/O 缓冲区可以减少系统调用的次数,从而加快 I/O 的速度。
        \end{choices}
        \proy{2014} 以下程序运行时系统调用均正确执行,\verb|buffer.txt| 的初始内容为 \verb|pekinguniv|。请给出代码运行后打印输出的结果,并给出程序运行结束后 \verb|buffer.txt| 文件的内容。
        \begin{minted}[frame=single, fontsize=\small, linenos]{c}
    #include <stdio.h>
    #include <stdlib.h>
    #include <fcntl.h>
    #include <unistd.h>

    int main() {
        char c;
        int file1 = open("buffer.txt", O_RDWR);
        int file2;

        read(file1, &c, 1);
        file2 = dup(file1);
        write(file2, &c, 1);
        printf("1 = %c\n", c);

        int pid = fork() ;
        if (pid == 0) {
            read(file1, &c, 1);
            write(file2, &c, 1);
            printf("2 = %c\n", c);
            read(file1, &c, 1);
            printf("3 = %c\n", c);
            close(file1);
            exit(0);
        } else {
            waitpid(pid, NULL, 0);
            close(file2);
            dup2(file1, file2);
            read(file2, &c, 1);
            write(file2, &c, 1);
            printf("4 = %c\n", c);
        }
        return 0;
    }
        \end{minted}
        \proy{2014} 某程序员实现了一个自己的 \verb|sleep| 函数,请分析该代码存在哪些问题。
        \begin{minted}[frame=single, fontsize=\small, linenos]{c}
    #include <signal.h>
    #include <unistd.h>
    static void sig_alrm(int signo) {
        /* nothing to do, just return to wake up the pause */
    }

    unsigned int sleep(unsigned int seconds) {
        if (signal(SIGALRM, sig_alrm) == SIG_ERR)
            return seconds;

        alarm(seconds); /* start the timer */
        pause(); /* next caught signal wakes us up */
        return alarm(0); /* turn off timer, return unslept time */
    }
        \end{minted}
        \proy{2014} 请阅读下面的代码:
        \begin{minted}[frame=single, fontsize=\small, linenos]{c}
    int main(int argc, char** argv) {
        int fd1 = open("ICS.txt", O_CREAT | O_RDWR, S_IRUSR | S_IWUSR);
        
        write(fd1, "abc", 3);

        int fd2 = fd1;
        int fd3 = dup(fd2);
        int fd4 = open("ICS.txt", O_APPEND | O_RDWR);
        write(fd2, "defghi", 6);
        write(fd4, "xyz", 3);

        int fd5 = fd4;
        dup2(fd3, fd5);
        write(fd4, "pqr", 3);

        close(fd1);
        return 0;
    }
        \end{minted}
        \qn 设初始时 \verb|ICS.txt| 文件不存在。程序执行时,所有的系统调用均会成功,所有表项均会从小到大依次分配,描述符表一开始被占用掉前 3 个表项。请填写在第 16 行代码刚刚执行完之后,下面的打开文件表和 v-node 表中表项 的部分值,并画出表项之间的指向关系。对于已经释放的打开文件表表项,请填写释放前那一刻的值和指向的 v-node 表表项。可以忽略多余的表项。
        \begin{table}[H]
            \tt
            \centering
            \begin{tabular}{ccccccc}
                描述符表 & {\qquad} & \multicolumn{3}{c}{打开文件表} & {\qquad} & v-node 表 \\ \cline{1-1} \cline{3-5} \cline{7-7} 
                \multicolumn{1}{|c|}{...} & \multicolumn{1}{c|}{} & \multicolumn{1}{c|}{pos} & \multicolumn{1}{c|}{refcount} & \multicolumn{1}{c|}{是否被释放} & \multicolumn{1}{c|}{} & \multicolumn{1}{c|}{文件名} \\ \cline{1-1} \cline{3-5} \cline{7-7} 
                \multicolumn{1}{|c|}{3} & \multicolumn{1}{c|}{} & \multicolumn{1}{c|}{} & \multicolumn{1}{c|}{} & \multicolumn{1}{c|}{} & \multicolumn{1}{c|}{} & \multicolumn{1}{c|}{} \\ \cline{1-1} \cline{3-5} \cline{7-7} 
                \multicolumn{1}{|c|}{4} & \multicolumn{1}{c|}{} & \multicolumn{1}{c|}{} & \multicolumn{1}{c|}{} & \multicolumn{1}{c|}{} & \multicolumn{1}{c|}{} & \multicolumn{1}{c|}{} \\ \cline{1-1} \cline{3-5} \cline{7-7} 
                \multicolumn{1}{|c|}{5} & \multicolumn{1}{c|}{} & \multicolumn{1}{c|}{} & \multicolumn{1}{c|}{} & \multicolumn{1}{c|}{} & \multicolumn{1}{c|}{} & \multicolumn{1}{c|}{} \\ \cline{1-1} \cline{3-5} \cline{7-7} 
                \multicolumn{1}{|c|}{6} & \multicolumn{1}{c|}{} & \multicolumn{1}{c|}{} & \multicolumn{1}{c|}{} & \multicolumn{1}{c|}{} & \multicolumn{1}{c|}{} & \multicolumn{1}{c|}{} \\ \cline{1-1} \cline{3-5} \cline{7-7} 
                \multicolumn{1}{|c|}{7} & \multicolumn{1}{c|}{} & \multicolumn{1}{c|}{} & \multicolumn{1}{c|}{} & \multicolumn{1}{c|}{} & \multicolumn{1}{c|}{} & \multicolumn{1}{c|}{} \\ \cline{1-1} \cline{3-5} \cline{7-7} 
            \end{tabular}
        \end{table}
        \qn 请填写在第 16 行代码刚刚执行完之后,下列变量的值:
        \begin{table}[H]
            \tt
            \centering
            \begin{tabular}{|c|c|c|c|c|}
                \hline
                fd1 & fd2 & fd3 & fd4 & fd5 \\ \hline
                {\qquad \qquad} & {\qquad \qquad} & {\qquad \qquad} & {\qquad \qquad} & {\qquad \qquad} \\ \hline
            \end{tabular}
        \end{table}
        \qn 请写出程序执行完之后,\verb|ICS.txt| 文件中的内容。
        \proy{2013} 关于信号的叙述,以下不正确的是哪一个?
        \begin{choices}
            \item 在任何时刻,一种类型至多只会有一个待处理信号。
            \item 信号既可以发送给一个进程,也可以发送给一个进程组。
            \item \verb|SIGTERM| 和 \verb|SIGKILL| 信号既不能被捕获,也不能被忽略。
            \item 当进程在前台运行时,键入 \verb|Ctrl-C|,内核就会发送一个 \verb|SIGINT| 信号给这个前台进程。
        \end{choices}
        \proy{2013} 下面关于非局部跳转的叙述,正确的是哪一个?
        \begin{choices}
            \item \verb|setjmp| 可以和 \verb|siglongjmp| 使用同一个 \verb|jmp_buf| 变量。
            \item \verb|setjmp| 必须放在 \verb|main()| 函数中调用。
            \item 虽然 \verb|longjmp| 通常不会出错,但仍然需要对其返回值进行出错判断。
            \item 在同一个函数中既可以出现 \verb|setjmp|,也可以出现 \verb|longjmp|。
        \end{choices}
        \proy{2013} 考虑如下代码,假设 \verb|result.txt| 的初始内容是 \verb|123|。
        \begin{minted}[frame=single, fontsize=\small, linenos]{c}
    int main(int argc, char** argv) {
        int fd1 = open("result.txt", O_RDWR);
        char str[] = "abc";
        char c;
        write(fd1, str, 1);
        read(fd1, &c, 1);
        write(fd1, &c, 1);
        return 0;
    }
        \end{minted}
        在这段代码执行完毕之后,\verb|result.txt| 的内容是什么?假设所有的系统调用都会成功。
        \begin{choices}
            \item \verb|a22|
            \item \verb|a21|
            \item \verb|a13|
            \item \verb|abb|
        \end{choices}
        \proy{2013} 已知在适当定义 \verb|fd1, fd2, str1, str2| 后,代码段
        \begin{center}
            \verb|write(fd1, str1, strlen(str1)); write(fd2, str2, strlen(str2));|
        \end{center}
        可以在原本为空的文件 \verb|ICS.txt| 中写下字符串 \verb|I love ICS!|。那么下面确实是“适当定义”的代码有(\qquad)。
        \vspace{.25em}
        \begin{compactenum}[(1)]
            \item\ \begin{minted}[frame=single, fontsize=\small]{c}
    int fd1 = open("ICS.txt", O_RDWR);
    int fd2 = open("ICS.txt", O_RDWR);
    char *str1 = "I love ";
    char *str2 = "ICS!";
            \end{minted}
            \item\ \begin{minted}[frame=single, fontsize=\small]{c}
    int fd1 = open("ICS.txt", O_RDWR);
    int fd2 = dup(fd1);
    char *str1 = "I love ";
    char *str2 = "ICS!";
            \end{minted}
            \item\ \begin{minted}[frame=single, fontsize=\small]{c}
    int fd1 = open("ICS.txt", O_RDWR);
    int fd2 = open("ICS.txt", O_RDWR);
    char *str1 = "I love ";
    char *str2 = "I love ICS!";
            \end{minted}
            \item\ \begin{minted}[frame=single, fontsize=\small]{c}
    int fd1 = open("ICS.txt", O_RDWR);
    int fd2 = dup(fd1);
    char *str1 = "I love ";
    char *str2 = "I love ICS!";
            \end{minted}
        \end{compactenum}
        \begin{choices}
            \item (1)(4)
            \item (2)(3)
            \item (1)(2)(3)(4)
            \item 以上都不正确
        \end{choices}
        \proy{2013} 请阅读以下程序,然后回答问题,假设程序中的函数调用都可以正确执行。
        \begin{minted}[frame=single, fontsize=\small, linenos]{c}
    int main() {
        printf("A\n");
        if (fork() == 0) {
            printf("B\n");
        } else {
            printf("C\n");
            __________A__________
        }
        printf("D\n");
        exit(0);
    }
        \end{minted}
        \qn 如果程序中的 \verb|A| 位置的代码为空,列出所有可能的输出。
        \qn 如果程序中的 \verb|A| 位置的代码为 \verb|waitpid(-1, NULL, 0);|,列出所有可能的输出。
        \qn 如果程序中的 \verb|A| 位置的代码为 \verb|printf("E\n");|,列出所有可能的输出。
        \proy{2013} 请阅读以下程序,然后回答问题。假设程序中的函数调用都可以正确执行,且每条语句都是原子动作。
        \begin{minted}[frame=single, fontsize=\small, linenos]{c}
    pid_t pid;
    int even = 0; // 根据程序设计要求初始值可能不同
    int counter1 = 0;
    int counter2 = 1;
    void handler1(int sig) {
        if (even % 2 == 0) {
            printf("%d\n", counter1);
            counter1 = __________A__________;
        } else {
            printf("%d\n", counter2);
            counter2 = __________B__________;
        }
        even = __________C__________;
    }
    void handler2(int sig) {
        if (__________D__________) {
            counter1 = even * even;
        } else {
            counter2 = even * even;
        }
    }
    int main() {
        signal(SIGUSR1, handler1);
        signal(SIGUSR2, handler2);
        if ((pid = fork()) == 0) {
            while (1) {};
        }
        while (even < 20) {
            kill(pid, __________E__________);
            sleep(1);
            kill(pid, __________F__________);
            sleep(1);
            even += 2;
        }
        kill(pid, SIGKILL);
        exit(0);
    }
        \end{minted}
        \qn 完成程序,使得程序输出斐波那契数列的前 20 项,其中 $F_0=0, F_1=1, \dotsc, F_n=F_{n-1}+F_{n-2}$。如果存在对本次程序执行结果没有影响的语句,请在相应位置填写“无关”。
        {\tt
        \begin{compactenum}[A:]
            \item \verb|______________________________|
            \item \verb|______________________________|
            \item \verb|______________________________|
            \item \verb|______________________________|
            \item \verb|______________________________|
            \item \verb|______________________________|
        \end{compactenum}}
        \qn 完成程序,其中 \verb|A, B| 处保持不变,使得程序可以分别输出前几个奇数或偶数的平方和。其中若要输出奇数的平方和,\verb|even| 的初始值为 3;若要输出偶数的平方和,\verb|even| 的初始值为 2。
        {\tt
        \begin{compactenum}[A:]
            \addtocontents{enumi}{2}
            \item \verb|______________________________|
            \item \verb|______________________________|
            \item \verb|______________________________|
            \item \verb|______________________________|
        \end{compactenum}}
    \end{problems}