\chapter{异常控制流与系统 I/O}
    \begin{summary}
        \begin{compactitem}
            \item 知道控制流和异常控制流的定义,知道异常及其处理是由软硬件结合实现的。
            \item 熟悉异常处理的整个过程,包括异常表等数据结构,会以某个异常为例,复述整个异常处理的过程。会区分异常和过程调用的区别。
            \item 熟练掌握异常的四种类别及其特征、常见例子,熟悉异常结束后的返回机制。了解 x86-64 系统调用的基本范式。
            \item 熟练掌握进程的概念,理解上下文、上下文切换、调度、抢占等概念及其内容,知道内核模式和用户模式的区别。熟悉并行和并发的相同和不同。
            \item 会使用 Linux 系统下提供的进程管理函数,记忆其参数填法、返回值和行为(特别是 \verb|waitpid|),熟练掌握 \verb|fork| 函数,会通过画进程图和拓扑排序的方式进行分析。
            \item 知道 Linux 的常见信号,熟悉信号处理的过程及其细节(例如 \verb|SIGKILL, SIGSTOP| 的默认行为不能修改、待处理信号至多只有一个、进程可以发信号给自己等),知道发送信号的机制和多种 API。
            \item 理解信号处理程序和主“程序”的并发性质,熟悉编写信号处理程序中需要注意的五个要点并理解其原因(以及用阻塞避免竞争)。信号不能用来计数。
            \item 了解非本地跳转的基本机制,知道 \verb|setjmp| 的返回值不能赋值。
            \item 知道 Linux 系统的文件模型和文件类型,会用各种文件读写系统调用编写程序。熟悉文件元数据的结构和读写方法。
            \item 熟练掌握描述符表(理解与打开文件的关系)、打开文件表、v-node 表、引用计数的概念,会结合异常控制流(\verb|fork|)和重定向(\verb|dup2|)的知识分析程序。
            \item 知道标准输出和底层系统调用的区别,熟练掌握输出缓冲的机制和 \verb|fflush|。
        \end{compactitem}
    \end{summary}

    \begin{problems}
        \pro 区分异常的四种类型:填写下表(打勾)。
        \begin{table}[H]
            \centering
            \begin{tabular}{|c|c|ccc|}
                \hline
                \multirow{2}{*}{异常的种类} & \multirow{2}{*}{是否同步} & \multicolumn{3}{c|}{可能的返回行为} \\ \cline{3-5} 
                &  & \multicolumn{1}{c|}{重复当前指令} & \multicolumn{1}{c|}{执行下一条指令} & 结束进程运行 \\ \hline
                中断(interrupt) & \rule{0pt}{3ex} & \multicolumn{1}{c|}{} & \multicolumn{1}{c|}{} &  \\ \hline
                陷阱(trap) & \rule{0pt}{3ex} & \multicolumn{1}{c|}{} & \multicolumn{1}{c|}{} &  \\ \hline
                故障(fault) & \rule{0pt}{3ex} & \multicolumn{1}{c|}{} & \multicolumn{1}{c|}{} &  \\ \hline
                终止(abort) & \rule{0pt}{3ex} & \multicolumn{1}{c|}{} & \multicolumn{1}{c|}{} &  \\ \hline
            \end{tabular}
        \end{table}
        \pro 下列行为分别触发什么类型的异常?
            \qn 执行指令 \verb|mov $57, %eax; syscall|
            \qn 程序执行过程中,发现它所使用的物理内存损坏了
            \qn 程序执行过程中,试图往 \verb|main| 函数的内存中写入数据
            \qn 按下键盘
            \qn 磁盘读出了一块数据
            \qn 用 \verb|read| 函数发起磁盘读
            \qn 用户程序执行了指令 \verb|lgdt|,但是这个指令只能在内核模式下执行
        \pro 在下面的程序中,父进程和子进程的输出分别是什么?
        \begin{minted}[frame=single, fontsize=\small]{c}
    int main() {
        int a = 9;
        if (Fork() == 0)
            printf("p1: a=%d\n", a--);
        printf("p2: a=%d\n", a++);
        exit(0);
    }
        \end{minted}
        \pro 阅读下述程序。
        \begin{minted}[frame=single, fontsize=\small]{c}
    int main() {
        char c = 'A';
        printf("%c", c); fflush(stdout);
        if (fork() == 0) {
            c++;
            printf("%c", c); fflush(stdout);
        } else {
            printf("%c", c); fflush(stdout);
            fork();
        }
        c++;
        printf("%c", c); fflush(stdout);
        return 0;
    }
        \end{minted}
        假设系统调用成功,所有子进程都正常运行。判断下列哪些输出是可能的:\verb|AABBBC|、\verb|ABCABB|、\verb|ABBABC|、\verb|AACBBC|、\verb|ABABCB|、\verb|ABCBAB|。
        \pro 阅读下述程序。
        \begin{minted}[frame=single, fontsize=\small]{c}
    int main() {
        int child_status;
        char c = 'A';
        printf("%c", c); fflush(stdout);
        c++;
        if (fork() == 0) {
            printf("%c", c); fflush(stdout);
            c++;
            fork();
        } else {
            printf("%c", c); fflush(stdout);
            c += 2;
            wait(&child_status);
        } 
        printf("%c", c); fflush(stdout);
        exit(0);
    }
        \end{minted}
        假设系统调用成功,所有子进程都正常运行。判断下列哪些输出是可能的:\verb|ABBCCD|、\verb|ABBCDC|、\verb|ABBDCC|、\verb|ABDBCC|、\verb|ABCDBC|、\verb|ABCDCB|。
        \pro 阅读下述程序。
        \begin{minted}[frame=single, fontsize=\small]{c}
    void handler() {
        printf("D\n");
        return;
    }
    int main() {
        signal(SIGCHLD, handler);
        if (fork() > 0) {
            printf("A\n");
        } else {
            printf("B\n");
        } 
        printf("C\n");
        exit(0);
    }
        \end{minted}
        假设系统调用成功,所有子进程都正常运行。判断下列哪些输出是可能的:\verb|ACBC|、\verb|ABCCD|、\verb|ACBDC|、\verb|ABDCC|、\verb|BCDAC|、\verb|ABCC|。
        \pro 在某年的 ICS 课堂上,老师给同学布置了一个作业,在 Linux 上写出一份代码,使得运行它以后可输出系统能创建的进程的最大数目。下面是几位同学的答案。
            \qn Alice 同学的答案是:
            \begin{minted}[frame=single, fontsize=\small]{c}
    int main() {
        int pid;
        int count = 1;
        while ((pid = fork()) != 0) {
            // parent process
            count++;
        } 
        if (pid == 0) {
            // child process
            exit(0);
        } 
        printf("max = %d", count);
    }
            \end{minted}
            这段代码不能够正确运行,原因在于对 \verb|fork| 的返回值处理得不正确。请修改至多一处代码,使得程序正确运行。
            \qn Bob 同学对 Alice 同学修改过后的正确代码发出了疑问。Bob 同学认为,由于进程的调度时间和顺序都是不确定的,因此有的时候会调度到子进程,子进程执行 \verb|exit(0)| 以后就结束了,因此父进程可以创建更多的进程,所以 Alice 的代码输出的答案大于真实上限。请问,Bob 的说法正确吗?如果正确,请指出 Alice 应当如何修改代码,以避免 Bob 提到的问题。如果 Bob 的说法错误,请指出他错在何处。
            \qn Carol同学的答案是:
            \begin{minted}[frame=single, fontsize=\small]{c}
    int main() {
        int pid;
        int count = 1;
        while ((pid = fork()) > 0) {
            // parent process
            count++;
        } 
        if (pid == 0) {
            // child process
            while(1)
                sleep(1);
        } 
        printf("max = %d", count);
    }
            \end{minted}
            连续运行 Carol 同学的答案两次,发现结果分别如下:
            \begin{minted}[frame=single, fontsize=\small]{shell-session}
    $ ./test
    max = 1795
    $ ./test
    max = 1
            \end{minted}
                \subqn 解释为什么会发生这种情况。
                \subqn 为了解决第一次运行后的遗留问题,可以不修改代码,而直接在 Linux 终端中使用指令来解决。假设在第一次程序运行完以后,使用 \verb|ps| 指令,得到的列表前几项如下:
                \begin{minted}[frame=single, fontsize=\small]{shell-session}
    $ ./test
    max = 1795
    $ ps
    22698 pts/0     00:00:00 bash
    22725 pts/0     00:00:00 test
    22726 pts/0     00:00:00 test
    22727 pts/0     00:00:00 test
    ............
                \end{minted}
                假设 \verb|test| 程序开始运行后,没有任何新的进程被创建,并且所有进程号均按照顺序递增、逐个地分配。于是,输入下列的指令,就可以让第二次运行得到正确的结果。其中 \verb|-9| 表示 \verb|SIGKILL|。请在横线上填入正确的值。
                \begin{minted}[frame=single, fontsize=\small]{shell-session}
    $ kill -9 _________
                \end{minted}
            \qn Dave 同学修改了 Carol 同学的答案。他将 Carol 的最后一句 \verb|printf| 改为如下代码:
            \begin{minted}[frame=single, fontsize=\small]{c}
    if (pid < 0) {
        printf("max = %d", count);
        kill(0, SIGKILL);
    }
            \end{minted}
            这段代码有时无法输出任何答案。Dave 想了一想,将 \verb|printf| 中的字符串做了些修改,这样这段代码就能正确运行了。他修改了什么?
        \pro 以下问题中我们均假设缓冲区足够大,且 \verb|stdout| 只有在关闭文件、换行与 \verb|fflush| 的情况下才会刷新缓冲区;程序运行过程中的所有系统调用均成功。
        \qn 考虑下面的程序。
        \begin{minted}[frame=single, fontsize=\small]{c}
    int main() {
        printf("a");
        fork();
        printf("b");
        fork();
        printf("c");
        return 0;
    }
        \end{minted}
        写出它的一个可能的输出 \rule{3.5cm}{0.25mm},这个输出是否是唯一可能的?说明理由。
        \qn 考虑下面的程序。
        \begin{minted}[frame=single, fontsize=\small]{c}
    int main() {
        write(1, "a", 1);
        fork();
        write(1, "b", 1);
        fork();
        write(1, "c", 1);
        return 0;
    }
        \end{minted}
        以上程序的输出中有 \rule{1cm}{0.25mm} 个 \verb|a|,\rule{1cm}{0.25mm} 个 \verb|b|,\rule{1cm}{0.25mm} 个 \verb|c|;其第一个输出的字符一定是 \rule{1cm}{0.25mm}。
        \qn 考虑下面的程序。
        \begin{minted}[frame=single, fontsize=\small]{c}
    int main() {
        printf("a");
        fork();
        write(1, "b", 1);
        fork();
        write(1, "c", 1);
        return 0;
    }
        \end{minted}
        以上程序的输出中有 \rule{1cm}{0.25mm} 个 \verb|a|,\rule{1cm}{0.25mm} 个 \verb|b|,\rule{1cm}{0.25mm} 个 \verb|c|;其第一个输出的字符一定是 \rule{1cm}{0.25mm}。
        \pro 假设磁盘上有空文件 \verb|file.txt|,以下程序运行过程中的所有系统调用均成功。
        \begin{minted}[frame=single, fontsize=\small]{c}
    int main() {
        int fd1 = open("file.txt", O_RDWR|O_CREAT, S_IRUSR|S_IWUSR); 
        int fd2 = open("file.txt", O_RDWR|O_CREAT, S_IRUSR|S_IWUSR); 
        printf("%d %d\n", fd1, fd2);
        write(fd1, "123", 3);
        write(fd2, "45", 2);
        close(fd1);
        close(fd2);
        return 0;
    }
        \end{minted}
        \qn 程序关闭 \verb|fd1| 前,补全下面的 Linux 三级表结构;填写打开文件表中的 \verb|refcnt| 值。
        \begin{table}[H]
            \tt
            \centering
            \begin{tabular}{ccccc}
                文件描述符表 & {\qquad \qquad \qquad} & 打开文件表 & {\qquad \qquad \qquad} & v-node 表 \\ \cline{1-1} \cline{3-3} \cline{5-5} 
                \multicolumn{1}{|c|}{0} & \multicolumn{1}{c|}{} & \multicolumn{1}{c|}{\multirow{2}{*}{refcount = \_\_\_\_}} & \multicolumn{1}{c|}{} & \multicolumn{1}{c|}{\multirow{6}{*}{file.txt}} \\ \cline{1-1}
                \multicolumn{1}{|c|}{1} & \multicolumn{1}{c|}{} & \multicolumn{1}{c|}{} & \multicolumn{1}{c|}{} & \multicolumn{1}{c|}{} \\ \cline{1-1} \cline{3-3}
                \multicolumn{1}{|c|}{2} &  &  & \multicolumn{1}{c|}{} & \multicolumn{1}{c|}{} \\ \cline{1-1}
                \multicolumn{1}{|c|}{3} &  &  & \multicolumn{1}{c|}{} & \multicolumn{1}{c|}{} \\ \cline{1-1} \cline{3-3}
                \multicolumn{1}{|c|}{4} & \multicolumn{1}{c|}{} & \multicolumn{1}{c|}{\multirow{2}{*}{refcount = \_\_\_\_}} & \multicolumn{1}{c|}{} & \multicolumn{1}{c|}{} \\ \cline{1-1}
                \multicolumn{1}{|c|}{5} & \multicolumn{1}{c|}{} & \multicolumn{1}{c|}{} & \multicolumn{1}{c|}{} & \multicolumn{1}{c|}{} \\ \cline{1-1} \cline{3-3} \cline{5-5} 
            \end{tabular}
        \end{table}
        \qn 程序结束时,标准输出上的内容是 \rule{3.5cm}{0.25mm},\verb|file.txt| 中的内容是 \rule{3.5cm}{0.25mm}。
        \pro 假设磁盘上有空文件 \verb|file.txt|,以下程序运行过程中的所有系统调用均成功。
        \begin{minted}[frame=single, fontsize=\small]{c}
    int main() {
        int fd1 = open("file.txt", O_RDWR|O_CREAT, S_IRUSR|S_IWUSR); 
        int fd2 = open("file.txt", O_RDWR|O_CREAT, S_IRUSR|S_IWUSR); 
        dup2(fd2, fd1);
        printf("%d %d\n", fd1, fd2);
        write(fd1, "123", 3);
        write(fd2, "45", 2);
        close(fd1);
        close(fd2);
        return 0;
    }
        \end{minted}
        \qn 程序关闭 \verb|fd1| 前,补全下面的 Linux 三级表结构;填写打开文件表中的 \verb|refcnt| 值。
        \begin{table}[H]
            \tt
            \centering
            \begin{tabular}{ccccc}
                文件描述符表 & {\qquad \qquad \qquad} & 打开文件表 & {\qquad \qquad \qquad} & v-node 表 \\ \cline{1-1} \cline{3-3} \cline{5-5} 
                \multicolumn{1}{|c|}{0} & \multicolumn{1}{c|}{} & \multicolumn{1}{c|}{\multirow{2}{*}{refcount = \_\_\_\_}} & \multicolumn{1}{c|}{} & \multicolumn{1}{c|}{\multirow{6}{*}{file.txt}} \\ \cline{1-1}
                \multicolumn{1}{|c|}{1} & \multicolumn{1}{c|}{} & \multicolumn{1}{c|}{} & \multicolumn{1}{c|}{} & \multicolumn{1}{c|}{} \\ \cline{1-1} \cline{3-3}
                \multicolumn{1}{|c|}{2} &  &  & \multicolumn{1}{c|}{} & \multicolumn{1}{c|}{} \\ \cline{1-1}
                \multicolumn{1}{|c|}{3} &  &  & \multicolumn{1}{c|}{} & \multicolumn{1}{c|}{} \\ \cline{1-1} \cline{3-3}
                \multicolumn{1}{|c|}{4} & \multicolumn{1}{c|}{} & \multicolumn{1}{c|}{\multirow{2}{*}{refcount = \_\_\_\_}} & \multicolumn{1}{c|}{} & \multicolumn{1}{c|}{} \\ \cline{1-1}
                \multicolumn{1}{|c|}{5} & \multicolumn{1}{c|}{} & \multicolumn{1}{c|}{} & \multicolumn{1}{c|}{} & \multicolumn{1}{c|}{} \\ \cline{1-1} \cline{3-3} \cline{5-5} 
            \end{tabular}
        \end{table}
        \qn 程序结束时,标准输出上的内容是 \rule{3.5cm}{0.25mm},\verb|file.txt| 中的内容是 \rule{3.5cm}{0.25mm}。
        \pro 假设磁盘上有空文件 \verb|file.txt|,以下程序运行过程中的所有系统调用均成功。
        \begin{minted}[frame=single, fontsize=\small]{c}
    int main() {
        int fd1 = open("file.txt", O_RDWR|O_CREAT, S_IRUSR|S_IWUSR); 
        int fd2 = open("file.txt", O_RDWR|O_CREAT, S_IRUSR|S_IWUSR); 
        printf("%d %d\n", fd1, fd2);
        write(fd1, "123", 3);
        write(fd2, "45", 2);
        close(fd1);
        close(fd2);
        return 0;
    }
        \end{minted}
        \qn 程序关闭 \verb|fd1| 前,补全下面的 Linux 三级表结构;填写打开文件表中的 \verb|refcnt| 值。
        \begin{table}[H]
            \tt
            \centering
            \begin{tabular}{ccccc}
                文件描述符表 & {\qquad \qquad \qquad} & 打开文件表 & {\qquad \qquad \qquad} & v-node 表 \\ \cline{1-1} \cline{3-3} \cline{5-5} 
                \multicolumn{1}{|c|}{0} & \multicolumn{1}{c|}{} & \multicolumn{1}{c|}{\multirow{2}{*}{refcount = \_\_\_\_}} & \multicolumn{1}{c|}{} & \multicolumn{1}{c|}{\multirow{6}{*}{file.txt}} \\ \cline{1-1}
                \multicolumn{1}{|c|}{1} & \multicolumn{1}{c|}{} & \multicolumn{1}{c|}{} & \multicolumn{1}{c|}{} & \multicolumn{1}{c|}{} \\ \cline{1-1} \cline{3-3}
                \multicolumn{1}{|c|}{2} &  &  & \multicolumn{1}{c|}{} & \multicolumn{1}{c|}{} \\ \cline{1-1}
                \multicolumn{1}{|c|}{3} &  &  & \multicolumn{1}{c|}{} & \multicolumn{1}{c|}{} \\ \cline{1-1} \cline{3-3}
                \multicolumn{1}{|c|}{4} & \multicolumn{1}{c|}{} & \multicolumn{1}{c|}{\multirow{2}{*}{refcount = \_\_\_\_}} & \multicolumn{1}{c|}{} & \multicolumn{1}{c|}{} \\ \cline{1-1}
                \multicolumn{1}{|c|}{5} & \multicolumn{1}{c|}{} & \multicolumn{1}{c|}{} & \multicolumn{1}{c|}{} & \multicolumn{1}{c|}{} \\ \cline{1-1} \cline{3-3} \cline{5-5} 
            \end{tabular}
        \end{table}
        \qn 程序结束时,标准输出上的内容是 \rule{3.5cm}{0.25mm},\verb|file.txt| 中的内容是 \rule{3.5cm}{0.25mm}。
        \pro 假设磁盘上有空文件 \verb|file.txt|,以下程序运行过程中的所有系统调用均成功;缓冲区足够大,且 \verb|stdout| 只有在关闭文件、换行与 \verb|fflush| 的情况下才会刷新缓冲区。
        \begin{minted}[frame=single, fontsize=\small]{c}
    int main() {
        pid_t pid;
        int child_status;
        int fd1 = open("file.txt", O_RDWR|O_CREAT, S_IRUSR|S_IWUSR); 
        if ((pid = fork()) > 0) {
            // Parent
            printf("P:%d ", fd1);
            write(fd1, "123", 3);
            waitpid(pid, &child_status, 0);
        } else {
            // Child
            printf("C:%d ", fd1);
            write(fd1, "45", 2);
        } 
        close(fd1);
        return 0;
    }
        \end{minted}
        \qn 子进程关闭 \verb|fd1| 前,补全下面的 Linux 三级表结构;填写打开文件表中的 \verb|refcnt| 值。
        \begin{table}[H]
            \centering
            \begin{tabular}{ccccc}
                文件描述符表 & {\qquad \qquad \qquad} & 打开文件表 & {\qquad \qquad \qquad} & v-node 表 \\ \cline{1-1} \cline{3-3} \cline{5-5} 
                \multicolumn{1}{|c|}{Parent 3} & \multicolumn{1}{c|}{} & \multicolumn{1}{c|}{\multirow{2}{*}{refcount = \_\_\_\_}} & \multicolumn{1}{c|}{} & \multicolumn{1}{c|}{\multirow{6}{*}{file.txt}} \\ \cline{1-1}
                \multicolumn{1}{|c|}{Parent 4} & \multicolumn{1}{c|}{} & \multicolumn{1}{c|}{} & \multicolumn{1}{c|}{} & \multicolumn{1}{c|}{} \\ \cline{1-1} \cline{3-3}
                &  &  & \multicolumn{1}{c|}{} & \multicolumn{1}{c|}{} \\
                &  &  & \multicolumn{1}{c|}{} & \multicolumn{1}{c|}{} \\ \cline{1-1} \cline{3-3}
                \multicolumn{1}{|c|}{Child 3} & \multicolumn{1}{c|}{} & \multicolumn{1}{c|}{\multirow{2}{*}{refcount = \_\_\_\_}} & \multicolumn{1}{c|}{} & \multicolumn{1}{c|}{} \\ \cline{1-1}
                \multicolumn{1}{|c|}{Child 4} & \multicolumn{1}{c|}{} & \multicolumn{1}{c|}{} & \multicolumn{1}{c|}{} & \multicolumn{1}{c|}{} \\ \cline{1-1} \cline{3-3} \cline{5-5} 
            \end{tabular}
        \end{table}
        \qn 程序结束时,标准输出上的内容是 \rule{3.5cm}{0.25mm},\verb|file.txt| 中的内容是 \rule{3.5cm}{0.25mm}。(写出所有可能答案)
    \end{problems}

\chapter{异常控制流与系统 I/O{---}往年考题}