\chapter{虚拟内存}\thispagestyle{empty}
    \begin{summary}
        \begin{compactitem}
            \item 理解虚拟内存的基本原理(是磁盘的缓存)和由硬件和软件共同实现的基本机制。
            \item 熟练掌握分页式虚拟内存中,虚拟地址的结构、页表的结构、页表条目一般要包括的各字段的含义。
            \item 结合高速缓存、TLB 熟练掌握地址翻译的完整过程,会根据页表的信息模拟虚拟地址到物理地址的转换。理解多级页表的设计目的,会根据系统信息计算多级页表的详细结构并进行地址翻译。
            \item 以 Core i7/Linux 系统为例,了解实际虚拟内存的实现。了解页表自映射的机制。
            \item 熟练掌握内存映射的概念,知道共享对象和私有对象的区别,理解 COW 机制并以 \verb|fork| 函数为例掌握其具体过程。会使用 \verb|mmap| 函数进行内存映射工作,知道其各参数和返回值的用法。
            \item 综合运用高速缓存、异常控制流、系统级 I/O 和虚拟内存的知识求解综合问题。
        \end{compactitem}
    \end{summary}

    \begin{problems}
        \pro 在某一 64 位体系结构中,每页的大小为 \SI{4}{KB},采用的是三级页表,每张页表占据 1 页,页表项长度为 8 字节。则虚拟地址的位数为 \rule{2.5cm}{0.25mm}。如果要映射满 64 位的虚拟地址空间,可通过增加页表级数来解决,那么至少要增加到 \rule{2.5cm}{0.25mm} 级页表。已知这个体系结构支持多种页大小,最小的三个页大小分别是 \SI{4}{KB}、\rule{2.5cm}{0.25mm} \si{MB}、\rule{2.5cm}{0.25mm} \si{GB}。
        \sol 每张页表占据一页,所以每个页表有 $2^{12}/8=512$ 个条目。所以每一级 \verb|VPN| 为 9 位,去掉 \verb|VPO| 的 12 位,至少需要六级页表。页大小的更改则一般是基于页表级数的更改(和硬件有关),所以增加页的大小就是每次给 \verb|VPO| 添加 9 位,所以剩下的两个页大小为 $2^{12} \times 2^9=2^{21} = \SI{2}{MB}$ 和 $2^{12} \times 2^{18}=2^{30} = \SI{1}{GB}$。
        \pro Intel IA32 体系中,每页的大小为 \SI{4}{KB},采用的是二级页表,每张页表占据一页,每个页表项(PTE、PDE)的长度均为 4 字节。支持的物理地址空间为 36 位。如果采用二级翻译,那么每个 PDE 条目格式如下(物理地址 35\textasciitilde 32 位必定为 0):
        \begin{table}[H]
            \tt
            \centering
            \begin{tabular}{|c|c|c|c|c|c|c|c|c|c|c|c|c|c|}
                \hline
                & 31\textasciitilde 12 & 11 & 10 & 9 & 8 & 7 & 6 & 5 & 4 & 3 & 2 & 1 & 0 \\ \hline
                PDE & 页表的物理地址 &  &  &  &  & 0 &  &  &  &  & U/S & R/W & V \\ \hline
            \end{tabular}
        \end{table}
        每个 PTE 条目格式如下:
        \begin{table}[H]
            \tt
            \centering
            \begin{tabular}{|c|c|c|c|c|c|c|c|c|c|c|c|c|c|}
                \hline
                & 31\textasciitilde 12 & 11 & 10 & 9 & 8 & 7 & 6 & 5 & 4 & 3 & 2 & 1 & 0 \\ \hline
                PTE & 虚拟页的物理地址 &  &  &  &  & 0 &  &  &  &  & U/S & R/W & V \\ \hline
            \end{tabular}
        \end{table}
        如果采用一级页表翻译(大页模式),那么每个 PDE 的条目如下:
        \begin{table}[H]
            \tt
            \centering
            \begin{tabular}{|c|c|c|c|c|c|c|c|c|c|c|c|c|}
                \hline
                & 31\textasciitilde 22 & 21\textasciitilde17 & 16\textasciitilde13 & 12\textasciitilde 8 & 7 & 6 & 5 & 4 & 3 & 2 & 1 & 0 \\ \hline
                \multirow{2}{*}{PDE} & 虚拟页的 & \multirow{2}{*}{0} & 物理地址的 & \multirow{2}{*}{\textasciitilde} & \multirow{2}{*}{1} & \multirow{2}{*}{} & \multirow{2}{*}{} & \multirow{2}{*}{} & \multirow{2}{*}{} & \multirow{2}{*}{U/S} & \multirow{2}{*}{R/W} & \multirow{2}{*}{V} \\
                & 物理地址 &  & 35\textasciitilde 32 位 &  &  &  &  &  &  &  &  &  \\ \hline
            \end{tabular}
        \end{table}

        上表中的最低位均为第 0 位。部分位的意义如下:
        \begin{asparadesc}
            \item[\texttt{V}] 当前条目是否有效(指向的页在物理内存中);
            \item[\texttt{R/W}] 指向的区域是否可写。只有两级页表均为 1 的时候,该虚拟内存地址才可以写。
            \item[\texttt{U/S}] 指向的区域用户程序是否可访问。只有两级页表均为 1 的时候,该虚拟内存地址才可以被用户程序访问。
        \end{asparadesc}

        某一时刻,一级页表的起始地址为 \verb|0x00C188000|。部分物理内存中的数据如下:
        \begin{table}[H]
            \tt
            \centering
            \begin{tabular}{|c|c|c|c|c|c|c|c|}
                \hline
                物理地址 & 内容 & 物理地址 & 内容 & 物理地址 & 内容 & 物理地址 & 内容 \\ \hline
                00C188000 & 63 & 00C188001 & A0 & 00C188002 & 67 & 00C188003 & C0 \\ \hline
                00C188004 & 0D & 00C188005 & A0 & 00C188006 & F0 & 00C188007 & A5 \\ \hline
                00C188008 & 67 & 00C188009 & A0 & 00C18800A & 32 & 00C18800B & 0D \\ \hline
                00C1880C0 & 67 & 00C1880C1 & 30 & 00C1880C2 & 88 & 00C1880C3 & C1 \\ \hline
                00C188300 & E7 & 00C188301 & 00 & 00C188302 & 80 & 00C188303 & 9A \\ \hline
                \textbf{00C188C00} & \textbf{65} & \textbf{00C1880C1} & \textbf{80} & \textbf{00C1880C2} & \textbf{18} & \textbf{00C1880C3} & \textbf{0C} \\ \hline
                00D32A294 & 67 & 00D32A295 & C0 & 00D32A296 & 83 & 00D32A297 & 67 \\ \hline
                00D32A298 & C0 & 00D32A299 & C0 & 00D32A29A & BB & 00D32A29B & DC \\ \hline
                00D32AA5C & 67 & 00D32AA5D & C0 & 00D32AA5E & 83 & 00D32AA5F & 9A \\ \hline
                00DA0C294 & 45 & 00DA0C295 & 82 & 00DA0C296 & 77 & 00DA0C297 & 67 \\ \hline
                00DA0C298 & 67 & 00DA0C299 & 83 & 00DA0C29A & 29 & 00DA0C29B & 44 \\ \hline
                00DA0CA5C & 0 & 00DA0CA5D & 9A & 00DA0CA5E & 88 & 00DA0CA5F & EF \\ \hline
            \end{tabular}
        \end{table}
        不采用 TLB 加速翻译。
        \qn 现在需要访问虚拟内存地址 \verb|0x00A97088|。
            \subqn 将该地址拆成 \verb|VPN1 + VPN2 + VPO| 的形式。
            \subqn 对应的 PDE 条目的物理地址是 \rule{2.5cm}{0.25mm},读出第二级页表的起始地址 \rule{2.5cm}{0.25mm},PTE 条目的起始地址为 \rule{2.5cm}{0.25mm},因此翻译得到的物理地址为 \rule{2.5cm}{0.25mm}。
            \subqn 用户模式能否访问该地址?能否写该地址?
        \qn 现要访问虚拟内存地址 \verb|0x3003C088|,则最终翻译得到的物理地址为 \rule{2.5cm}{0.25mm}。
        \qn 下列 IA32 汇编代码执行结束以后,\verb|%eax| 的值是多少?假设一开始 \verb|%ebx| 的值为 \verb|0x00A97088|,\verb|%edx| 的值为 \verb|0x3003C088|。
        \begin{minted}[frame=single, fontsize=\small]{text}
    movl   $0xC, (%ebx)
    movl   $0x9, (%edx)
    movl   (%ebx), %eax
    xorl   (%edx), %eax
        \end{minted}
        \qn 下列 IA32 汇编代码执行结束以后,\verb|%eax| 的值是多少?
        \begin{minted}[frame=single, fontsize=\small]{text}
    movl   0xC0002A5C, %eax
        \end{minted}
        重点关注加粗的内存。以此为启发,写出读出第一级页表中 \verb|VPN1 = 2| 的条目的代码
        \begin{minted}[frame=single, fontsize=\small]{text}
    movl   ____________________, %eax
        \end{minted}
        \sol 首先根据开头指明的每张页表占一页知道两级 \verb|VPN| 都是 10 位,\verb|VPO| 为 10 位。
        
        第一问。拆分时注意,尽量不要增加二进制和十六进制转换的工作,因此只需要拆开 \verb|A|,也就是说,马上可看出 \verb|VPO = 0x088| 和 \verb|VPN2 = 0x297|。由于这里高位都是 0,所以 \verb|VPN1| 立刻看出是 \verb|0x2|。所以 \verb|PDE| 的地址是 \verb|0x00C188000 + 4 * 0x2 = 0x00C188008|,该位置的内存内容是(注意用小端法)\verb|0D32A067|。根据题给信息,最低的 8 位中,大页位为零,所以是两级翻译。故二级页表的起始地址是 \verb|0x0D32A000|(留意每个页都是 \SI{4}{KB} 对齐的,页表项中就只需要记录物理页号,这是我们希望页表都恰占一页的原因)。结合 \verb|VPN2|,二级页表项的地址为 \verb|0x0D32A000 + 4 * 0x297 = 0x0D32AA5C|,读出此处的内存为 \verb|9A83C067|,所以最终的物理页号是 \verb|9A83C|,和 \verb|VPO| 拼出物理地址为 \verb|0x9A83C088|。鉴于其低 8 位是 \verb|0x67 = 0110 0111|,所以不发生缺页,用户态程序可读也可写。

        第二问。拆分得到的 \verb|VPN1 = 0xC0, VPN2 = 0x03C, VPO = 0x088|。\verb|PDE| 的地址是 \verb|0x00C188000 + 4 * 0xC0 = 0xC188300|,读出一级页表项的内容是 \verb|9A8000E7|。注意大页位是 1,所以是大页翻译,物理地址的 \texttt{31\textasciitilde 22} 位是 \verb|1001 1010 10|,其最高 4 位是 \verb|0000|,所以其物理地址的 \texttt{35\textasciitilde 22} 位是 \verb|0x9A83|,和剩下 22 位拼在一起的到 \verb|0x9A83C088|。

        由前两问立刻得到第三问结果为 0,因为这两个虚拟地址对应的物理内存为同一块内存。

        第四问涉及页表自映射,请回忆课上讲的内容。利用地址 \verb|C0002A5C|算出 \verb|VPN1 = 0x300| 以及 \verb|VPN2 = 0x2|,因此 \verb|PDE| 条目的物理地址为 \verb|0x00C188C00|,在物理内存中读出第二级页表的起始地址为 \verb|0x00C188000|(这里就发生了一次自映射,指向页表开头),于是计算出 \verb|PTE| 条目地址为 \verb|0x00C188008|,读出物理页号为 \verb|0x00D32A000|,物理地址为 \verb|0x00D32AA5C|,所以 \verb|%eax| 的值为 \verb|0x9A83C067|。为了指向第一级页表的第三个条目(\verb|VPN1 = 0x2|),也就希望二级页表映射以后映射到的物理页就是第一级页表(即自映射两次)。因此前两次映射都应当映射到页表自己。注意到如果 \verb|VPN1 = 0x300| 的话,那么第一级映射就映射到自己开头;所以如果还有 \verb|VPN2 = 0x300| 的话,第二 级映射也还是映射到自己开头。又对应的 \verb|VPO = 0x8|,所以拼接得到虚拟地址 \verb|0xC0300008|。
        \pro 有下列代码:
        \begin{minted}[frame=single, fontsize=\small, linenos]{c}
    int main() {
        pid_t pid;
        int child_status;
        long* f = mmap(NULL, 8, PROT_READ | PROT_WRITE,
                _____X_____ | MAP_ANONYMOUS, -1, 0);
        *f = 0;
        if ((pid = fork()) > 0) {
            waitpid(pid, &child_status, 0);
            *f = *f + 1;
            printf("Parent: %ld\n", *f);
        } else {
            *f = *f + 1;
            printf("Child: %ld\n",*f);
        } 
        return 0;
    }
        \end{minted}
        当 \verb|X| 处为 \verb|MAP_PRIVATE| 时,标准输出上的两个整数是什么?如果是 \verb|MAP_SHARED| 呢?
        \sol 若映射为私有的,则写时复制,因此答案相同均为 1;若映射为共享的,则两个进程共享更新,所以第一次子进程打印 1,然后父进程打印 2。
        \pro 有下列 C 程序。其中 \verb|sleep(3)| 是为了让 \verb|fork| 以后子进程先运行。\verb|hello.txt| 的初始内容为字符串 \verb|ABCDEFG|,紧接着为 \verb|\0|。Linux 采用写时复制(Copy-on-Write)技术。
        \begin{minted}[frame=single, fontsize=\small, linenos]{c}
    char* f;
    int count = 0, parent = 0, child = 0, done = 0;
    void handler1() {
        if (count >= 4) {
            done = 1;
            return;
        } 
        f[count] = '0' + count;
        count++;
        kill(parent,SIGUSR2);
    }
    void handler2() {
        _____Y_____
        write(STDOUT_FILENO, f, 7);
        write(STDOUT_FILENO, "\n", 1);
        kill(child,SIGUSR1);
    }
    int main() {
        signal(SIGUSR1, handler1);
        signal(SIGUSR2, handler2);
        int child_status;
        parent = getpid();
        int fd = open("hello.txt", O_RDWR);
        if ((child = fork()) > 0) { // Parent
            f = mmap(NULL, 8, PROT_READ | PROT_WRITE, MAP_PRIVATE, fd, 0);
            sleep(3);
            kill(child, SIGUSR1);
            waitpid(child, &child_status, 0);
        } else { // Child
            f = mmap(NULL, 8, PROT_READ | PROT_WRITE, MAP_SHARED, fd, 0);
            while (done == 0);
        }
        return 0;
    }
        \end{minted}
            \qn 若 \verb|Y| 处为空,程序运行结束以后,标准输出上的内容是什么(四行)?\verb|hello.txt| 中的内容是什么?
            \qn 若 \verb|Y| 处为 \verb|f[6] = 'X';|,程序运行结束以后,标准输出上的内容是什么(四行)?\verb|hello.txt| 中的内容是什么?
        \sol 首先阅读程序。父子进程都进行了内存映射。由于父子进程 \verb|fork()| 结束之后状态是一样的,因此 \verb|mmap| 映射到的虚拟地址空间相同;又它们的页表内容一样,所以 \verb|f| 对应同一块物理内存(内容是 \verb|hello.txt| 的前 8 个字符)。除非发生写时复制,否则二者将共享这块内存。其中,父进程中,\verb|hello.txt| 映射为私有的可读写;子进程中,\verb|hello.txt| 映射为共享的可读写。
        
        完成 \verb|fork()| 之后,子进程收到信号,进入信号处理程序 1,然后父子进程轮流执行信号处理程序。子进程会依次在 \verb|f| 对应的内存写 \verb|0123|;每写一次,都会让父进程执行信号处理程序 2,父进程打印出 \verb|f| 的内存映射中的内容。

        当 \verb|Y| 为空时,映射为私有的父进程从来不写内存,因此它会打印出子进程修改的内容。所以每次的标准输出分别是 \verb|0BCDEFG, 01CDEFG, 012DEFG, 0123EFG|,文件中的内容是 \verb|0123EFG|。
        
        而当父进程尝试在标记为私有的内存进行写时(\verb|Y| 为 \verb|f[6] = 'X'|)时,会触发 COW 机制,内容 \verb|0BCDEFG| 会被复制,然后被修改为 \verb|0BCDEFX| 打印出来(\verb|mmap(...MAP_PRIVATE...)| will protect you from changes made to the file on disk)。由于后来两个进程关于 \verb|f| 的物理内存是独立的,所以每行打印出的都是 \verb|0BCDEFX|。留意到这段内存映射被复制之后,父进程对其的更改是不会反映到文件中的(否则导致一对多),所以文件内容还是反映为子进程的修改,即 \verb|0123EFG|。
        \pro 有如下 C 程序:
        \begin{minted}[frame=single, fontsize=\small, linenos]{c}
    int main() {
        char* hello;
        char* bye;
        int fd1 = open("hello.txt", O_RDWR);
        int fd2 = open("bye.txt", O_RDWR);
        hello = mmap(NULL, 16, PROT_READ, MAP_SHARED, fd1, 0);
        bye = mmap(NULL, 16, PROT_READ | PROT_WRITE,
                MAP_SHARED, fd2, 0);
        for (int i = 0; i < 8; i++)
            bye[i] = toupper(hello[i]);
        /************A**************/
        munmap(hello, 16);
        munmap(bye, 16);
        return 0;
    }
        \end{minted}
        某次运行,该程序对应进程的 \verb|PID = 2333|。此时代码运行到 \verb|A| 处时,\verb|/proc/2333/maps| 中的内容如下:
        \begin{minted}[frame=single, fontsize=\small]{text}
ADDRESS                    PERM   PATH 
00400000-00401000          (1)    /home/larry/map/pstate 
00600000-00601000          r--p   /home/larry/map/pstate 
00601000-00602000          rw-p   /home/larry/map/pstate 
7fb596fb5000-7fb59719c000  r-xp   /lib/x86_64-linux-gnu/libc-2.27.so
7fb59719c000-7fb59739c000  ---p   /lib/x86_64-linux-gnu/libc-2.27.so
7fb59739c000-7fb5973a0000  r--p   /lib/x86_64-linux-gnu/libc-2.27.so
7fb5973a0000-7fb5973a2000  rw-p   /lib/x86_64-linux-gnu/libc-2.27.so 
7fb5973a2000-7fb5973a6000  rw-p  
7fb5973a6000-7fb5973cd000  r-xp   /lib/x86_64-linux-gnu/(4)-2.27.so 
7fb5975b1000-7fb5975b3000  rw-p  
7fb5975cb000-7fb5975cc000  (2)    /home/larry/map/bye.txt 
7fb5975cc000-7fb5975cd000  (3)    /home/larry/map/hello.txt 
7fb5975cd000-7fb5975ce000  r--p   /lib/x86_64-linux-gnu/(4)-2.27.so
7fb5975ce000-7fb5975cf000  rw-p   /lib/x86_64-linux-gnu/(4)-2.27.so 
7fb5975cf000-7fb5975d0000  rw-p  
7ffe671ef000-7ffe67210000  rw-p   [(5)] 
7ffe673cc000-7ffe673cf000  r—p    [vvar]
7ffe673cf000-7ffe673d1000  r-xp   [vdso] 
ffffffffff600000-ffffffffff601000 r-xp [vsyscall]
        \end{minted}
        上面内容中,\verb|PERM| 下条目有四位。前三位是 \verb|r=readable|、\verb|w=writeable|、\verb|x=executable|,如果是 \verb|-| 则表示没有这一权限。第四位是 \verb|s=shared|、\verb|p=private|,表示内存映射是共享的还是私有的。
            \qn 填写上面(1)\textasciitilde(5)空的内容。
            \qn \verb|7fb5973a0000| 对应的页在页表中被标为了只读。对该页进行写操作会发生 \verb|SIGSEGV| 吗?说明理由。
        \sol Linux 加载可执行文件时代码段一般都放在 \verb|0x400000| 所以(1)应填写 \verb|r-xp|。(2)、(3)是程序创建的内存映射,其中 \verb|bye.txt| 是共享可读写,所以(2)是 \verb|rw-s|,而 \verb|hello.txt| 是共享只读,所以(3)是 \verb|r--s|。(4)是动态链接器即 \verb|ld.so|。结合加载可执行文件的内存布局知道(5)应该是栈,即写 \verb|stack|。

        注意到题目中给出的内核内存映射中,\verb|7fb5973a0000| 处是标记为可写的。这说明该段发生写会触发 COW 机制:页表项表明不可写,触发故障,异常处理程序完成复制,然后再写。
    \end{problems}

\chapter{虚拟内存{---}往年考题}\thispagestyle{empty}
    \begin{problems}
        \proy{2019} 考虑本课程介绍的 Intel x86-64 存储系统中的某一个进程。它从用户态切换至内核态时,TLB \rule{2.5cm}{0.25mm},高速缓存 \rule{2.5cm}{0.25mm};当发生进程上下文切换时, TLB \rule{2.5cm}{0.25mm},高速缓存 \rule{2.5cm}{0.25mm}。
        \begin{choices}
            \item 不必刷新 \quad 不必刷新 \quad 需要刷新 \quad 不必刷新
            \item 不必刷新 \quad 不必刷新 \quad 不必刷新 \quad 不必刷新
            \item 需要刷新 \quad 不必刷新 \quad 需要刷新 \quad 需要刷新
            \item 需要刷新 \quad 不必刷新 \quad 不必刷新 \quad 需要刷新
        \end{choices}
        \sol A。高速缓存是物理寻址的,所以一定不用刷新。用户态和内核态的转换不会改变内存映射,而上下文切换会改变虚拟内存到物理地址的对应关系,所以 TLB 在上下文切换时要刷新。
        \proy{2019} 已知某系统页面大小为 \SI{2}{KB},页表项为 8 字节,采用多层分页策略映射 48 位虚拟地址空间。若限定最高层页表占 1 页,则它可以采用多少层的分页策略?
        \proy{2019} 在一个具有 TLB 和高速缓存的系统中,假设地址翻译使用四级页表来进行,且不发生缺页异常,那么在 CPU 访问某个虚拟内存地址的过程中,至少会访问 \rule{2.5cm}{0.25mm} 次物理内存,至多会访问 \rule{2.5cm}{0.25mm} 次物理内存。
        \proy{2019, 有改动} 某 32 位机器有 28 位虚拟地址空间,物理地址为 24 位。采用二级页表,一级页表中有 64 个 PTE,二级页表中有 1024 个 PTE,页表页均按 \SI{4}{KB} 对齐。PTE 最高一位是有效位,页表项中低 12\textasciitilde 23 位为物理页号(最低位为 第 0 位)。系统中没有 TLB 和高速缓存。某人在该机器上执行了如下代码:
        \begin{minted}[frame=single, fontsize=\small, linenos]{c}
    int* a = calloc(10000, sizeof(int));
    int* b = a + 5000;
    fork();
    *b = 0x80C3F110;
        \end{minted}
        设编译后 \verb|b| 的值保存在寄存器中。
        \qn 该机器页面大小为 \rule{2.5cm}{0.25mm} 字节,\verb|VPN1| 长度为 \rule{2.5cm}{0.25mm},\verb|VPN2| 长度为 \rule{2.5cm}{0.25mm}。
        \qn 已知父进程执行第四行代码时进行了若干次物理内存的读写操作,其中第一次读内存操作从地址 \verb|0x67F0E8| 读出 \verb|0x80AA32C4|,另外有两次将 \verb|0x80C3F110| 写入内存的操作,写入的地址分别是 \verb|0xC3F50D| 和 \verb|0xAA3AD0|,但写入顺序并不清楚。据此回答(地址都用 16 进制表示):程序结束瞬间变量 \verb|b| 中保存的值是\rule{2.5cm}{0.25mm}。在(父进程)解析 \verb|*b| 的过程中,一级页表的起始地址为 \rule{2.5cm}{0.25mm};二级页表的起始地址为 \rule{2.5cm}{0.25mm};物理页面的起始地址为 \rule{2.5cm}{0.25mm}。
        \sol 解答之前先指出,此题原题条件不足,数据错误,叙述令人十分摸不着头脑,属于质量非常糟糕的题目。

        易做出(1),\verb|VPN1| 和 \verb|VPN2| 分别占 6、10 位,所以 \verb|VPO| 为 12 位,页面大小为 4096 字节。
        
        现在考虑父进程执行 \verb|*b| 的写时的读写问题。在解析 \verb|*b| 时,第一步当然是找出一级页表项,因此从地址 \verb|0x67F0E8| 读出 \verb|0x80AA32C4| 就是从一级页表中,物理地址 \verb|0x67F0E8| 的位置,读出一级页表项的内容 \verb|0x80AA32C4|。根据题目对 PTE 结构的描述,我们知道 \verb|0xAA3| 是二级页表页的页号,即 \verb|0xAA3000| 是二级页表的起始物理地址。由于一级页表是 \SI{4}{KB} 对齐的,所以一级页表项的物理地址就是 \verb|0x67F000|。

        为什么会写两次?显然,我们知道 COW 机制在起作用,这段内存首先要被复制。\verb|0x80C3F110| 这个数字当然是精心设计的,必然暗藏玄机。首先,一次写肯定是写 \verb|*b|,那么另一次写呢?注意这次写也是父进程,因此和子进程无关。由于 COW 后内存映射需要修改,结合后面也有类似于 \verb|0xC3F...| 的物理地址,我们可以推定这个实际上就是 COW 后,\verb|b| 对应的二级页表项。这样我们马上知道,其二级页表项的物理地址在 \verb|0xAA3AD0|,而 \verb|b| 指向的真实物理地址是 \verb|0xC3F50D|(页号 \verb|0xC3F|)。这时我们就可以计算出 \verb|b| 的值(即虚拟地址)了。一级页表项的偏移是 \verb|(0x67F0E8 - 0x67F000) / 4 = 0x3A|,二级页表项的偏移是 \verb|(0xAA3AD0 - 0xAA3000) / 4 = 0x2B4|,\verb|VPO = 0x50D|,所以虚拟地址是它们的拼接。注意不是直接将 16 进制拼接,虚拟页号为 \verb|11 1010 10 1011 0100 = 0xEAB4|,所以 \verb|b = 0xEAB450D|。

        此题页表项中的最低 12 位是何含义,属于未解之谜。
        \proy{2018} 下列与虚拟内存有关的说法中哪些是不对的?
        \begin{choices}
            \item 操作系统为每个进程提供一个独立的页表,用于将其虚拟地址空间映射到物理地址空间。
            \item MMU 使用页表进行地址翻译时,虚拟地址的虚拟页面偏移与物理地址的物理页面偏移是相同的。
            \item 若某个进程的工作集大小超出了物理内存的大小,则可能出现抖动现象。
            \item 动态内存分配管理中可以采用双向链表组织空闲块,使得首次适配的分配与释放的时间均和空闲块数量成线性关系。
        \end{choices}
        \proy{2018} 假定整型变量 \verb|A| 的虚拟地址为 \verb|0x12345cf0|,另一整型变量 \verb|B| 的虚拟地址为 \verb|0x12345d98|,假定一个页的长度为 \verb|0x1000| 字节,那么 \verb|A| 的物理地址数值和 \verb|B| 的物理地址数值之关系应该为:
        \begin{choices}
            \item \verb|A| 的物理地址数值始终大于 \verb|B| 的物理地址数值
            \item \verb|A| 的物理地址数值始终小于 \verb|B| 的物理地址数值
            \item \verb|A| 的物理地址数值和 \verb|B| 的物理地址数值大小取决于动态内存分配策略
            \item 无论如何都无法判定两个物理地址数值的大小关系
        \end{choices}
        \sol B,因为两个变量的虚拟页号相同,所以一定在同一个物理页中,先后顺序取决于 \verb|PPO = VPO|。
        \proy{2018} 虚拟内存中两层页表和单层页表相比,最主要的优势在于:
        \begin{choices}
            \item 能够达到更快的地址翻译速度
            \item 能够提供更加精细的保护措施
            \item 能够充分利用代码的空间局部性
            \item 能够充分利用稀疏的内存使用模式
        \end{choices}
        \proy{2018} 回答下列问题。
        \qn 在进行地址翻译的过程中,操作系统需要借助页表。考虑一个 32 位的系统,页大小是 \SI{4}{KB},页表项的大小是 4 字节。如果不使用多级页表,则常驻内存的页表一共需要 \rule{2.5cm}{0.25mm} 页。

        再考虑下图示出的物理内存分配情况。假设系统使用二级页表,则已经显示的区域的页表需要占据 \rule{2.5cm}{0.25mm} 页。
        \begin{table}[H]
            \tt
            \centering
            \begin{tabular}{|c|c|}
                \hline
                {\quad VP0 \quad} & \multirow{6}{*}{\quad 已分配页 \quad} \\ \cline{1-1}
                ... &  \\ \cline{1-1}
                VP1023 &  \\ \cline{1-1}
                VP1024 &  \\ \cline{1-1}
                ... &  \\ \cline{1-1}
                VP2047 &  \\ \hline
                \multirow{2}{*}{gap} & \multirow{4}{*}{未分配页} \\
                &  \\ \cline{1-1}
                \multirow{2}{*}{1023 个未分配页} &  \\
                &  \\ \hline
                VP10239 & \multirow{4}{*}{已分配页} \\ \cline{1-1}
                ... &  \\ \cline{1-1}
                VP11263 &  \\ \hline
                \end{tabular}
        \end{table}
        \qn IA32 体系采用小端法和二级页表。其中两级页表大小相同,页大小均为 \SI{4}{KB},结构也相同。TLB 采用直接映射。TLB 和页表每一项的前 20 位为物理地址,后 7 位含义如下图所示。为简便起见,假设 TLB 和页表每一项的后 8\textasciitilde 12 位都是 0 且不会被改变。例如后 7 位 值为 \verb|0x27| 则表示可读写。
        \begin{table}[H]
            \tt
            \centering
            \begin{tabular}{cccccccc}
                {\footnotesize 7} & {\footnotesize 6} & {\footnotesize 5} & {\footnotesize 4} & {\footnotesize 3} & {\footnotesize 2} & {\footnotesize 1} & {\footnotesize 0} \\ \hline
                \multicolumn{1}{|c|}{\cellcolor{gray!50}} & \multicolumn{1}{c|}{D} & \multicolumn{1}{c|}{A} & \multicolumn{1}{c|}{CD} & \multicolumn{1}{c|}{WT} & \multicolumn{1}{c|}{U} & \multicolumn{1}{c|}{W} & \multicolumn{1}{c|}{P} \\ \hline
                & {\footnotesize dirty} & {\footnotesize accessed} & {\footnotesize cache-disabled} & {\footnotesize write-through} & {\footnotesize user} & {\footnotesize writable} & {\footnotesize present}
            \end{tabular}
        \end{table}
        当系统运行到某一时刻时,TLB 内容如下:
        \begin{table}[H]
            \tt
            \centering
            \begin{tabular}{|c|c|c|c|}
                \hline
                索引 & TLB 标记 & 内容 & 有效位 \\ \hline
                0 & 0x04013 & 0x3312D027 & 1 \\ \hline
                1 & 0x01000 & 0x24833020 & 0 \\ \hline
                2 & 0x005AE & 0x00055004 & 1 \\ \hline
                3 & 0x00402 & 0x24AEE020 & 0 \\ \hline
                4 & 0x0AA00 & 0x0005505C & 0 \\ \hline
                5 & 0x0000A & 0x29DEE000 & 1 \\ \hline
                6 & 0x1AE82 & 0x00A23027 & 1 \\ \hline
                7 & 0x28DFC & 0x00023000 & 0 \\ \hline
            \end{tabular}
        \end{table}
        一级页表的基地址为 \verb|0x0C23B00|,物理内存中的部分内容如下:
        \begin{table}[H]
            \tt
            \centering
            \begin{tabular}{|c|c|c|c|c|c|c|c|}
                \hline
                地址 & 内容 & 地址 & 内容 & 地址 & 内容 & 地址 & 内容 \\ \hline
                00023000 & E0 & 00023001 & BE & 00023002 & EF & 00023003 & BE \\ \hline
                00023120 & 83 & 00023121 & C8 & 00023122 & FD & 00023123 & 12 \\ \hline
                00023200 & 23 & 00023201 & FD & 00023202 & BC & 00023203 & DE \\ \hline
                00023320 & 33 & 00023321 & 29 & 00023322 & E5 & 00023323 & D2 \\ \hline
                0005545C & 97 & 0005545D & C2 & 0005545E & 7B & 0005545F & 45 \\ \hline
                00055464 & 97 & 00055465 & D2 & 00055466 & 7B & 00055467 & 45 \\ \hline
                0C23B020 & 27 & 0C23B021 & EB & 0C23B022 & AE & 0C23B023 & 24 \\ \hline
                0C23B040 & 27 & 0C23B041 & 40 & 0C23B042 & DE & 0C23B043 & 29 \\ \hline
                0C23B080 & 05 & 0C23B081 & 5D & 0C23B082 & 05 & 0C23B083 & 00 \\ \hline
                2314D200 & 23 & 2314D201 & 12 & 2314D202 & DC & 2314D203 & 0F \\ \hline
                2314D220 & A9 & 2314D221 & 45 & 2314D222 & 13 & 2314D223 & D2 \\ \hline
                29DE404C & 27 & 29DE404D & 42 & 29DE404E & BA & 29DE404F & 00 \\ \hline
                29DE4400 & D0 & 29DE4401 & 5C & 29DE4402 & B4 & 29DE4403 & 2A \\ \hline
            \end{tabular}
        \end{table}
        此刻,系统先后试图对两个已经缓存在 cache 中的内存地址进行写操作,请分析完成写之后系统的状态(写的地址和上表中的内存地址无交集),完成下面的填空。若不需要某次访问或者缺少所需信息,请填写 X。
        第一次向地址 \verb|0xD7416560| 写入内容,TLB 索引为 \rule{2.5cm}{0.25mm},该项 TLB 内容为 \rule{2.5cm}{0.25mm},二级页表页表项地址为 \rule{2.5cm}{0.25mm},物理地址为 \rule{2.5cm}{0.25mm}。第二次向地址 \verb|0x0401369B| 写入内容,TLB 索引为 \rule{2.5cm}{0.25mm},完成写之后该项 TLB 内容为 \rule{2.5cm}{0.25mm},二级页表页表项地址为 \rule{2.5cm}{0.25mm},物理地址为 \rule{2.5cm}{0.25mm}。
        \qn 本学期的 fork bomb 作业中,大家曾用 \verb|fork()| 逼近系统的进程数量上限。下面有一个类似的程序,请仔细阅读程序并填空。
        \begin{minted}[frame=single, fontsize=\small, linenos]{c}
    #define N 4
    int main() {
        volatile int pid, cnt = 1;
        for (int i = 0; i < N; i++) {
            if ((pid = fork()) > 0) {
                cnt++;
            }
        }
        while (wait(NULL) > 0);
        return 0;
    }
        \end{minted}
        整个过程中,变量 \verb|cnt| 最大值是 \rule{2.5cm}{0.25mm}。假设所有的数据都已经在内存中,\verb|pid| 和 \verb|cnt| 在同一个物理页中。从第一个进程开始执行 \verb|for| 语句起计算,此过程对 \verb|cnt| 的操作至少会导致页表中发生 \rule{2.5cm}{0.25mm} 次如下事件:虚拟页所对应的物理页被修改。
        \sol (1)问计算页表大小是容易的。一共有 $2^{20}$ 个页面,每个页面都需要一个页表项,所以答案为 $2^{20} \times 2^2=2^{22} = \SI{4}{MB}$。在二级页表下,可以算出 \verb|VPN1, VPN2| 都有 10 位,每个页表恰好 \SI{4}{KB},有 1024 个页表项(即对应 1024 个页面)。所以图中 \verb|VP0..VP2047| 需要两张二级页表,\verb|VP10239| 需要独占一张二级页表(注意对齐),\verb|VP10240..11263| 需要一张二级页表,加上一级页表一直是需要的,所以一共需要 5 页。

        (2)问需要结合 TLB 解决问题,留意 TLB 中存储的都是最后一级页表项,通过 \verb|VPN| 就可以查出。由于 TLB 有 8 个项,且是直接映射,所以索引占 3 位,剩下 17 位都是 TLB 标记。对于第一次访问,其 \verb|VPN| 为 \verb|0xD7416 = 1101 0111 0100 0001 0110|,于是 \verb|TLBI = 6, TLBT = 1 1010 1110 1000 0010 = 0x1AE82|,恰好命中 TLB,所以页表项是 \verb|0x00A32027|,物理页号是 \verb|0x00A32|,所以物理地址是 \verb|0x00A32560|。这一过程中不需要访问二级页表项所在的内存位置。
        
        下一个内存访问的 \verb|VPN| 为,同理 \verb|TLBI = 3, TLBT = 0x00802|,没有命中 TLB,所以需要经过两级翻译。\verb|VPN1 = 0x10, VPN2 = 0x13, VPO = 69B|。一级页表项的地址为 \verb|0x0C23B00 + 4 * 0x10 = 0x0C23B40|,页表项内容为 \verb|29DE4027|,用户态程序可以读写。进一步得到二级页表项的地址是 \verb|0x29DE4000 + 4 * 0x13 = 0x29DE404C|,页表项内容为 \verb|00BA4227|,即物理页号是 \verb|0x00BA4|,物理地址是 \verb|0x00BA469B|。需要注意的是,完成写之后,TLB 会被替换,而且因为写过,dirty 位会变为 1,所以 TLB 条目的内容是 \verb|0x00BA4267|。

        (3)也是简单的,\verb|cnt| 代表了进程树中进程所处的深度,所以最大值是 5。每次产生的子进程时,相对于产生进程的父进程都要修改 \verb|cnt|,会触发 COW,也就是虚拟页对应的物理页(内存映射)被修改。因为一共新产生了 15 个进程,所以答案为 15。
        \proy{2016} 在 Core i7 中,以下哪个页表项属于 4 级页表项,不属于 1 级页表项?
        \begin{choices}
            \item \verb|G|
            \item \verb|D|
            \item \verb|XD|
            \item \verb|U/S|
        \end{choices}
        \sol AB,因为只有最后一级页表项有 dirty 的问题。当然,就事实而言,A 也是正确选项。
        \proy{2016} 在 Core i7 中,关于虚拟地址和物理地址的说法,不正确的是:
        \begin{choices}
            \item \verb|VPO = CI + CO|
            \item \verb|PPN = TLBT + TLBI|
            \item \verb|VPN1 = VPN2 = VPN3 = VPN4|
            \item \verb|TLBT + TLBI = VPN|
        \end{choices}
        \proy{2016} 考虑如下程序,其被编译成可执行文件 \verb|i16|。
        \begin{minted}[frame=single, fontsize=\small, linenos]{c}
    int main(int argc, char **argv){
        int fd1, fd2;
        char *cs;
        struct stat stat;
        int sz, i;

        fd1 = Open(argv[1], O_RDWR, S_IRUSR | S_IWUSR);
        Fstat(fd1, &stat);
        sz = stat.st_size;
        fd2 = Open(argv[2], O_RDWR | O_TRUNC | O_CREAT, \
                        S_IRUSR | S_IWUSR);

        cs = (char *) Mmap(NULL, sz, PROT_READ | PROT_WRITE, \
                __________X__________, fd1, 0);
        if (Fork()) {
            Wait(&i);
            Write(fd2, cs, sz);
        } else {
            for (i = 0; i < sz; i++) {
                if (islower(cs[i]))
                    cs[i] = (char) toupper(cs[i]);
                else if (isupper(cs[i]))
           }
            cs[i] = (char) tolower(cs[i]);
            __________Y__________;
        }
        Munmap(cs, sz);
        Close(fd1);
        Close(fd2);
    }
        \end{minted}
        \qn 设文件 \verb|t1| 内容为 \verb|ICSexam|,根据 \verb|X/Y| 处的内容,判断执行 \verb|./i16 t1 t2| 之后的结果。
            \subqn \verb|X| 为 \verb|MAP_PRIVATE|,\verb|Y| 为空:
            \begin{choices}
                \item \verb|t1| 为 \verb|ICSexam|,\verb|t2| 为 \verb|ICSexam|
                \item \verb|t1| 为 \verb|ICSexam|,\verb|t2| 为 \verb|icsEXAM|
                \item \verb|t1| 为 \verb|icsEXAM|,\verb|t2| 为 \verb|icsEXAM|
                \item \verb|t1| 为 \verb|icsEXAM|,\verb|t2| 为 \verb|ICSexam|
            \end{choices}
            \subqn \verb|X| 为 \verb|MAP_SHARED|,\verb|Y| 为空:
            \begin{choices}
                \item \verb|t1| 为 \verb|ICSexam|,\verb|t2| 为 \verb|ICSexam|
                \item \verb|t1| 为 \verb|ICSexam|,\verb|t2| 为 \verb|icsEXAM|
                \item \verb|t1| 为 \verb|icsEXAM|,\verb|t2| 为 \verb|icsEXAM|
                \item \verb|t1| 为 \verb|icsEXAM|,\verb|t2| 为 \verb|ICSexam|
            \end{choices}
            \subqn \verb|X| 为 \verb|MAP_PRIVATE|,\verb|Y| 为 \verb|Write(fd2, cs, sz)|,\verb|t2| 的内容:
            \begin{choices}
                \item \verb|ICSexam|
                \item \verb|icsEXAM|
                \item \verb|icsEXAMICSexam|
                \item \verb|icsEXAMicsEXAM|
            \end{choices}
            \subqn \verb|X| 为 \verb|MAP_SHARED|,\verb|Y| 为 \verb|Write(fd2, cs, sz)|,\verb|t2| 的内容:
            \begin{choices}
                \item \verb|ICSexam|
                \item \verb|icsEXAM|
                \item \verb|icsEXAMICSexam|
                \item \verb|icsEXAMicsEXAM|
            \end{choices}
        \qn 设 \verb|X| 为 \verb|MAP_SHARED|。当子进程运行到 \verb|Y| 处时,内存映射的内容如下(只保留了 \verb|/proc/pid/maps| 中的三部分内容,分别为 \verb|address/perm/path|):
        \begin{minted}[frame=single, fontsize=\small]{text}
ADDRESS                    PERM   PATH 
00400000-00405000          r-xp   /home/larry/i16/i16
00604000-00605000          r--p   /home/larry/i16/i16
00605000-00606000          __A__  /home/larry/i16/i16
7f5df1e61000-7f5df2020000  r-xp   /lib/x86_64-linux-gnu/libc-2.23.so
7f5df2020000-7f5df2220000  ---p   /lib/x86_64-linux-gnu/libc-2.23.so
7f5df2220000-7f5df2224000  r--p   /lib/x86_64-linux-gnu/libc-2.23.so
7f5df2224000-7f5df2226000  rw-p   /lib/x86_64-linux-gnu/libc-2.23.so
7f5df2226000-7f5df222a000  rw-p
7f5df222a000-7f5df2250000  r-xp   /lib/x86_64-linux-gnu/____B____
7f5df243f000-7f5df2442000  rw-p
7f5df244c000-7f5df244d000  rw-s   __________C__________
7f5df244d000-7f5df244f000  rw-p
7f5df244f000-7f5df2450000  r--p   /lib/x86_64-linux-gnu/____B____
7f5df2450000-7f5df2451000  rw-p   /lib/x86_64-linux-gnu/____B____
7f5df2451000-7f5df2452000  rw-p
7ffd4e35f000-7ffd4e380000  rw-p   [_____D_____]
7ffd4e3b0000-7ffd4e3b2000  r--p   [vvar]
7ffd4e3b2000-7ffd4e3b4000  r-xp   [vdso]
ffffffffff600000-ffffffffff601000 r-xp [vsyscall]
        \end{minted}
        根据程序执行情况,填充上面的空白。
        \qn 设 \verb|X| 为 \verb|MAP_SHARED|。当子进程运行到 \verb|Y| 处时,回答有关问题。
            \subqn 关于页表的描述,正确的是:
            \begin{choices}
                \item 由于 \verb|i16| 本身的代码和数据只占用了低 32 位空间,所以这部分空间只需使用 2 级页表。
                \item 所有虚拟地址空间都是 48 位地址空间,都需要使用完整的 4 级页表。
                \item 当页表项无效时(\verb|P| 位为 0),MMU 不会使用到其它位的内容。
                \item 除读写权限外,需要在页表项中为 COW 机制提供另外的专门支持
            \end{choices}
            \subqn 如果子进程在 \verb|Y| 处访问 \verb|7f5df2224| 这一页,则可能发生以下异常:
            \begin{choices}
                \item Page Fault 或 General Protection Fault
                \item General Protection Fault 或 Segmentation Fault
                \item Page Fault 或 Segmentation Fault
                \item Page Fault 或 General Protection Fault 或 Segmentation Fault
            \end{choices}
            \subqn 假设 TLB 有 64 个表项,4 路组相联,访问 \verb|7f5df2224| 这一页时所对应的 \verb|TLBT| 是:
            \begin{choices}
                \item \verb|0111 1111 0101 1101 1111 0010 0010 0010|
                \item \verb|0111 1111 0101 1101 1111 0010 0010 0010 01|
                \item \verb|1111 0101 1101 1111 0010 0010 0010 0100|
                \item \verb|11 1111 0101 1101 1111 0010 0010 0010 0100|
            \end{choices}
            \subqn 发生 TLB 命中时意味着:
            \begin{choices}
                \item 相应的页表项在 L1 Cache 里
                \item 要访问的数据内容在 L1 Cache 里
                \item 如果要访问的数据内容已在 L1 Cache 中,则该 Cache Line 的权限位中应具有相应的读写执行权限
                \item 以上都不对
            \end{choices}
        \proy{2015} 虚拟内存管理方式可行性的基础是
        \begin{choices}
            \item 程序执行的离散性
            \item 程序执行的顺序性
            \item 程序执行的局部性
            \item 程序执行的并发性
        \end{choices}
        \proy{2015} Intel 的 IA32 体系结构采用二级页表,称第一级页表为页目录,第二级页表为页表。页面的大小为 \SI{4}{KB},页表项长 4 字节。以下给出了页目录与若干页表中的部分内容。例如,页目录中的 第 1 个项索引到的是页表 3,页表 1 中的第 3 个项索引到的是物理地址中的 第 5 个页。则十六进制逻辑地址 \verb|8052CB| 经过地址转换后形成的物理地址应为十进制的 \rule{2.5cm}{0.25mm}。
        \begin{table}[H]
            \tt
            \centering
            \begin{tabular}{|cc|cc|cc|cc|}
                \hline
                \multicolumn{2}{|c|}{页目录} & \multicolumn{2}{c|}{页表 1} & \multicolumn{2}{c|}{页表 2} & \multicolumn{2}{c|}{页表 3} \\ \hline
                \multicolumn{1}{|c|}{VPN} & 页表号 & \multicolumn{1}{c|}{VPN} & 页号 & \multicolumn{1}{c|}{VPN} & 页号 & \multicolumn{1}{c|}{VPN} & 页号 \\ \hline
                \multicolumn{1}{|c|}{1} & 3 & \multicolumn{1}{c|}{3} & 5 & \multicolumn{1}{c|}{2} & 1 & \multicolumn{1}{c|}{2} & 9 \\ \hline
                \multicolumn{1}{|c|}{2} & 1 & \multicolumn{1}{c|}{4} & 2 & \multicolumn{1}{c|}{4} & 4 & \multicolumn{1}{c|}{3} & 8 \\ \hline
                \multicolumn{1}{|c|}{3} & 2 & \multicolumn{1}{c|}{5} & 7 & \multicolumn{1}{c|}{8} & 6 & \multicolumn{1}{c|}{5} & 3 \\ \hline
            \end{tabular}
        \end{table}
        \proy{2015} 已知某系统页面大小为 \SI{8}{KB},页表项长 4 字节,采用多层分页策略映射 64 位虚拟地址空间。若限定最高层页表占 1 页,则它可以采用\rule{2.5cm}{0.25mm} 层的分页策略。
        \proy{2015} 为了提升虚拟内存地址的转换效率,降低遍历两级页表结构所带来的地址转换开销,Intel 处理器针对 32 位模式引入了大页 TLB,即一个 TLB 项可以涵盖整个 \SI{4}{MB} 对齐的地址空间。只要设置页目录项的大页标志位,即可让 MMU 识别这是一个大页 PDE,并加载到大页 TLB 项中。大页 PDE 中记录的物理内存页面号必须是 \SI{4}{MB} 对齐的,而且整个连续的 \SI{4}{MB} 内存均可统一通过该大页 PDE 进行地址转换。
        
        在 32 位的 Linux 系统中,为了方便访问物理内存,内核将地址 0\textasciitilde\SI{768}{MB} 间的物理内存映射到虚拟内存地址 \SI{3}{G}\textasciitilde\SI{3}{GB} + \SI{768}{MB} 间,并通过大页 PDE 进行进行该区间的地址转换。任何 0\textasciitilde\SI{768}{MB} 的物理内存地址可以直接通过加 \SI{3}{G}(\verb|0xC0000000|)的方式得到其虚拟内存地址。在内核中,除了该区间的内存外,其他地址的内存通常都通过普通的两级页表结构来进行地址转换。
        
        假设在我们使用的处理器中有 2 个大页 TLB 项,其当前状态如下:
        \begin{table}[H]
            \tt
            \centering
            \begin{tabular}{|c|c|c|c|}
                \hline
                索引号 & TLB 标记 & 页面号 & 有效位 \\ \hline
                0 & 0xC48 & 0x04800 & 1 \\ \hline
                1 & 0xC9C & 0x09C00 & 1 \\ \hline
            \end{tabular}
        \end{table}
        有 4 个普通 TLB 项,当前的状态如下:
        \begin{table}[H]
            \tt
            \centering
            \begin{tabular}{|c|c|c|c|}
                \hline
                索引号 & TLB标记 & 页面号 & 有效位 \\ \hline
                0 & 0xF8034 & 0x04812 & 1 \\ \hline
                1 & 0xF8033 & 0x09812 & 1 \\ \hline
                2 & 0xF4427 & 0x12137 & 1 \\ \hline
                3 & 0xF44AE & 0x17343 & 1 \\ \hline
            \end{tabular}
        \end{table}
        当前页活跃的目录页中的部分 PDE 的内容如下:
        \begin{table}[H]
            \tt
            \centering
            \begin{tabular}{|c|c|c|c|}
                \hline
                PDE 索引 & 页面号 & 大页位 & 存在位 \\ \hline
                786 & 0x04800 & 1 & 1 \\ \hline
                807 & 0x09C00 & 1 & 1 \\ \hline
                977 & 0x09C33 & 0 & 1 \\ \hline
                992 & 0x09078 & 0 & 1 \\ \hline
            \end{tabular}
        \end{table}
        注意,普通页面大小为 \SI{4}{KB},并且 \SI{4}{KB} 对齐。每个页面的页面号为其页面起始物理地址除以 4096 得到。大页由连续 1024 个 \SI{4}{KB} 小页组成,且 \SI{4}{MB} 对齐。
        \qn 分析下面的指令序列,
        \begin{minted}[frame=single, fontsize=\small]{text}
    movl   $0xC48012024, %ebx
    movl   $128, (%ebx)
    movl   $0xF8034000, %ecx
    movl   $36(%ecx), %eax
        \end{minted}
        执行完上述指令后,\verb|%eax| 寄存器中的内容是 \rule{2.5cm}{0.25mm};在执行上述指令过程中,共发生了 \rule{2.5cm}{0.25mm} 次 TLB miss;同时会发生 \rule{2.5cm}{0.25mm} 次 page fault。不确定的空填 X。
        \qn 请判断下列页面号对应的页面中,哪些一定是页表页,哪些不是,哪些不确
        定:\verb|0x04800|、\verb|0x09C33|、\verb|0x09812|。
        \qn 下列虚拟地址中哪一个对应能将虚拟地址 \verb|0xF4427048| 映射到物理地址 \verb|0x14321048| 的页表项?
        \begin{choices}
            \item \verb|0x09C33027|
            \item \verb|0xC9C3309C|
            \item \verb|0xC9C33027|
            \item \verb|0x09C3309C|
        \end{choices}
        通过上述虚拟地址,利用 \verb|movl| 指令修改对应的页表项,完成上述映射,在此过程中,是否会产生 TLB miss?

        进一步,修改页表项后,是否可以立即直接使用下面的指令序列将物理内存地址 \verb|0x14321048| 开始的一个 32 位整数清零?说明理由。
        \begin{minted}[frame=single, fontsize=\small]{text}
    movl   $0xF4427048, %ebx
    movl   $0, (%ebx)
        \end{minted}
        \proy{2014} IA32 体系采用小端法和二级页表。其中两级页表大小相同,页大小均为 \SI{4}{KB},结构也相同。TLB 采用直接映射。TLB 和页表每一项的前 20 位为物理地址,后 7 位含义如下图所示。为简便起见,假设 TLB 和页表每一项的后 8\textasciitilde 12 位都是 0 且不会被改变。例如后 7 位 值为 \verb|0x27| 则表示可读写。
        \begin{table}[H]
            \tt
            \centering
            \begin{tabular}{cccccccc}
                {\footnotesize 7} & {\footnotesize 6} & {\footnotesize 5} & {\footnotesize 4} & {\footnotesize 3} & {\footnotesize 2} & {\footnotesize 1} & {\footnotesize 0} \\ \hline
                \multicolumn{1}{|c|}{\cellcolor{gray!50}} & \multicolumn{1}{c|}{D} & \multicolumn{1}{c|}{A} & \multicolumn{1}{c|}{CD} & \multicolumn{1}{c|}{WT} & \multicolumn{1}{c|}{U} & \multicolumn{1}{c|}{W} & \multicolumn{1}{c|}{P} \\ \hline
                & {\footnotesize dirty} & {\footnotesize accessed} & {\footnotesize cache-disabled} & {\footnotesize write-through} & {\footnotesize user} & {\footnotesize writable} & {\footnotesize present}
            \end{tabular}
        \end{table}
        已知一级页表的地址为 \verb|0x0c23b000|,物理内存中的部分内容如下图所示。
        \begin{table}[H]
            \tt
            \centering
            \begin{tabular}{|c|c|c|c|c|c|c|c|}
                \hline
                地址 & 内容 & 地址 & 内容 & 地址 & 内容 & 地址 & 内容 \\ \hline
                00023000 & E0 & 00023001 & BE & 00023002 & EF & 00023003 & BE \\ \hline
                00023120 & 83 & 00023121 & C8 & 00023122 & FD & 00023123 & 12 \\ \hline
                00023200 & 23 & 00023201 & FD & 00023202 & BC & 00023203 & DE \\ \hline
                00023320 & 33 & 00023321 & 29 & 00023322 & E5 & 00023323 & D2 \\ \hline
                00023FF8 & 29 & 00023FF9 & FF & 00023FFA & DE & 00023FFB & BC \\ \hline
                00055004 & 03 & 00055005 & D0 & 00055006 & 74 & 00055007 & 89 \\ \hline
                0005545C & 97 & 0005545D & C2 & 0005545E & 7B & 0005545F & 45 \\ \hline
                00055460 & 97 & 00055461 & D2 & 00055462 & 7B & 00055463 & 45 \\ \hline
                00055464 & 97 & 00055465 & E2 & 00055466 & 7B & 00055467 & 45 \\ \hline
                0C23B020 & 55 & 0C23B021 & EB & 0C23B022 & AE & 0C23B023 & 24 \\ \hline
                0C23B040 & 55 & 0C23B041 & AB & 0C23B042 & 2A & 0C23B043 & 01 \\ \hline
                0C23B080 & 05 & 0C23B081 & 5D & 0C23B082 & 05 & 0C23B083 & 00 \\ \hline
                0C23B09D & 05 & 0C23B09E & D3 & 0C23B09F & F2 & 0C23B0A0 & 0F \\ \hline
                0C23B274 & 05 & 0C23B275 & 3D & 0C23B276 & 02 & 0C23B277 & 00 \\ \hline
                0C23B9FC & 25 & 0C23B9FD & D2 & 0C23B9FE & 14 & 0C23B9FF & 23 \\ \hline
                2314D200 & 23 & 2314D201 & 12 & 2314D202 & DC & 2314D203 & 0F \\ \hline
                2314D220 & A9 & 2314D221 & 45 & 2314D222 & 13 & 2314D223 & D2 \\ \hline
                2314D4A0 & BD & 2314D4A1 & BC & 2314D4A2 & 88 & 2314D4A3 & D3 \\ \hline
            \end{tabular}
        \end{table}
        TLB 的内容如下所示。
        \begin{table}[H]
            \tt
            \centering
            \begin{tabular}{|c|c|c|c|}
                \hline
                索引 & TLB 标记 & 内容 & 有效位 \\ \hline
                0 & 0x08001 & 2314d220 & 1 \\ \hline
                1 & 0x01000 & 24aee520 & 0 \\ \hline
                2 & 0x005AE & 00055004 & 0 \\ \hline
                3 & 0x016BA & 0c23b09d & 1 \\ \hline
                4 & 0x0AA00 & 0005545c & 1 \\ \hline
                5 & 0x0000A & 29dee500 & 0 \\ \hline
                6 & 0x5AE82 & 00023320 & 1 \\ \hline
                7 & 0x28DFC & 00023000 & 1 \\ \hline
            \end{tabular}
        \end{table}
        \qn 某用户态进程试图写入虚拟地址 \verb|0x080016ba|,该访问的最后结果是:
        \begin{choices}
            \item 该进程成功写入,未触发异常
            \item 该进程触发了一个缺页异常
            \item 该进程触发了一个非法访问异常
        \end{choices}
        \qn 下面描述了上一问的具体访问过程,请填空。如果某个空在访问过程中已不可用,请填入 X。TLB 的索引为 \rule{2.5cm}{0.25mm},访问 \rule{2.5cm}{0.25mm}(填“命中”或“不命中”)。一级页表表项地址为 \rule{2.5cm}{0.25mm};二级页表表项地址为 \rule{2.5cm}{0.25mm};最后得到的物理地址为 \rule{2.5cm}{0.25mm}。
        \proy{2014} 对于虚拟存储系统,一次访存过程中,下列命中组合不可能发生的是
        \begin{choices}
            \item TLB 未命中,Cache 未命中,Page 未命中
            \item TLB 未命中,Cache 命中,Page 命中
            \item TLB 命中,Cache 未命中,Page 命中
            \item TLB 命中,Cache 命中,Page 未命中
        \end{choices}
        \proy{2013} 假设有一台 64 位的计算机的物理页块大小是 \SI{8}{KB},采用三级页表进行虚拟地址寻址,则一般而言它的虚拟地址的 \verb|VPN| 有 \rule{2.5cm}{0.25mm} 位。
        \proy{2013} 进程 \verb|P1| 通过 \verb|fork| 函数产生一个子进程 \verb|P2|。假设执行 \verb|fork| 函数之前,进程 \verb|P1| 占用了 53 个用户态的物理页,则 \verb|fork| 函数之后,进程 \verb|P1| 和进程 \verb|P2| 共占用 \rule{2.5cm}{0.25mm} 个用户态的物理页。设执行 \verb|fork| 函数之前进程 \verb|P1| 中有一个可读写的物理页,则执行 \verb|fork| 函数之后,进程 \verb|P1| 对该物理页的页表项权限为 \rule{2.5cm}{0.25mm}。
        \proy{2013} Intel 的 IA32 体系结构采用二级页表,称第一级页表为页目录,第二级页表为页表。其虚拟地址到物理地址的翻译方式如:先根据 \verb|CR3| 寄存器找到页目录地址,然后依据偏移找到一个页目录项,页目录项的高 20 位为二级页表地址;在二级页表中根据偏移找到页表项,页表项中的高 20 位即为物理地址的高 20 位,将这 20 位与虚拟地址的低 12 位拼在一起形成完整的物理地址。

        页目录和页表均有 1024 项,每一项为 4 字节,31\textasciitilde 12 位为物理页号,11\textasciitilde 7 为保留和系统使用的位,最后 7 位的含义如下:
        \begin{table}[H]
            \tt
            \centering
            \begin{tabular}{cccccccc}
                {\footnotesize 7} & {\footnotesize 6} & {\footnotesize 5} & {\footnotesize 4} & {\footnotesize 3} & {\footnotesize 2} & {\footnotesize 1} & {\footnotesize 0} \\ \hline
                \multicolumn{1}{|c|}{\cellcolor{gray!50}} & \multicolumn{1}{c|}{D} & \multicolumn{1}{c|}{A} & \multicolumn{1}{c|}{CD} & \multicolumn{1}{c|}{WT} & \multicolumn{1}{c|}{U} & \multicolumn{1}{c|}{W} & \multicolumn{1}{c|}{P} \\ \hline
                & {\footnotesize dirty} & {\footnotesize accessed} & {\footnotesize cache-disabled} & {\footnotesize write-through} & {\footnotesize user} & {\footnotesize writable} & {\footnotesize present}
            \end{tabular}
        \end{table}

        页目录和页表由操作系统维护,通常只能在内核态下访问,为了给用户提供一个访问页表项和页目录项内容的接口,假设操作系统中已经执行过如下代码段:
        \begin{minted}[frame=single, fontsize=\small, linenos]{c}
    #define UVPT 0xef400000
    #define PDX(la) ((((unsigned int) (la)) >> 22) & 0x3FF)
    ...
    kern_pgdir[PDX(UVPT)] = PADDR(kern_pgdir) | PTE_U | PTE_P;
        \end{minted}
        其中 \verb|kern_pgdir| 是操作系统维护的页目录数组,共 1024 项,每一项的类型为 \verb|unsigned int|。\verb|PADDR(kern_pgdir)| 用于获得 \verb|kern_pgdir| 的物理地址,页目录在物理内存中正好占一页,所以 \verb|kern_pgdir| 的物理地址是 \SI{4}{KB} 对齐 的。\verb|PTE_U| 和 \verb|PTE_P| 代表了这个页目录项的权限,即用户态可访问(只读)。可以看到,这条语句将页目录的第 \verb|PDX(UVPT)| 项指向了页目录自身。
        
        利用这一点,对于给定的虚拟地址 \verb|va|,可以获得 \verb|va| 对应的页目录项和页表项内容,分别对应于函数 \verb|get_pde| 和 \verb|get_pte|,请完成这两个函数:
        \begin{minted}[frame=single, fontsize=\small, linenos]{c}
    #define UVPT 0xef400000
    // get_pde(va) 获取虚拟地址 va 对应的一级页表(页目录)中的页目录项内容
    unsigned int get_pde(unsigned int va) {
        unsigned int pdx = (va >> __________) & __________;
        unsigned int addr = UVPT + (__________) + pdx * 4;
        return *((unsigned int *) addr);
    }
    // get_pte(va) 获取虚拟地址 va 对应的二级页表中的页表项内容
    unsigned int get_pte(unsigned int va) {
        unsigned int PGNUM = va >> __________;
        unsigned int addr = __________ + PGNUM;
        return *((unsigned int *) addr);
    }
        \end{minted}
    \end{problems}