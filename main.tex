\documentclass[a4paper, 11pt]{book}
\usepackage{anyfontsize}
\usepackage[CJKchecksingle, AutoFakeBold]{xeCJK}
\usepackage{CJKnumb}

\usepackage{setspace}
\onehalfspacing
\renewcommand{\arraystretch}{1}

\setlength{\parindent}{2em}

\usepackage[intlimits, leqno]{amsmath}
\usepackage{amsthm}
\usepackage{amssymb}

\RequirePackage{newtxtext}
\RequirePackage[cmintegrals, cmbraces]{newtxmath}
\renewcommand{\mathbb}{\varmathbb}
\RequirePackage{esint}

\DeclareSymbolFont{CMlargesymbols}{OMX}{cmex}{m}{n}
\let\sumop\relax\let\prodop\relax
\DeclareMathSymbol{\sumop}{\mathop}{CMlargesymbols}{"50}
\DeclareMathSymbol{\prodop}{\mathop}{CMlargesymbols}{"51}

\RequirePackage{fontspec}

\setsansfont{TeX Gyre Heros}
\RequirePackage{courier}

\setCJKmainfont[
	ItalicFont=FZKai-Z03S
]{FZShuSong-Z01S}

\setCJKsansfont[
	BoldFont=* Medium,
	ItalicFont=FZKai-Z03S
]{Noto Sans CJK SC}

\setCJKmonofont[
	ItalicFont=FZKai-Z03S
]{FZShuSong-Z01S}

\setCJKfamilyfont{kai}{FZKai-Z03S}

\setCJKfamilyfont{song}[
	ItalicFont=FZKai-Z03S
]{FZShuSong-Z01S}

\setCJKfamilyfont{fangsong}[
	ItalicFont=FZKai-Z03S
]{FZFangSong-Z02S}

\setCJKfamilyfont{hei}{Noto Sans CJK SC Medium}
\setCJKfamilyfont{hei2}[
	BoldFont=* Bold
]{Noto Sans CJK SC}

\setCJKfamilyfont{black}[
	BoldFont=* Black
]{Noto Sans CJK SC}

\setCJKfamilyfont{sectionfont}[
	BoldFont=* Black
]{Noto Sans CJK SC}

\setCJKfamilyfont{pffont}[
	BoldFont=* Medium
]{Noto Sans CJK SC}

\setCJKfamilyfont{emfont}[
	BoldFont=* Medium
]{Noto Sans CJK SC}

\defaultfontfeatures{Ligatures=TeX} 
\XeTeXlinebreaklocale "zh"
\XeTeXlinebreakskip=0pt plus 1pt minus 0.1pt

\newcommand\kaishu{\CJKfamily{kai}}
\newcommand\songti{\CJKfamily{song}}
\newcommand\heiti{\CJKfamily{hei}}
\newcommand\thmheiti{\CJKfamily{hei2}}
\newcommand\fangsong{\CJKfamily{fangsong}}
\renewcommand{\em}{\bfseries\CJKfamily{emfont}}

\usepackage[unicode, colorlinks]{hyperref}
\hypersetup{
	linkcolor=blue,
	citecolor=red,
	urlcolor=teal,
	pdfauthor={王畅},
	pdftitle={ICS 问题求解(2021 秋)}
}

\usepackage[iso, english]{isodate}
\usepackage{geometry}
\geometry{
	paper=a4paper,
	top=3cm,
	inner=2.54cm,
	outer=2.54cm,
	bottom=3cm,
	headheight=6ex,
	headsep=6ex,
}

\usepackage[dvipsnames]{xcolor}

\usepackage{paralist}
\usepackage{commands}
\usepackage[figurewithin=none]{caption}
\usepackage{graphicx}
\usepackage{float}
\usepackage{tikz}
\usetikzlibrary{shapes.symbols, backgrounds, matrix, calc, arrows, math, arrows.meta}
\usepackage{minted}

\numberwithin{equation}{chapter}
\renewcommand{\theequation}{\thechapter.\arabic{equation}}

\usepackage{datetime}

\usepackage{xstring}
\usepackage{fancyhdr}
\newcommand\sectionname{\arabic{chapter}.\arabic{section}}
\pagestyle{fancy}
\renewcommand{\chaptermark}[1]{\markboth{
	第\CJKnumber{\thechapter}周\quad #1
	}{}}
\renewcommand{\sectionmark}[1]{\markright{\arabic{chapter} \quad #1}}
\fancyhf{}
\fancyhead[EC]{\CJKfamily{hei2}\footnotesize{\leftmark}\vspace{1mm}}
\fancyhead[OC]{\CJKfamily{hei2}\footnotesize{\rightmark}\vspace{1mm}}
\fancyhead[LE, RO]{{\footnotesize \thepage}\vspace{1mm}}
\fancyhead[RE, LO]{}
\renewcommand{\headrulewidth}{0.5pt}
\renewcommand{\footrulewidth}{0pt}
\addtolength{\headheight}{0.5pt}

\fancypagestyle{plain}{%
	\fancyhead{}
	\renewcommand{\headrulewidth}{0pt}
}

\usepackage[many]{tcolorbox}

\newtcolorbox{summary}{
	breakable,
	enhanced,
	width=\textwidth,
	colback=white, colbacktitle=white,
	colframe=gray!50, boxrule=0.2mm,
	coltitle=black,
	fonttitle=\bfseries\sffamily,
	attach boxed title to top left={yshift=-\tcboxedtitleheight/2, xshift=\tcboxedtitlewidth/4},
	boxed title style={boxrule=0pt, colframe=white},
	after skip=1cm,
	top=3mm,
	bottom=3mm,
	title={要点}
}

\usepackage[calcwidth, explicit, nobottomtitles, newparttoc, indentafter]{titlesec}
\usepackage{titletoc}
\usepackage[titles]{tocloft}
\cftsetpnumwidth{1.75em}
\providecommand{\dmchapterttl}[1]{\IfSubStr{ABCDEFGHIJKLMNOPQRSTUVWXYZ}{#1}{附录 #1	}{第\CJKnumber{#1}周}}

\newlength{\BoxTtlwidth}

\newcommand{\MakeChapBox}[2]{%
	\settowidth{\BoxTtlwidth}{\dmchapterttl{#1}}
	\begin{tcolorbox}[
		enhanced jigsaw,
		skin=bicolor,
		frame engine=path,
		sharp corners=all,
		width=0.9\textwidth,
		top=4mm, bottom=4mm,
		sidebyside,
		frame hidden,
		boxrule=0mm,
		lefthand width=\BoxTtlwidth,
		colupper=white,
		colback=gray!80,
		colbacklower=gray!10,
		sidebyside align=center,
		halign=center]
		\dmchapterttl{#1}
		\tcblower
		#2
	\end{tcolorbox}%
}

\newcommand{\MakeChapBoxSingle}[1]{%
	\begin{tcolorbox}[
		enhanced,
		width=0.7\textwidth,
		sharp corners=all,
		top=4mm, bottom=4mm,
		frame hidden,
		boxrule=0mm,
		colback=gray!10,
		halign=center]
		#1
	\end{tcolorbox}
}

\newtcbox{\MakeSectBox}{
	enhanced,
	arc=0pt, outer arc=0pt,
	before skip=0pt, after skip=0.4em, left skip=0pt, right skip=0pt,
	top=0pt, left=0pt, right=0pt, bottom=1.5mm,
	sharp corners=all,
	valign=bottom,
	colback=white,
	colframe=white,
	boxsep=0pt, leftrule=0pt, rightrule=0pt, toprule=0pt, bottomrule=0pt,
	overlay={ \draw[line width=1pt] (interior.south west) -- (interior.south east); }
}

\titleformat{name=\chapter}
	{\filright\sffamily\CJKfamily{black}\bfseries\Huge}
	{}
	{0mm}
	{\MakeChapBox{\thechapter}{#1}}
	[]
\titlespacing*{name=\chapter}
	{1pc}{*4}{1em}

\titleformat{name=\chapter, numberless}
	{\filcenter\sffamily\CJKfamily{black}\bfseries\Huge}
	{}
	{0mm}
	{\MakeChapBoxSingle{#1}}
	[{\if@mainmatter
		\addcontentsline{toc}{chapter}{#1}
		\markboth{#1}{}
	\fi}]
\titlespacing*{name=\chapter, numberless}
	{1pc}{*4}{1em}

\titleformat{name=\section}
	{\filleft\normalfont\sffamily\bfseries\CJKfamily{sectionfont}}
	{}
	{0mm}
	{ \settowidth{\BoxTtlwidth}{\Huge \thesection \hspace{0.7em} \Large #1}
		\ifdim \BoxTtlwidth < \textwidth
			\MakeSectBox{\Huge \thesection \hspace{0.7em} \Large #1}\vskip-18pt%
		\else
			\Huge \underline{\thesection} \hspace{0.7em} \Large #1%
		\fi%
	}
	[]

\titleformat{name=\section, numberless}
	{\filleft\normalfont\sffamily\bfseries\CJKfamily{sectionfont}}
	{}
	{0mm}
	{ \settowidth{\BoxTtlwidth}{\Large #1}
		\ifdim \BoxTtlwidth < \textwidth
			\MakeSectBox{\Large #1}\vskip-18pt%
		\else
			\Large #1
		\fi
	}
	[]
	
\titlespacing*{name=\section}
	{1pc}{*1.3}{*1}

\usepackage{zhnumber}
\usepackage[answerdelayed, lastexercise]{exercise}
\renewcommand{\ExerciseListName}{}
\renewcommand{\AnswerListName}{}
\renewcommand{\ExerciseHeaderNB}{\theExercise}
\renewcommand{\QuestionNB}{(\arabic{Question})}
\renewcommand{\ExerciseHeaderOrigin}{\ ({\kaishu\ExerciseOrigin})}
\renewcommand{\ExerciseListHeader}{\ExerciseHeaderDifficulty\textbf{\ExerciseHeaderNB}\ExerciseHeaderOrigin\textbf{.}\enspace}
\renewcommand{\AnswerListHeader}{\textbf{\ExerciseHeaderNB.}\enspace}
\renewcounter{Exercise}[chapter]
\renewcommand{\AtBeginAnswer}{\vspace{-1\baselineskip}}
\newcommand{\pro}{\Exercise}
\newcommand{\qn}{\Question}
\newcommand{\subqn}{\subQuestion}
\newcommand{\sol}{\Answer}

\makeatletter
\renewcommand{\@@@ExeCmd}{%
	\ifnum\@QuestionLevel=0
		\advance \@QuestionLevel by 1
		\begin{list}{\@getExerciseInfo\ExerciseListHeader}%
		{\partopsep \Exepartopsep \labelsep \Exelabelsep \itemsep \Exesep \listparindent 2em%
		\parsep \Exeparsep \topsep \Exetopsep \labelwidth \Exelabelwidth%
		\leftmargin \Exeleftmargin \rightmargin \Exerightmargin}
	\else
		\termineliste{1}\@EndExeBox
	\fi
	\@selectExercise
	\global\@Answerfalse\@BeginExeBox\refstepExecounter%
	\addcontentsline{\ext@exercise}{\toc@exercise}{\ExerciseName\
	\theExercise\ \expandafter{\itshape \ExerciseTitle}\hspace{.66em}}
	\item\ignorespaces\AtBeginExercise
}
\makeatother

\newcounter{soleq}
\counterwithin{soleq}{chapter}
\newenvironment{soleq}{\refstepcounter{soleq}\equation}{\tag{S-\thesoleq}\endequation}

\makeatletter
\newenvironment{problems}{%
	\begin{ExerciseList}
		\global\setbox\temp@Answerbox%
		\vbox{\unvbox\all@Answerbox\section*{第\zhnum{chapter}周}\unvbox\@Exercisebox\vskip\z@}%
		\global\setbox\all@Answerbox\copy\temp@Answerbox
		}{
	\end{ExerciseList}
}

\newenvironment{choices}{\begin{compactenum}[\quad\ \,A.{\ \ \ }]}{\end{compactenum}}

\newenvironment{hint}{%
	\ifvmode
		\ignorespaces
	\else
		\quad
	\fi
	\begin{tikzpicture}[baseline=(H.base), every node/.style={signal, draw, very thin, signal to=east, signal from=nowhere, signal pointer angle=120, inner sep=2pt}]
		\node[anchor=mid west] (H) at (0,0) {\heiti\footnotesize 提示};
	\end{tikzpicture}
}{}

\g@addto@macro\frontmatter{\setcounter{tocdepth}{1}
	\setcounter{page}{1}
	\thispagestyle{empty}
	\addtocontents{toc}{\protect\thispagestyle{empty}}
	
	\renewcommand{\contentsname}{目录}
	\tableofcontents
}

\renewcommand{\figurename}{图}
\renewcommand{\tablename}{表}

\renewcommand{\listfigurename}{图片索引}
\renewcommand{\listtablename}{表格索引}

\pretocmd{\listoffigures}{%
	\cleardoublepage
	\phantomsection
	\addcontentsline{toc}{chapter}{\listfigurename}
}{}{}
\pretocmd{\listoftables}{%
	\cleardoublepage
	\phantomsection
	\addcontentsline{toc}{chapter}{\listtablename}
}{}{}

\begin{document}
	\begin{titlepage}
		\vspace*{\stretch{1}}
		\noindent\begin{center}\rule{\linewidth}{2pt}\end{center}
		\vspace{0.5em}
		\begin{center}
			{\fontsize{32}{32} \selectfont \bfseries \sffamily \CJKfamily{sectionfont} ICS 问题求解} \\ \vspace{1.5em}
			\LARGE PKU 04832363:计算机系统导论讨论班
		\end{center}
		\noindent\begin{center}\rule{\linewidth}{2pt}\end{center}
		\vspace{1em}
		\begin{center}
			{\Large 2021 年秋} \\ \vspace{1em}
			{\Large 北京大学} \\ \vspace{1.5em}
		\end{center}
		\vfill
		\begin{center} \href{https://www.pku.edu.cn/}{\includegraphics[height=64pt]{logo.eps}} \end{center}
	\end{titlepage}
	
	\newpage
	\thispagestyle{empty}
	\begin{center}
		\Large 编译于 {\today} \currenttime
	\end{center}
	\vfill
	\begin{flushleft}
		\large
    	\textbf{汇编:}王畅 \\
    	\textbf{勘误:}请写信到 \href{mailto:wchang@pku.edu.cn}{\texttt{wchang@pku.edu.cn}} \\
		编者仅做了汇编和微小的改动,主要内容均来源于往年计算机系统导论课程,请悉知。
    \end{flushleft}

	\frontmatter

	\mainmatter
	\chapter{位级表示}
	\begin{summary}
		\begin{compactitem}
			\item 知道整数、浮点数在系统的存储方式。
			\item 熟练掌握整数、浮点数的位级表示规则,快速完成其同十进制数的转换。
			\item 理解整数、浮点数规范中的各类特殊数的计算方法及其性质。
			\item 运用浮点数舍入的规则进行运算,知道类型转换的基本规则,能针对整数、浮点数的一些常见“反常情况”进行判断。
			\item *了解位级操作中常见的技巧和套路,会用位运算处理非常规的编程要求。
		\end{compactitem}
	\end{summary}

    \begin{problems}
		\pro 在 x86-64 机器上,定义 \texttt{unsigned int A = 0x123456}。请画出 \texttt{A} 在内存中的存储方式:
		\begin{table}[H]
			\centering
			\begin{tabular}{|c|c|c|c|c|c|c|c|}
				\hline
				... & 低地址 & \multicolumn{4}{c|}{\texttt{A}} & 高地址 & ... \\ \hline
				\multicolumn{2}{|c|}{...} & {\qquad \qquad} & {\qquad \qquad} & {\qquad \qquad} & {\qquad \qquad} & \multicolumn{2}{c|}{...} \\ \hline
			\end{tabular}
		\end{table}
		定义 \texttt{unsigned short B[2] = \{0x1234, 0x5678\}}。请画出 \texttt{B} 在内存中的存储方式:
		\begin{table}[H]
			\centering
			\begin{tabular}{|c|c|c|c|c|c|c|c|}
				\hline
				... & 低地址 & \multicolumn{4}{c|}{\texttt{B}} & 高地址 & ... \\ \hline
				\multicolumn{2}{|c|}{...} & {\qquad \qquad} & {\qquad \qquad} & {\qquad \qquad} & {\qquad \qquad} & \multicolumn{2}{c|}{...} \\ \hline
			\end{tabular}
		\end{table}
		\pro 下列代码的目的是将字符串 \texttt{A} 的内容复制到字符串 \texttt{B},覆盖 \texttt{B} 原有的内容,并输出“Hello World”;但实际运行输出是“Buggy Codes”。尝试找到代码中的错误。
		\begin{minted}[frame=single, fontsize=\small]{c}
    int main() {
        char A[12] = "Hello World";
        char B[12] = "Buggy Codes";
        int pos;
        for (pos = 0; pos - sizeof(B) < 0; pos++)
            B[pos] = A[pos];
        printf("%s\n", B);
    }
		\end{minted}
		\pro 在 x86-64 机器上,有下列 C 代码:
		\begin{minted}[frame=single, fontsize=\small]{c}
    int main() {
        unsigned int A = 0x11112222;
        unsigned int B = 0x33336666;
        void *x = (void *)&A;
        void *y = 2 + (void *)&B;
        unsigned short P = *(unsigned short *)x;
        unsigned short Q = *(unsigned short *)y;
        printf("0x%04x", P + Q);
        return 0;
    }
		\end{minted}
		运行该代码,结果是什么?
		\pro 在 x86-64 机器上,有下列 C 代码:
		\begin{minted}[frame=single, fontsize=\small]{c}
    int main() {
        char A[12] = "11224455";
        char B[12] = "11445577";
        void *x = (void *)&A;
        void *y = 2 + (void *)&B;
        unsigned short P = *(unsigned short *)x;
        unsigned short Q = *(unsigned short *)y;
        printf("0x%04x", Q - P);
        return 0;
    }
		\end{minted}
		运行该代码,结果是什么?
		\pro 在 x86-64 机器上,有如下的定义:
		\begin{minted}[frame=single, fontsize=\small]{c}
    int x = ...; // 表达式 A
    int y = ...; // 表达式 B
    unsigned int ux = x;
    unsigned int uy = y;
		\end{minted}
		判断下表中的表达式是否等价:
		\begin{table}[H]
			\centering
			\begin{tabular}{|c|c|c|}
				\hline
				序号 & 表达式 \verb|A| & 表达式 \verb|B| \\ \hline
				1 & \verb|x > y| & \verb|ux > uy| \\ \hline
				2 & \verb+(x > 0) || (x < ux)+ & \verb|1| \\ \hline
				3 & \verb|x ^ y ^ x ^ y ^ x| & \verb|x| \\ \hline
				4 & \verb|((x >> 1) << 1) <= x| & \verb|1| \\ \hline
				5 & \verb|((x / 2) * 2) <= x| & \verb|1| \\ \hline
				6 & \verb|x ^ y ^ (~x) - y| & \verb|y ^ x ^ (~y) - x| \\ \hline
				7 & \verb|(x == 1) && (ux - 2 < 2)| & \verb|(x == 1) && ((!!ux) - 2 < 2)| \\ \hline
			\end{tabular}
		\end{table}
		\begin{hint}
			减法的运算优先级比按位异或高。布尔运算的结果都是有符号数。
		\end{hint}
		\pro 现有一个二进制浮点的表示规则,其中 $E$ 为指数部分(3 比特),bias为 3;$M$ 为小数部分(5 比特),采用二进制补码表示形式,且取值 $0.5 \leq |M|<1$,$s$ 是浮点的符号位。该形式包含一个值为 1 的隐藏位。问 $+5_{10}$ 在该表示下的值是下列哪一个?
		\begin{choices}
			\item \texttt{010001100}
			\item \texttt{010100100}
			\item \texttt{011011010}
			\item \texttt{011110101}
		\end{choices}
        \pro 下面关于 IEEE 浮点数标准说法正确的是哪个?
		\begin{choices}
			\item 在位数一定的情况下,不论怎么分配 exponent bits 和 fraction bits,所能表示的数的个数是不变的。
			\item 若甲类浮点数有 10 位,乙类浮点数有 11 位,那么甲所能表示的最大数一定比乙小。
			\item 若甲类浮点数有 10 位,乙类浮点数有 11 位,那么甲所能表示的最小正数一定比乙小。
		    \item “\texttt{0111000}”可能是 7 位浮点数的 \texttt{NaN} 表示。
		\end{choices}
        \pro 对于 IEEE 浮点数,如果减少 1 位指数位,将其用于小数部分,下列叙述正确的是哪个?
		\begin{choices}
			\item 能表示更多数量的实数值,但实数值取值范围比原来小了。
			\item 能表示的实数数量没有变化,但数值的精度更高了。
			\item 能表示的最大实数变小,最小的实数变大,但数值的精度更高。
			\item 以上说法都不正确。
        \end{choices}
		\pro 假设某浮点数格式为 1 位符号、3 位阶码、4 位小数。下表给出了用该格式表达的浮点数 $(-1)^SM \cdot 2^E$ 与其二进制表示的关系。完成下表。
		\begin{table}[H]
			\centering
			\begin{tabular}{|c|c|c|c|c|}
				\hline
				描述 & 二进制表示 & $M$(写成分数) & $E$ & $f$ \\ \hline
				负零 &  & / & / & \texttt{-0.0} \\ \hline
				/ & \verb|01000101| &  &  &  \\ \hline
				最小的非规格化负数 &  &  &  &  \\ \hline
				最大的规格化正数 &  &  &  &  \\ \hline
				一 &  &  &  & \texttt{1.0} \\ \hline
				/ &  &  &  & \texttt{5.5} \\ \hline
				$+\infty$ &  & / & / & / \\ \hline
			\end{tabular}
		\end{table}
		\pro 假设浮点数格式 A 为 1 位符号、3 位阶码、4 位小数,浮点数格式 B 为 1 位符号、4 位阶码、3 位小数。回答下列问题。
			\qn 格式 A 中有多少个二进制表示对应于正无穷大?
			\qn 考虑能精确表示的实数的最大绝对值。A 比 B 大还是比 B 小,还是两者一样?
			\qn 考虑能精确表示的实数的最小非零绝对值。A 比 B 大还是比 B 小,还是两者一样?
			\qn 考虑能精确表示的实数的个数。A 比 B 多还是比 B 少,还是两者一样?
		\pro 判断下列说法的正确性。
			\qn 对于任意的单精度浮点数 \texttt{a} 和 \texttt{b},如果 \texttt{a > b},那么 \texttt{a + 1 > b}。
			\qn 对于任意的单精度浮点数 \texttt{a} 和 \texttt{b},如果 \texttt{a > b},那么 \texttt{a + b > b + b}。
			\qn 对于任意的单精度浮点数 \texttt{a} 和 \texttt{b},如果 \texttt{a} > \texttt{b},那么 \texttt{a + 1 > b + 1}。
			\qn 对于任意的双精度浮点数 \texttt{d},如果 \texttt{d < 0},那么 \texttt{d * d > 0}。
			\qn 对于任意的双精度浮点数 \texttt{d},如果 \texttt{d < 0},那么 \texttt{d * 2 < 0}。
			\qn 对于任意的双精度浮点数 \texttt{d},\texttt{d == d}。
			\qn 将 \texttt{float} 转换成 \texttt{int} 时,既有可能造成舍入,又有可能造成溢出。
		\pro 遵循 IEEE 754 浮点数标准,考虑下列代码:
		\begin{minted}[frame=single, fontsize=\small]{c}
    for (int x = 0; ; x++) {
        float f = x;
        if (x != (int)f) {
            printf("%d", x);
            break;
        }
    }
		\end{minted}
		试问代码的运行结果是什么?或者死循环?
		\pro 遵循 IEEE 754 浮点数标准,考虑下列代码:
		\begin{minted}[frame=single, fontsize=\small]{c}
    int x = 33554466; // 2^25 + 34
    int y = x + 8;
    for ( ; x < y; x++) {
        float f = x;
        printf("%d ", x - (int)f);
    }
		\end{minted}
		写出程序的运行结果。
    \end{problems}
	\chapter{部分参考答案}
    \vspace{-1cm}
    \shipoutAnswer
\end{document}