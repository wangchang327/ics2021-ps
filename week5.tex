\chapter{链接}
    \begin{summary}
        \begin{compactitem}
            \item 知道源代码编译为可执行文件的全过程,知道静态链接和动态链接的概念,会正确书写含有多文件的代码的编译命令(顺序问题)。
            \item 熟悉 Linux 下典型目标文件格式 ELF 的各个部分,掌握符号表、局部符号、全局符号、强符号、弱符号等概念及其存储的区域。
            \item 掌握符号解析的过程,会判断链接是否成功以及失败的原因。
            \item 熟练掌握重定位的过程,会根据重定位条目来定位符号的相对引用和绝对引用。
            \item 了解目标程序被加载到内存的过程,知道动态链接的概念及其和静态链接的区别和优劣,了解位置无关代码和库打桩的基本技术。
        \end{compactitem}
    \end{summary}

    \begin{problems}
        \pro 下图为一个典型的编译过程。将正确的过程填上,并补充缺失的拓展名。可供选择的过程有:汇编器 \verb|as|、预处理器 \verb|cpp|、编译器 \verb|cc1|。
        \begin{table}[H]
            \centering
            \begin{tabular}{cc}
                \texttt{\qquad \qquad main.c \qquad \qquad} & \texttt{\qquad \qquad \ lib.c \qquad \qquad} \\ \hline
                \multicolumn{1}{|c|}{} & \multicolumn{1}{c|}{} \\ \hline
                \texttt{main.i} & \texttt{lib.i} \\ \hline
                \multicolumn{1}{|c|}{} & \multicolumn{1}{c|}{} \\ \hline
                \verb|main.___| &  \\ \hline
                \multicolumn{1}{|c|}{} & \multicolumn{1}{c|}{} \\ \hline
                \verb|main.___| & \verb|lib.___| \\ \cline{2-2} 
                \multicolumn{1}{c|}{} & \multicolumn{1}{c|}{创建静态库 \texttt{ar}} \\ \cline{2-2} 
                & \verb|lib.___| \\ \hline
                \multicolumn{2}{|c|}{链接器 \texttt{ld}} \\ \hline
                \multicolumn{2}{c}{目标程序 \texttt{prog}}
            \end{tabular}
        \end{table}
        \pro 判断下面关于静态链接的说法是否正确。
            \qn 链接时,链接器会拷贝静态库(\verb|.a|)中的所有模块(\verb|.o|)。
            \qn 链接时,链接器只会从每个模块(\verb|.o|)中拷贝出被用到的函数。
            \qn 链接时,如果所有的输入文件都是 \verb|.o| 或 \verb|.c| 文件,那么任意交换输入文件的顺序,都不会影响链接是否成功。
            \qn 链接时,通过合理地安排静态库和模块的顺序,每个静态库都可以在命令中出现至多一次。
        \pro 有下面两个程序。将他们先分别编译为 \verb|.o| 文件,再链接为可执行文件。
        \begin{minted}[frame=single, fontsize=\small]{c}
    /* main.c */
    #include <stdio.h>
    __________A__________
    int foo(int n) {
        static int ans = 0;
        ans = ans + x;
        returnn + ans;
    }
    int bar(int n);
    void op(void) { x = x + 1; }

    int main() {
        for (int i = 0; i < 3; i++) {
            int a1 = foo(0);
            int a2 = bar(0);
            op();
            printf("%d %d ", a1, a2);
        } 
        return 0;
    }
        \end{minted}
        \begin{minted}[frame=single, fontsize=\small]{c}
    /* count.c */
    __________B__________
    int bar(int n) {
        static int ans = 0;
        ans = ans + x;
        return n + ans;
    }
        \end{minted}
        \qn 当 \verb|A| 处为 \verb|int x = 1;|,\verb|B| 处为 \verb|int x;| 时,完成下表。如果某个变量不在符号表中,那么在名字那一栏划 X;如果它在符号表中的名字含有随机数字,那么请用不同的四位数字区分多个不同的符号。对于局部符号,不需要填最后一栏。
        \begin{table}[H]
            \centering
            \begin{tabular}{|c|c|c|c|c|}
                \hline
                文件名 & 变量名 & 在符号表中的名字 & 是局部符号吗? & 是强符号吗? \\ \hline
                \multirow{3}{*}{\texttt{main.c}} & \texttt{x} &  &  &  \\ \cline{2-5} 
                & \texttt{bar} &  &  &  \\ \cline{2-5} 
                & \texttt{ans} &  &  &  \\ \hline
                \multirow{3}{*}{\texttt{count.c}} & \texttt{x} &  &  &  \\ \cline{2-5} 
                & \texttt{bar} &  &  &  \\ \cline{2-5} 
                & \texttt{ans} &  &  &  \\ \hline
            \end{tabular}
        \end{table}
        程序能够链接成功吗?如果可以,程序的运行结果是什么?如果不可以,链接器报什么错?
        \qn 当 \verb|A| 处为 \verb|static int x = 1;|,\verb|B| 处为 \verb|static int x = 1;| 时,完成下表。
        \begin{table}[H]
            \centering
            \begin{tabular}{|c|c|c|c|c|}
                \hline
                文件名 & 变量名 & 在符号表中的名字 & 是局部符号吗? & 是强符号吗? \\ \hline
                \multirow{3}{*}{\texttt{main.c}} & \texttt{x} &  &  &  \\ \cline{2-5} 
                & \texttt{bar} &  &  &  \\ \cline{2-5} 
                & \texttt{ans} &  &  &  \\ \hline
                \multirow{3}{*}{\texttt{count.c}} & \texttt{x} &  &  &  \\ \cline{2-5} 
                & \texttt{bar} &  &  &  \\ \cline{2-5} 
                & \texttt{ans} &  &  &  \\ \hline
            \end{tabular}
        \end{table}
        程序能够链接成功吗?如果可以,程序的运行结果是什么?如果不可以,链接器报什么错?
        \qn 当 \verb|A| 处为 \verb|int x = 1;|,\verb|B| 处为 \verb|int x = 1;| 时。程序能够链接成功吗?如果可以,程序的运行结果是什么?如果不可以,链接器报什么错?
        \pro 有如下 C 代码:
        \begin{minted}[frame=single, fontsize=\small]{c}
    #define k 100
    long foo(long n);
    long bar(long n) {
        static long ans = 0;
        long acc = 0;
        for (int i = 0; i < n; i++) {
            ans += i;
            acc += ans * n;
        }
        return ans + acc;
    } 
    long t;
    static long y;
    extern long z;
    int main() {
        long x;
        myScanf("%ld%ld%ld", &x, &y, &z);
        myPrintf("%ld %ld\n", foo(x + y + t), bar(z + k));
        return 0;
    }
        \end{minted}
        采用命令 \verb|gcc test.c -c -Og -no-pie -fno-pie| 与 \verb|readelf -a test.o > t.txt| 后得到解析文件。

        \verb|t.txt| 中的部分节头部表信息如下:
        \begin{table}[H]
            \tt
            \centering
            \begin{tabular}{|ccccc|}
                \hline
                号 & 名称 & 类型 & 地址 & 偏移量 \\
                1 & .text & PROGBITS & 0000000000000000 & 00000040 \\
                3 & .data & PROGBITS & 0000000000000000 & 000000ff \\
                4 & .bss & NOBITS & 0000000000000000 & 00000100 \\
                5 & .rodata.str1.1 & PROGBITS & 0000000000000000 & 00000100 \\
                10 & .symtab & SYMTAB & 0000000000000000 & 00000190 \\ \hline
            \end{tabular}
        \end{table}

        \verb|t.txt| 中的部分符号表如下:
        \begin{table}[H]
            \tt
            \centering
            \begin{tabular}{cccccccc}
                \multicolumn{8}{l}{Symbol table '.symtab' contains ?? entries:} \\ \hline
                \multicolumn{1}{|c}{Num} & Value & Size & Type & Bind & Vis & Ndx & \multicolumn{1}{c|}{Name} \\
                \multicolumn{1}{|c}{5} & 0000000000000000 &  & OBJECT &  & DEFAULT &  & \multicolumn{1}{c|}{ans.1797} \\
                \multicolumn{1}{|c}{7} & 0000000000000008 &  & OBJECT &  & DEFAULT &  & \multicolumn{1}{c|}{y} \\
                \multicolumn{1}{|c}{11} & 0000000000000000 & 52 & FUNC &  & DEFAULT &  & \multicolumn{1}{c|}{bar} \\
                \multicolumn{1}{|c}{12} & 0000000000000034 & 139 & FUNC & GLOBAL & DEFAULT &  & \multicolumn{1}{c|}{main} \\
                \multicolumn{1}{|c}{13} & 0000000000000000 & 0 & NOTYPE &  & DEFAULT &  & \multicolumn{1}{c|}{z} \\
                \multicolumn{1}{|c}{15} & 0000000000000008 &  & OBJECT &  & DEFAULT &  & \multicolumn{1}{c|}{t} \\ \hline
            \end{tabular}
        \end{table}
        \qn 除了上述已经列出的符号外,判断下列名字是否在符号表中。
        \begin{table}[H]
            \tt
            \centering
            \begin{tabular}{|c|c|c|c|c|c|c|c|}
                \hline
                名称 & k & ans & acc & foo & y.???? & x & n \\ \hline
                是否出现 & {\qquad \qquad} & {\qquad \qquad} & {\qquad \qquad} & {\qquad \qquad} & {\qquad \qquad} & {\qquad \qquad} & {\qquad \qquad} \\ \hline
            \end{tabular}
        \end{table}
        \qn 补全上述符号表中漏掉的信息。其中 \verb|Bind| 可以是 \verb|LOCAL| 或者 \verb|GLOBAL|,\verb|Ndx| 可以是表示节头标号的数字,也可以是 \verb|UND (undefined)| 或 \verb|COM (common)|。
        \qn 字符串 \verb|"%ld %ld\n"| 位于哪个节中?
        \qn 假设在全局区域定义 \verb|long A[1000000]|,那么在 \verb|test.o| 中,\verb|.bss| 节占用的空间为多少字节?
        \qn 使用 \verb|objdump -dx test.o| 查看发现有如下的汇编代码:
        \begin{minted}[frame=single, fontsize=\small]{text}
    0000000000000000 <bar>:
        0: b9 00 00 00 00         mov    $0x0, %ecx
    ...
    0000000000000034 <main>:
        34: 53                    push   %rbx
    ...
        6b: e8 00 00 00 00        callq  70 <main+0x3c>
        6c: R_X86_64_PC32 bar-0x4
    ...
        90: bf 00 00 00 00        mov    $0x0, %edi
        91: R_X86_64_32 .rodata.str1.1+0xa
        \end{minted}
        现在将若干个 \verb|.o| 文件链接成可执行文件 \verb|done|。假设链接器已经确定 \verb|test.o| 的 \verb|.text| 节在 \verb|done| 中的起始地址为 \verb|ADDR(.text) = 0x400517|。

        \subqn 链接后,\verb|test.o| 中的 \verb|6b| 处的指令变为 \verb|done| 中如下的指令:
        \begin{minted}[frame=single, fontsize=\small]{text}
    ____________: e8 ___ ___ ___ ___         callq 400517 <bar>
        \end{minted}
        请补充以上五个空格(其中第一个空格是指令地址,之后四个空格是机器码)。
        \subqn \verb|test.o| 中 \verb|90| 处的指令变为 \verb|done| 中如下的指令:
        \begin{minted}[frame=single, fontsize=\small]{text}
    4005a7: bf 9e 06 40 00         mov $0x40069e, %edi
        \end{minted}
        则可执行文件 \verb|done| 中,\verb|.rodata.str1.1| 的起始地址为 \verb|0x|\rule{2.5cm}{0.25mm}。
        \qn 对 \verb|done| 使用 \verb|objdump|,发现有如下的函数:
        \begin{minted}[frame=single, fontsize=\small]{text}
    0000000000400430 <_start>
    000000000040054b <main>
    0000000000600ff0 <__libc_start_main@GLIBC_2.2.5>
        \end{minted}
        则 \verb|done| 的入口点地址是 \verb|0x|\rule{2.5cm}{0.25mm}。
    \end{problems}

\chapter{链接{---}往年考题}
    \begin{problems}
        \pro 下列关于链接技术的描述,错误的是:
        \begin{choices}
            \item 在 Linux 系统中,对程序中全局符号的不当定义,会在链接时刻进行报告。
            \item 在使用 Linux 的默认链接器时,如果有多个弱符号同名,那么会从这些弱符号中任选一个占用靠年间最大的符号。
            \item 编译时打桩需要能够访问程序的源代码,链接时打桩需要能够访问程序的可重定位对象文件,运行时打桩只需要能访问可执行目标文件。
            \item 链接器的两个主要任务是符号解析和重定位。符号解析将目标文件中的全局符号都绑定到唯一的定义,重定位确定每个符号的最终内存地址,并修改对那些目标的引用。
        \end{choices}
    \end{problems}