\chapter{网络和并发编程}\thispagestyle{empty}
    \begin{summary}
        \begin{compactitem}
            \item 了解网络的物理层、链路层、网络层、传输层和应用层的基本结构,知道每一层所对应的硬件及其特性(如交换机、路由器)。
            \item 知道什么是网络协议,熟悉网络包在传输过程中包装和拆解的过程。了解常见的几种应用层网络协议(HTTP、DNS)。知道 TCP 和 UDP 两种协议及其是否有连接、是否可靠等性质。
            \item 熟悉 IP 地址的结构和用途。知道域名和 IP 地址之间的对应关系。
            \item 掌握客户端{--}服务器模型和套接字的概念(地址、端口、协议),熟悉 Linux 中读写套接字的 API 并能编写简单的网络服务器。
            \item 了解编写并发服务器的三种手段。
            \item 熟悉线程的概念,知道 POSIX 线程的常见 API。会判断线程程序中,变量是否共享以及共享变量在内存中的性质。
            \item 熟悉互斥同步问题的概念。会绘制进度图并分析互斥区域和不安全区域。
            \item 熟练掌握使用信号量解决互斥锁(mutex)、生产者{--}消费者问题和读者{--}写者问题的方法,能灵活运用到实际代码中。知道线程提高并行性和同步带来的问题。
            \item 理解线程安全和可重入的概念,会判断代码是否为线程安全。
            \item 熟练掌握死锁的概念,会判断程序是否有死锁及如何解决。
        \end{compactitem}
    \end{summary}

    \begin{problems}
        \pro \verb|volatile| 修饰符保证定义的变量存放在内存中,而不总在寄存器中。请阅读下面的代码,在两个进程的地址空间中标出变量 \verb|gCount|、\verb|vCount| 与 \verb|lCount| 的位置。如果一个量出现多次,那么就标多次。
        \begin{table}[H]
            \tt
            \centering
            \begin{tabular}{ccc}
                \cline{1-1} \cline{3-3}
                \multicolumn{1}{|c|}{} & \multicolumn{1}{c|}{高地址} & \multicolumn{1}{c|}{} \\ \cline{1-1} \cline{3-3} 
                \multicolumn{1}{|c|}{...} & \multicolumn{1}{c|}{} & \multicolumn{1}{c|}{...} \\ \cline{1-1} \cline{3-3} 
                \multicolumn{1}{|c|}{共享库} & \multicolumn{1}{c|}{共享库} & \multicolumn{1}{c|}{共享库} \\ \cline{1-1} \cline{3-3} 
                \multicolumn{1}{|c|}{...} & \multicolumn{1}{c|}{} & \multicolumn{1}{c|}{...} \\ \cline{1-1} \cline{3-3} 
                \multicolumn{1}{|c|}{} & \multicolumn{1}{c|}{堆} & \multicolumn{1}{c|}{} \\ \cline{1-1} \cline{3-3} 
                \multicolumn{1}{|c|}{} & \multicolumn{1}{c|}{数据段} & \multicolumn{1}{c|}{} \\ \cline{1-1} \cline{3-3} 
                \multicolumn{1}{|c|}{} & \multicolumn{1}{c|}{代码段} & \multicolumn{1}{c|}{} \\ \cline{1-1} \cline{3-3} 
                \multicolumn{1}{|c|}{} & \multicolumn{1}{c|}{低地址} & \multicolumn{1}{c|}{} \\ \cline{1-1} \cline{3-3} 
                父进程 &  & 子进程
            \end{tabular}
        \end{table}
        \begin{minted}[frame=single, fontsize=\small, linenos]{c}
    long gCount = 0;
    void *thread(void *vargp) {
        volatile long vCount = *(long *) vargp;
        static long lCount = 0;
        gCount++; vCount++; lCount++;
        printf("%ld\n", gCount + vCount + lCount);
        return NULL;
    }
    int main() {
        long var; pthread_t tid1, tid2;
        scanf("%ld", &var);
        fork();
        pthread_create(&tid1, NULL, thread, &var);
        pthread_create(&tid2, NULL, thread, &var);
        pthread_join(tid1, NULL);
        pthread_join(tid2, NULL);
        return 0;
    }
        \end{minted}
        \sol \verb|gCount| 是全局变量,\verb|lCount| 是静态变量,都会存储在数据段,因此两个进程各一个,线程共用。\verb|vCount| 是局部变量,但要求存在内存中,所以是在栈中。答案如下:
        \begin{table}[H]
            \tt
            \centering
            \begin{tabular}{ccc}
                \cline{1-1} \cline{3-3}
                \multicolumn{1}{|c|}{vCount} & \multicolumn{1}{c|}{高地址} & \multicolumn{1}{c|}{vCount} \\
                \multicolumn{1}{|c|}{vCount} & \multicolumn{1}{c|}{} & \multicolumn{1}{c|}{vCount} \\ \cline{1-1} \cline{3-3} 
                \multicolumn{1}{|c|}{...} & \multicolumn{1}{c|}{} & \multicolumn{1}{c|}{...} \\ \cline{1-1} \cline{3-3} 
                \multicolumn{1}{|c|}{共享库} & \multicolumn{1}{c|}{共享库} & \multicolumn{1}{c|}{共享库} \\ \cline{1-1} \cline{3-3} 
                \multicolumn{1}{|c|}{...} & \multicolumn{1}{c|}{} & \multicolumn{1}{c|}{...} \\ \cline{1-1} \cline{3-3} 
                \multicolumn{1}{|c|}{} & \multicolumn{1}{c|}{堆} & \multicolumn{1}{c|}{} \\ \cline{1-1} \cline{3-3} 
                \multicolumn{1}{|c|}{gCount} & \multicolumn{1}{c|}{数据段} & \multicolumn{1}{c|}{gCount} \\
                \multicolumn{1}{|c|}{lCount} & \multicolumn{1}{c|}{} & \multicolumn{1}{c|}{lCount} \\ \cline{1-1} \cline{3-3} 
                \multicolumn{1}{|c|}{} & \multicolumn{1}{c|}{代码段} & \multicolumn{1}{c|}{} \\ \cline{1-1} \cline{3-3} 
                \multicolumn{1}{|c|}{} & \multicolumn{1}{c|}{低地址} & \multicolumn{1}{c|}{} \\ \cline{1-1} \cline{3-3} 
                父进程 &  & 子进程
            \end{tabular}
        \end{table}
        \pro 下面的程序会引发竞争(不考虑原子性问题)。一个可能的输出结果为 \verb|2 1 2 2|。解释输出这一结果的原因。
        \begin{minted}[frame=single, fontsize=\small, linenos]{c}
    long foo = 0, bar = 0;
    void *thread(void *vargp) {
        foo++; bar++;
        printf("%ld %ld ", foo, bar); fflush(stdout);
        return NULL;
    }
    int main() {
        pthread_t tid1, tid2;
        pthread_create(&tid1, NULL, thread, NULL);
        pthread_create(&tid2, NULL, thread, NULL);
        pthread_join(tid1, NULL);
        pthread_join(tid2, NULL);
        return 0;
    }
        \end{minted}
        \sol 线程 1 将 \verb|foo|、\verb|bar| 改为 1 以后被线程 2 打断,线程 2 将 \verb|foo| 改为 2 以后被线程 1 打断,线程 1 再输出 \verb|2 1|,线程 2 将 \verb|bar| 改为 2,然后输出了 \verb|2 2|。
        \pro 信号量 \verb|w, x, y, z| 均被初始化为 \verb|1|。下面的两个线程运行时可能会发生死锁。给出发生死锁的执行顺序。
        \begin{table}[H]
            \tt
            \centering
            \begin{tabular}{|c|c|}
                \hline
                线程 1 & P(w), P(x), P(y), P(z), V(w), V(x), V(y), V(z) \\ \hline
                线程 2 & P(x), P(z), P(y), P(w), V(x), V(y), V(w), V(z) \\ \hline
            \end{tabular}
        \end{table}
        \sol 可看出有 \verb|w, x| 的潜在反序。线程 1 先执行完 \verb|P(w)| 后线程 2 连续执行 \verb|P(x), P(z), P(y)|,随后二者就发生了相互等待。
        \pro 某次考试有 30 名学生与 1 名监考老师,该教室的门很狭窄,每次只能通过一人。考试开始前,老师和学生进入考场(有的学生来得比老师早),当人来齐以后,老师开始发放试卷。拿到试卷后,学生就可以开始答卷。学生可以随时交卷,交卷后就可以离开考场。当所有的学生都上交试卷以后,老师才能离开考场。
        请用信号量与 PV 操作,解决这个过程中的同步问题。以下所有空缺语句均为 PV 操作。

        程序使用的全局变量为 \verb|stu_count: int|,表示考场中的学生数量,初值为 \verb|0|。信号量有:
        \begin{compactitem}
            \item\ \verb|mutex_stu_count|,保护全局变量,初值为 \verb|1|。
            \item\ \verb|mutex_door|,保证门每次通过一人,初值为 \rule{2.5cm}{0.25mm}。
            \item\ \verb|mutex_all_present|,保证学生都到了,初值为 \rule{2.5cm}{0.25mm}。
            \item\ \verb|mutex_all_handin|,保证学生都交了,初值为 \rule{2.5cm}{0.25mm}。
            \item\ \verb|mutex_test[30]|,表示学生拿到了试卷,初值均为 \rule{2.5cm}{0.25mm}。
        \end{compactitem}
        代码如下:
        \begin{minted}[frame=single, fontsize=\small, linenos]{c}
    Teacher: // 老师
        ____________________
        从门进入考场
        ____________________
        ____________________ // 等待同学来齐
        for (i = 1; i <= 30; i++)
            ____________________ // 给 i 号学生发放试卷
        ____________________ // 等待同学将试卷交齐
        ____________________
        从门离开考场
        ____________________

    Student(x): // x 号学生
        ____________________
        从门进入考场
        ____________________
        P(mutex_stu_count);
        stu_count++;
        if (stu_count == 30)
            ____________________
        V(mutex_stu_count);
        P(mutex_test[i]) // 等待拿自己的卷子
        学生答卷
        P(mutex_stu_count);
        stu_count--;
        if (stu_count == 0)
            ____________________
        V(mutex_stu_count);
        ____________________
        从门离开考场
        ____________________
        \end{minted}
        \sol 此题属于一种读者{--}写者问题的变种。因为已经设计了信号量,就比较容易。答案如下:
        \begin{compactitem}
            \item\ \verb|mutex_stu_count| 初值为 \verb|1|。
            \item\ \verb|mutex_door| 初值为 \verb|1|。
            \item\ \verb|mutex_all_present| 初值为 \verb|0|。
            \item\ \verb|mutex_all_handin| 初值为 \verb|0|。
            \item\ \verb|mutex_test[30]| 初值均为 \verb|0|。
        \end{compactitem}
        代码:
        \begin{minted}[frame=single, fontsize=\small, linenos]{c}
    Teacher: // 老师
        P(mutex_door);
        从门进入考场
        V(mutex_door);
        P(mutex_all_present); // 等待同学来齐
        for (i = 1; i <= 30; i++)
            V(mutex_test[i]); // 给 i 号学生发放试卷
        P(mutex_all_handin); // 等待同学将试卷交齐
        P(mutex_door);
        从门离开考场
        V(mutex_door);

    Student(x): // x 号学生
        P(mutex_door);
        从门进入考场
        V(mutex_door);
        P(mutex_stu_count);
        stu_count++;
        if (stu_count == 30)
            V(mutex_all_present);
        V(mutex_stu_count);
        P(mutex_test[i]) // 等待拿自己的卷子
        学生答卷
        P(mutex_stu_count);
        stu_count--;
        if (stu_count == 0)
            V(mutex_all_handin);
        V(mutex_stu_count);
        P(mutex_door);
        从门离开考场
        V(mutex_door);
        \end{minted}
        \pro 假设每个资源都有一个互斥锁,以下设定是否可能导致死锁?说明理由。
            \qn 一个系统有四个相同类型的资源,它们被三个进程同时共享,每个进程最多需要两个资源一个系统有 $m$ 个相同类型的资源,它们被 $n$ 个进程同时共享。
            \qn 在每个时刻,进程只能至多申请或者释放一个资源。假设系统满足以下两个条件:每个进程所需的最大资源数量不超过 $m$,所有进程的最大资源需求数量之和小于 $m+n$ 个。
        \sol 都不会。假设 1 出现死锁,那么 4 个资源耗尽,而且每个进程都想要请求资源,但至少有一个进程有 2 个资源,矛盾。类似,假设 2 出现死锁,则全部资源都被消耗,而且每个进程都在请求资源,这样所需求的资源不少于 $m+n$ 个,矛盾。
    \end{problems}

\chapter{网络与并发编程{---}往年考题}\thispagestyle{empty}
    \begin{problems}
        \proy{2018} 下面有关计算机网络概念的叙述中,正确的是:
        \begin{choices}
            \item 大写字母的 Internet 用来描述互联网的一般概念,而小写字母的 internet 用来描述一种具体的实现,也就是全球 IP 互联网。
            \item 在一个基于集线器的以太网中,如果往一台主机发送一段数据帧,那么其他主机无法看到这个帧。
            \item IP 协议提供基本的命名方法和递送机制,因此我们能够借助 IP 协议,从一台主机往另一台主机发送包,即使两台主机不在同一个 LAN 内。
            \item 当一段数据通过路由器,从 LAN1 被发送到 LAN2 时,附加的互联网络包头和局域网帧头保持不变。
        \end{choices}
        \sol C。A 反了,Internet 是指全球 IP 互联网,是一个具体的实现。B 不对,集线器是广播的,大家都能看到。D 也不对,会发生反复的打包和拆包过程,因此帧头会有变化。
        \proy{2018} 下面有关套接字接口的叙述中,错误的是:
        \begin{choices}
            \item 套接字接口常常被用来创建网络应用。
            \item Windows 10 系统没有实现套接字接口。
            \item \verb|getaddrinfo()| 和 \verb|getnameinfo()| 可以被用于编写独立于特定版本的 IP 协议的程序。
            \item \verb|socket()| 函数返回的描述符,可以使用标准 Unix I/O 函数进行读写。
        \end{choices}
        \sol B,系统一般都实现了套接字接口。其他都是正确的。注意 D 不够准确,其返回的描述符是半打开的,并不是能马上读写。
        \proy{2018} 使用浏览器打开网页 \verb|www.pku.edu.cn| 的过程中,下列网络协议中,可能会被用到的网络协议有 \rule{2.5cm}{0.25mm} 个:DNS、TCP、IP、HTTP。
        \proy{2018} 下面关于线程安全和可重入的叙述中,哪一个是正确的?
        \begin{choices}
            \item 如果一个函数的所有参数都是值传递的且无返回值,该函数一定是可重入的。
            \item 函数的可重入版本一定比不可重入版本高效。
            \item 可重入函数一定是线程安全的。
            \item 以上说法都不正确。
        \end{choices}
        \sol C。A 不对,因为可以访问共享变量。B 不对,如果不需要同步则不一定。C 是对的,可重入函数真包含于线程安全的函数(举一个不可重入但线程安全的函数的例子?)。
        \proy{2018} 下列关于进程与线程的描述中,哪一个是不正确的?
        \begin{choices}
            \item 一个进程可以包含多个线程。
            \item 进程中的各个线程共享进程的代码、数据、堆和栈。
            \item 进程中的各个线程拥有自己的线程上下文。
            \item 线程的上下文切换比进程的上下文切换快。
        \end{choices}
        \proy{2018} 给定下列代码片段:
        \begin{minted}[frame=single, fontsize=\small, linenos]{c}
    char **ptr; /* global var */
    int main(int main, char *argv[]) {
        long i; pthread_t tid;
        char *msgs[2] = {" Hello from foo", " Hello from bar" };
        ptr = msgs;
        for (i = 0; i < 2; i++)
            Pthread_create(&tid, NULL, thread, (void *) i);
        Pthread_exit(NULL);
    }
    void *thread(void *vargp) {
        long myid = (long) vargp;
        static int cnt = 0;
        printf("[%ld]: %s (cnt=%d)\n", myid, ptr[myid], ++cnt);
        return NULL;
    }
        \end{minted}
        下列哪一组变量集合是对等线程 1 引用的?
        \begin{choices}
            \item \verb|ptr, cnt, i.m, msgs.m, myid.p0, myid.p1|
            \item \verb|ptr, cnt, msgs.m, myid.p0|
            \item \verb|ptr, cnt, msgs.m, myid.p1|
            \item \verb|ptr, cnt, i.m, msgs.m, myid.p1|
        \end{choices}
        \sol C。注意 \verb|i| 是传值的,因此没有引用;\verb|msg| 则是通过指针 \verb|ptr| 引用的。
        \proy{2018} 给定如下程序:
        \begin{minted}[frame=single, fontsize=\small, linenos]{c}
    #include <stdio.h>
    #include <pthread.h>
    int i = 0;
    int j = 0;
    void *do_stuff1(void *arg __attribute__((unused))) {
        int a;
        for (a = 0; a < 1000; a++) { i++; j++; }
        return NULL;
    }
    void *do_stuff2(void *arg __attribute__((unused))) {
        int a;
        for (a = 0; a < 1000; a++) { j++; i++; }
        return NULL;
    }
    int main() {
        pthread_t tid1, tid2;
        pthread_create(&tid1, NULL, do_stuff1, NULL);
        pthread_create(&tid2, NULL, do_stuff2, NULL);
        pthread_join(tid1, NULL);
        pthread_join(tid2, NULL);
        printf("%d, %d\n", i, j);
        return 0;
    }
        \end{minted}
            \qn 用以下元素的编号写出  \verb|i++| 编译后最可能的汇编代码。(a) \verb|mov| (b) \verb|add| (c) \verb|0x601040| (d) \verb|%eax| (e) \verb|$0x1|
            \qn 请分别回答该程序是否有可能输出如下结果,并简述原因。 (a) \verb|2000, 2000| (b) \verb|1500, 1500| (c) \verb|1000, 1000| (d) \verb|2, 2|
            \qn 某同学在学习了信号量之后,决定要让程序能稳定输出 \verb|2000, 2000|, 于是对程序进行了如下改写。
            \begin{minted}[frame=single, fontsize=\small, linenos]{c}
    #include <stdio.h>
    #include <pthread.h>
    int i = 0;
    int j = 0;
    sem_t si;
    sem_t sj;
    void *do_stuff1(void *arg __attribute__((unused))) {
        int a;
        for (a = 0; a < 1000; a++) {
            P(&si);
            P(&sj);
            i++;
            j++;
            V(&si);
            V(&sj);
        }
        return NULL;
    }
    void *do_stuff2(void *arg __attribute__((unused))) {
        int a;
        for (a = 0; a < 1000; a++) {
            P(&sj);
            P(&si);
            j++;
            i++;
            V(&si);
            V(&sj);
        }
        return NULL;
    }
    int main() {
        pthread_t tid1, tid2;
        sem_init(&si, 0, 1);
        sem_init(&sj, 0, 1);
        pthread_create(&tid1, NULL, do_stuff1, NULL); 
        pthread_create(&tid2, NULL, do_stuff2, NULL);
        pthread_join(tid1, NULL);
        pthread_join(tid2, NULL);
        printf("%d\n", i);
        return 0;
    }
        \end{minted}
        请问该同学的程序有什么潜在问题?为什么?如果对该同学的程序改动一处数字来消除上述问题,同时仍然保证输出结果稳定为 \verb|2000, 2000|,应该如何改动?回答:将 \rule{2.5cm}{0.25mm} (填行号)行的 \rule{2.5cm}{0.25mm}(填数字)改为 \rule{2.5cm}{0.25mm}(填数字)。
        \sol (1)问考察对课本内容的理解,根据题目的意思我们知道自增应该分 load, inc, store 三步(当然,本题考虑不周,实验表明不一定这样翻译!)。三句汇编为 (a)(c)(d)、(b)(e)(d)、(a)(d)(c)。

        (2)问比较有意思(但是只是理论上的,由于硬件和体系结构的缘故,实际情况和理论分析不会一致,所以本题是个错题。首先,\verb|2000, 2000| 当然是可能的(就是没有原子性问题的情况)。两个线程实际上执行了如下的
        \begin{center}
            \verb|i++, j++, i++, j++, ...| \\
            \verb|j++, i++, j++, i++, ...|
        \end{center}
        的序列。为什么会出现不足值的情况?就是说某个进程在执行load, inc, store 的间隙中,另一个线程对 \verb|i| 的操作都是全部无效的(会被写回覆盖)。我们可以构造 \verb|1500, 1500| 的情形。比如线程 1 读出 \verb|i| 后被 2 打断,2 进行了 500 次 \verb|j++| 和 \verb|i++|,此时\verb|i = 500, j = 500|。然后 1 写回,得到 \verb|i = 1, j = 500|。此时 1 再读 \verb|j|,又被打断,2 进行了 500 次的 \verb|j++| 和 \verb|i++|,得到 \verb|i = 501, j = 1000|。然后 1 写回,得到 \verb|i = 501, j = 501|。这时 2 已经结束,1 还有 \verb|i++| 和 \verb|j++| 各 999 个,所以得到 \verb|1500, 1500|。
        
        做到这里我们可以发现两个执行序列可以通过一种“break window”的方式来构造可能的输出。这样可以构造出 \verb|1001, 1001| 到 \verb|2000, 2000| 的所有可能,于是猜测(3)(4)不可能。
        
        让我们来考虑 1 的某个 \verb|i++| 被打断的情形,我们知道最坏情况是 2 在这个窗口中完成了 $n$ 次 \verb|i++|,然后都因为写回失效了,由于自增交替进行,所以这中间至少会有 $n-1$ 次的 \verb|j++| 被完成,而补上 1 被打断的那个 \verb|i++|,我们发现损失的自增最多不超过有效的自增数目。所以 \verb|2, 2| 绝对不可能。另外我们注意到开头的 \verb|i++| 和 \verb|j++| 至少有一个完全有效(也不会在被打断的过程中被用来抵消别的损失),结尾的两个也是这样(反证即可),所以 \verb|i + j| 至少是 $2000-1+2=2001$。因此 \verb|1000, 1000| 不可能。
        
        但是 \verb|1001, 1000| 是可能的,表明我们的估计是紧的。事实上我们可以考虑如下序列:
        \begin{center}
            \verb|i++, j++, i++, j++, ..., i++, j++| \\
            \verb|          j++, i++, j++, ..., i++, j++, i++|
        \end{center}
        首先让 1 完成一次 \verb|i++|,剩下的 999 个配对的 \verb|i++| 和 1000 对配对的 \verb|j++| 均可用以下方法使得每对只造成一次自增:1 load 之后被打断,2 再 load 之后又被打断,然后二者分别写回同一个自增 1 的值即可。结尾的 \verb|i++| 正常完成,所以 \verb|i = 999 + 2 = 1001, j = 1000|。

        (3)问很显然,出现了反序的加锁,所以有潜在的死锁的问题。解决方案是将任何一个信号量的初始值改为 $\geq 2$ 的值。
        \proy{2016} 关于 IP,以下说法正确的是:
        \begin{choices}
            \item IP 协议提供了可信赖的主机与主机之间的数据包传输能力。
            \item IP 地址的长度固定为 32 位。
            \item 一个域名可以映射到多个 IP,但一个 IP 不能被多个域名映射。
            \item 一个域名可以不映射到任何 IP。
        \end{choices}
        \proy{2016} 以下说法正确的是:
        \begin{choices}
            \item HTTP 协议规定服务器端使用 80 端口提供服务。
            \item 使用 TCP 来实现数据传输一定是可靠的。
            \item Internet 上的两台的主机要通信必须先建立端到端连接。
            \item 在 Linux 中只能通过 Socket 接口进行网络编程。
        \end{choices}
        \sol D。IP 属于网络层协议,其传输是 best effort,不保证传输可靠。IPv6 协议的地址长度为 128 位,B 错误。C 则是明显的,IP 地址和域名之间属于“关系”但不保证是“映射”,更不保证是单射或者满射。
        \proy{2016} 假设有一个 HTTPS(基于 HTTP 的一种安全的应用层协议)客户端程序想要 通过一个 URL 连接一个电子商务网络服务器获取一个文件,并且这个服务器 URL 的 IP 地址是已知的,以下哪种协议是一定不需要的?
        \begin{choices}
            \item HTTP
            \item TCP
            \item DNS
            \item SSL/TLS
        \end{choices}
        \proy{2016} 请阅读如下代码:
        \begin{minted}[frame=single, fontsize=\small, linenos]{c}
    sem_t s;
    int main() {
        int i;
        pthread_t tids[3];
        sem_init(&s, 0, 1);
        for (i = 0; i < 3; i++)
            pthread_create(&tids[i], NULL, justdoit, NULL);
        for (i = 0; i < 3; i++)
            pthread_join(&tids[i], NULL);
        return 0;
    }
    int j = 0;
    void *justdoit(void *arg) {
       P(&s);
       j = j + 1;
       V(&s);
       printf("%d\n", j);
    }
        \end{minted}
        下列哪一个输出结果是不可能的?
        \begin{choices}
            \item $(1, 3, 2)$
            \item $(2, 3, 2)$
            \item $(3, 3, 2)$
            \item $(2, 1, 2)$
        \end{choices}
        \sol D。信号量的操作导致 \verb|j| 的增加是原子的,也就是说实际上是 3 个 \verb|j++| 和 3 个 \verb|printf|(分两步)拓扑排序。这里我们需要假设 \verb|printf| 是先读取 \verb|j|,然后再输出,所以结果不一定是单调递增的。但无论如何,最后的输出结果一定会有一个是 3,因此 D 不可能。
        \proy{2016} 请阅读下列代码 \verb|badcnt.c|:
        \begin{minted}[frame=single, fontsize=\small, linenos]{c}
    /* Global shared variable */
    volatile long cnt = 0;
    /* Counter */
    int main(int argc, char **argv) {
        long niters;
        pthread_t tid1, tid2;
        niters = atoi(argv[1]);
        Pthread_create(&tid1, NULL, thread, &niters);
        Pthread_create(&tid2, NULL, thread, &niters);
        Pthread_join(tid1, NULL);
        Pthread_join(tid2, NULL);
        /* Check result */
        if (cnt != (2 * niters))
            printf("BOOM! cnt=%ld\n", cnt);
        else
            printf("OK cnt=%ld\n", cnt);
        exit(0);
    }
    /* Thread routine */
    void *thread(void *vargp) {
        long i, niters = *((long *) vargp);
        for (i = 0; i < niters; i++)
        cnt++;
        return NULL;
    }
        \end{minted}
        运行后可能产生如下结果:
        \begin{minted}[frame=single, fontsize=\small]{shell-session}
    $ ./badcnt 10000
    OK cnt=20000
        \end{minted}
        或者:
        \begin{minted}[frame=single, fontsize=\small]{shell-session}
    $ ./badcnt 10000
    BOOM! cnt=13051
        \end{minted}
        为什么会产生不同的结果?请分析产生不同结果的原因。
        \proy{2016} 某辆公交车的司机和售票员为保证乘客的安全,需要密切配合、协调工作。司机和售票员的工作流程如下所示。请编写程序,用 PV 操作来实现司机与售票员之间的同步。
        \begin{table}[H]
            \centering
            \begin{tabular}{|c|c|}
                \hline
                司机进程循环执行以下命令 & 售票员进程循环指令以下命令 \\ \hline
                \circled{1} & \circled{5} \\
                启动车辆 & 关门 \\
                \circled{2} & \circled{6} \\
                正常行驶 & 报站名或维持秩序 \\
                \circled{3} & \circled{7} \\
                到站停车 & 到站开门 \\
                \circled{4} & \circled{8} \\ \hline
            \end{tabular}
        \end{table}
            \qn 请设计若干信号量,给出每一个信号量的作用和初值。
            \qn 请将信号量对应的 PV 操作填写在代码中适当位置。注:每一标号处可以不填入语句(请标记为 X)或多条语句。
        \sol 此题没有互斥关系,有两对同步关系,即停车之后才能开门以及关门之后才能开车,所以要设计两个信号量:开车信号量(初始化为 0)和开门信号量(初始化为0)。这样即可填写代码:
        \begin{table}[H]
            \tt
            \centering
            \begin{tabular}{|c|c|}
                \hline
                司机进程循环执行以下命令 & 售票员进程循环指令以下命令 \\ \hline
                P(开车信号量) & X \\
                启动车辆 & 关门 \\
                X & V(开车信号量) \\
                正常行驶 & 报站名或维持秩序 \\
                X & P(开门信号量) \\
                到站停车 & 到站开门 \\
                V(开门信号量) & X \\ \hline
            \end{tabular}
        \end{table}
        \proy{2015} HTTP 协议中,哪个命令可以用来获取动态内容?
        \begin{choices}
            \item \verb|HEAD|
            \item \verb|GET|
            \item \verb|POST|
            \item \verb|PUT|
        \end{choices}
        \proy{2015} 下列关于计算机网络概念的说法中,哪一项是正确的?
        \begin{choices}
            \item HUB 会把它任意端口上接收到的帧只转发到它的目的地去。
            \item 当在不同的 LAN 中的主机 A 和主机 B 通信的过程中,他们的数据包中的 LAN frame header 不会变化。
            \item \verb|162.105.0.0| 是一个 B 类地址。
            \item 同一台主机每次进入相同的网络,通过动态地址分配的到的 IP 地址总是相同的。
        \end{choices}
        \proy{2015} 有如下代码:
        \begin{minted}[frame=single, fontsize=\small, linenos]{c}
    int counter = 0;
    void *thread(void *vargp) {
        int thread_var = ((int *) vargp);
        static int thread_counter = 0;
        thread_internal(thread_var);
        thread_counter++;
        return NULL;
    }

    int main(int argc, const char **argv) {
        int tid1, tid2;
        int var = atoi(argv[1]);
        Pthread_create(&tid1, NULL, thread, (void *) var);
        Pthread_create(&tid2, NULL, thread, (void *) var);
        Pthread_join(tid1, NULL);
        Pthread_join(tid2, NULL);
        return 0;
    }
        \end{minted}
        那么线程 \verb|tid1| 与线程 \verb|tid2| 可以共享的变量是
        \begin{choices}
            \item \verb|counter, var|
            \item \verb|counter, thread_counter|
            \item \verb|var, thread_counter|
            \item \verb|thread_var, thread_counter|
        \end{choices}
        \proy{2015} 有四个信号量,初值分别为:\verb|a = 1, b = 1, c = 1, d = 1|。
        \begin{table}[H]
            \tt
            \centering
            \begin{tabular}{|c|c|c|}
                \hline
                进程 1 & 进程 2 & 进程 3 \\ \hline
                P(a) & P(d) & P(d) \\
                P(d) & P(a) & P(c) \\
                P(c) & P(c) & P(b) \\
                P(b) & P(b) & P(a) \\
                V(c) & V(d) & V(c) \\
                V(b) & P(d) & V(b) \\
                V(d) & V(a) & V(a) \\
                V(a) & V(b) & V(d) \\
                & V(c) &  \\
                & V(d) &  \\ \hline
            \end{tabular}
        \end{table}
        下列哪两个线程并发执行时,一定不会发生死锁?
        \begin{choices}
            \item 1, 2
            \item 1, 3
            \item 2, 3
            \item 都可能发生死锁
        \end{choices}
        \sol D。1, 2 和 1, 3 都存在 \verb|a, d| 的反序,会出现死锁。2, 3 的死锁出现在 \verb|c, d| 的反序,比如 2 第二次要 \verb|P(d)| 时还持有 \verb|c| 的锁,但是 3 在获得 \verb|c| 的锁之前不会释放 \verb|d|。所以都是可能发生死锁的。
        \proy{2015} 以下关于网络的说法哪些是正确的?
        \begin{choices}
            \item 在 client-server 模型中,server 通常使用监听套接字 \verb|listenfd| 和多个 client 同时通信。
            \item 在 client-server 模型中,套接字是一种文件标识符。
            \item 准确地说,IP 地址是用于标识主机的 adapter(network interface card),并非用于标识主机。
            \item Web 是一种互联网协议。
            \item 域名和 IP 地址是一一对应的。
            \item Internet 是一种 internet。
        \end{choices}
        \proy{2015} 有三个线程 \verb|PA|、\verb|PB|、\verb|PC| 协作工作以解决文件打印问题。\verb|PA| 将记录从磁盘读入内存缓冲区 \verb|Buff1|,每执行一次读一个记录;\verb|PB| 将缓冲区 \verb|Buff1| 的内容复制到缓冲区 \verb|Buff2|,每执行一次复制一个记录;\verb|PC| 将缓冲区 \verb|Buff2| 的内容打印出来,每执行一次打印一个记录。缓冲区 \verb|Buff1| 可以放 4 个记录;缓冲区 \verb|Buff2| 可以放 8 个记录。请用信号量及 PV 操作实现上述三个线程以保证文件的正确打印。
        \begin{table}[H]
            \centering
            \begin{tabular}{|c|c|c|}
                \hline
                \verb|PA| 循环执行以下命令 & \verb|PB| 循环执行以下命令 & \verb|PC| 循环执行以下命令 \\ \hline
                \circled{1} & \circled{4} & \circled{7} \\
                从磁盘读入一个记录 & 从 \verb|Buff1| 中取出一个记录 & 从 \verb|Buff2| 中取出一个记录 \\
                \circled{2} & \circled{5} & \circled{8} \\
                将记录放入 \verb|Buff1| & 将记录放入 \verb|Buff2| & 打印 \\
                \circled{3} & \circled{6} &  \\ \hline
            \end{tabular}
        \end{table}
            \qn 请设计若干信号量,给出每一个信号量的作用和初值。
            \qn 请将信号量对应的 PV 操作填写在代码中适当位置。注:每一标号处可以不填入语句(请标记为 X)或多条语句。
        \sol 此题是二重生产者{--}消费者问题。有两个互斥关系,即两个缓冲区的访问;有两对同步关系,即两个缓冲区的空和满。这样,一共需要六个信号量,两个互斥,两组同步(空/满)。空信号量初始化为容量(4、8),满信号量初始化为 0。这样即可填写代码:
        \begin{table}[H]
            \tt
            \centering
            \begin{tabular}{|c|c|c|}
                \hline
                PA 循环执行 & PB 循环执行 & PC 循环执行 \\ \hline
                X & P(b1\_full) & P(b2\_full) \\
                从磁盘读入一个记录 & P(b1\_mutex) & P(b2\_mutex) \\
                P(b1\_empty) & 从 Buff1 中取出一个记录 & 从 Buff2 中取出一个记录 \\
                P(b1\_mutex) & V(b1\_mutex) & V(b2\_mutex) \\
                从 Buff1 中取出一个记录 & V(b1\_empty) & V(b2\_empty) \\
                V(b1\_mutex) & P(b2\_empty) & 打印 \\
                V(b1\_full) & P(b2\_mutex) &  \\
                & 将记录放入 Buff2 &  \\
                & V(b2\_mutex) &  \\
                & V(b2\_full) &  \\ \hline
            \end{tabular}
        \end{table}
        \proy{2014} 如果两个局域网高层分别采用 TCP/IP 协议和 SPX/IPX 协议,那么可以选择的互连设备应是
        \begin{choices}
            \item 网桥
            \item 集线器
            \item 路由器
            \item 交换机
        \end{choices}
        \sol C,因为协议不同,所以需要路由器来完成工作。另一方面,只有路由器属于网络层,能够处理 IP 协议。
        \proy{2014} 下面说法正确的是:
        \begin{choices}
            \item TCP 是一种可靠的无连接协议。
            \item UDP 是一种不可靠的无连接协议。
            \item Web 浏览器与 Web 服务器通信采用的协议是 HTML。
            \item 数字数据只能通过数字信号传输。
        \end{choices}
        \proy{2014} 对于如下 C 语言程序:
        \begin{minted}[frame=single, fontsize=\small, linenos]{c}
    #include "csapp.h"
    void *thread (void *arg) {
        printf("Hello World");
        Pthread_detach(pthread_self());
    }
    int main(void) {
        pthread_t tid;
        int sta;
        sta = Pthread_create(&tid, NULL, thread, NULL);
        if (sta == 0)
            printf("Oops, I can not create thread\n");
        exit(NULL);
    }
        \end{minted}
        在上述程序中,\verb|Pthread_detach| 函数的作用是
        \begin{choices}
            \item 使主线程阻塞以等待线程 thread 结束。
            \item 线程 thread 运行结束后会自动释放所有资源。
            \item 线程 thread 运行后主动释放 CPU 给其他线程。
            \item 线程 thread 运行后成为僵尸线程。
        \end{choices}
        \proy{2014} 两个线程中共享如下一段 C 代码:
        \begin{minted}[frame=single, fontsize=\small, linenos]{c}
    for(j = 0; j < N; j++)
        count += 2;
        \end{minted}
        假设其对应的汇编代码如下:
        \begin{table}[H]
            \centering
            \begin{tabular}{lll}
                & \verb|movq   (%rdi), %rcx| & \multirow{4}{*}{$\left. \begin{aligned} \\ \\ \\ \end{aligned} \right\}H_i$} \\
                & \verb|testq  %rcx, %rcx| &  \\
                & \verb|jle    .L2| &  \\
                & \verb|movl   $0, %eax| &  \\
                \verb|.L3:| &  &  \\
                & \verb|movq   count(%rip), %rdx| & $L_i$ \\
                & \verb|addq   $2, %rdx| & $U_i$ \\
                & \verb|movq   %rdx, count(%rip)| & $S_i$ \\
                & \verb|addq   $1, %rax| & \multirow{3}{*}{$\left. \begin{aligned} \\ \\ \end{aligned} \right\}T_i$} \\
                & \verb|cmpq   %rcx, %rax| &  \\
                & \verb|jne    .L3| &  \\
                \verb|.L2:| &  & 
            \end{tabular}
        \end{table}
        上面用 $H_i, L_i, U_i, S_i, T_i\,(i=1, 2)$ 划分了指令的几个部分,下标代表执行的线程。在下列指令顺序对应的轨迹线中,哪一个是安全轨迹线?
        \begin{choices}
            \item $H_1 \to H_2 \to L_2 \to L_1 \to U_2 \to U_1 \to S_1 \to S_2 \to T_1 \to T_2$
            \item $H_1 \to L_1 \to U_1 \to H_2 \to L_2 \to S_1 \to T_1 \to U_2 \to S_2 \to T_2$
            \item $H_2 \to L_2 \to U_2 \to H_1 \to S_2 \to L_1 \to T_2 \to U_1 \to S_1 \to T_1$
            \item $H_2 \to L_2 \to H_1 \to L_1 \to U_1 \to U_2 \to S_2 \to T_2 \to S_1 \to T_1$
        \end{choices}
        \sol C。这里的不安全问题就是 $LUS$ 必须一步执行到位,否则出错(一个线程的 $S$ 必须在另一个线程 $L$ 之前)。所以如果存在 $L_i, U_i$ 被打断给另一边的 $L_i, U_i$,就不是安全的。A、B、D 都出现了这样的问题。
        \proy{2014} 以下问题默认为 IPv4 协议。
            \qn 一个服务器拥有四个独立的固定 IP 地址,那么它在 web 应用端口 80,理论上可以最多再监听多少个来自一个客户端独立的 socket 连接(设每个客户端只有一个固定 IP 地址)?
            \qn 设客户端 IP 为 \verb|162.105.192.178|,内网 IP 为 \verb|192.168.100.121|。HTTP 服务器端 IP 为 \verb|208.216.181.15|。服务器使用的是默认监听端口号。指出下面这个网页浏览器应用的 connection socket pair 有什么错误,并简要说明原因:客户端 \verb|192.168.100.121:15321|,服务器 \verb|208.216.181.15:25|。
        \proy{2014} 桌子上有一个水果盘,恰能容纳一个水果。一家四口人中,爸爸专门往盘子里放苹果,妈妈专门往盘子里放橘子;儿子专等盘子里的苹果吃,女儿专等盘子里的橘子吃。请用信号量及 PV 操作实现上述四个过程以保证家庭和睦。
        \begin{table}[H]
            \centering
            \begin{tabular}{|c|c|c|c|}
                \hline
                爸爸循环执行 & 妈妈循环执行 & 儿子循环执行 & 女儿循环执行 \\ \hline
                准备一个苹果 & 准备一个橘子 & \circled{5} & \circled{7} \\
                \circled{1} & \circled{3} & 从果盘中拿出苹果 & 从果盘中拿出橘子 \\
                往盘子里放一个苹果 & 往盘子里放一个橘子 & \circled{6} & \circled{8} \\
                \circled{2} & \circled{4} & 吃苹果 & 吃橘子 \\ \hline
            \end{tabular}
        \end{table}
            \qn 请设计若干信号量,给出每一个信号量的作用和初值。
            \qn 请将信号量对应的 PV 操作填写在代码中适当位置。注:每一标号处可以不填入语句(请标记为 X)或多条语句。
        \proy{2013} 下列关于计算机网络概念的说法中,哪一项是正确的?
        \begin{choices}
            \item 全球最大的计算机网络是互联网 Internet,所以计算机网络协议是
            Internet Protocol 即 IP 协议。
            \item 计算机之间的网络通信是一个机器上的一个 process(如 client process)与另一个机器上的 process(如 server process)之间的通信。
            \item 网络应用程序有默认的端口号,大部分应用的端口号可以修改,而少部分知名应用如 Web 服务程序的端口号 80 是无法修改的。
            \item 一个域名只能对应一个 IP 地址,而一个 IP 地址可以对应多个域名。
        \end{choices}
        \proy{2013} 在 client-server 模型中,一个连接可以由 IP 地址和端口号的组合来表示。假设一个访问网页服务器的应用,客户端 IP 地址为 \verb|128.2.194.24|,目标服务器端 IP 地址为 \verb|208.216.181.15|,用户设置的代理服务器 IP 地址为 \verb|155.232.108.39|。目标服务器同时提供网页服务(默认端口 80)和邮件服务(默认端口 25)。当客户端向目标服务器发送访问网页的请求时,下面 connection socket pairs 正确的一组是
        \begin{choices}
            \item 客户端请求 \verb|(128.2.194.242:25, 155.232.108.39:80)| \\ 代理请求 \verb|(128.2.194.242:51213, 208.216.181.15:80)|
            \item 客户端请求 \verb|(128.2.194.242:51213, 155.232.108.39:80)| \\ 代理请求 \verb|(128.2.194.242:12306, 208.216.181.15:80)|
            \item 客户端请求 \verb|(128.2.194.242:25, 208.216.181.15:80)| \\ 代理请求 \verb|(155.232.108.39:51213, 208.216.181.15:80)|
            \item 客户端请求 \verb|(128.2.194.242:51213, 208.216.181.15:80)| \\ 代理请求 \verb|(155.232.108.39:12306, 208.216.181.15:80)|
        \end{choices}
        \proy{2013} 下列代码输出正确的是:
        \begin{minted}[frame=single, fontsize=\small, linenos]{c}
    void *th_f(void *arg) {
       printf("Hello World");
       pthread_exit(0);
    }
    int main(void) {
        pthread_t tid;
        int st;
        st = pthread_create(&tid, NULL, th_f, NULL);
        if (st < 0) {
            printf("Oops, I can not create thread\n");
            exit(-1);
        }
        sleep(1);
        exit(0);
    }
        \end{minted}
        \begin{choices}
            \item Oops, I can not create thread
            \item Hello World \\ Oops, I can not create thread
            \item Hello World
            \item 不输出任何信息
        \end{choices}
        \proy{2013} 以下问题默认为 IPv4 协议。
            \qn 一个服务器拥有两个独立的固定 IP 地址,那么它在 web 应用端口 80 上最 多可以监听多少个独立的 socket 连接?
            \qn 该服务器在所有有web应用端口上最多可以监听多少个独立的 socket 连接?
        \proy{2013} 某个城市为了解决市中心交通拥堵的问题,决定出台一项交通管制措施,对进入市中心区的机动车辆实行单双日限制行驶。具体要求是,逢单日,只允许车辆牌号号码为单数的机动车进入市中心区;逢双日,只允许车辆牌号号码为双数的机动车进入市内中心区。
        
        有一个进入市中心区的交通路口,进入该路口的道路有一条,离开该路口的道路有两条,其中一条是通往市中心区的道路,而另一条是绕过市中心区的环路,在进入路口处设置了自动识别车辆牌号的识别设备与放行栅栏控制设备。
        
        在单日,遇到单号车辆进入路口车辆号码识别区,号码识别设备打开通往市中心区道路的放行栅栏;而遇到双号车辆,则打开绕过市中心区环路的放行栅栏。反之亦然。显然,只有在该路口车辆号码识别区中无车时,才允许一辆车进入车辆号码识别区。同时为了防止有车辆混过路口,两个放行栅栏平时处于关闭状态,只有在车辆号码识别区中的车辆已被识别出单双号之后,放行栅栏才会在识别设备的控制下,打开对应的放行栅栏,在车辆通过之后,该放行栅栏自行关闭。

        请编写程序,用 PV 操作来实现车牌号检查和两个放行栅栏之间的同步。
        \begin{table}[H]
            \centering
            \begin{tabular}{|c|c|c|}
                \hline
                检查车辆牌号线程循环执行 & 市区放行栅栏线程循环执行 & 环路放行栅栏线程循环执行 \\ \hline
                车辆到达识别路口 & \circled{4} & \circled{6} \\
                \circled{1} & 允许车辆进入市中心区 & 允许车辆绕行环路 \\
                车辆进入号码识别区 & \circled{5} & \circled{7} \\
                如果号码为奇数 \circled{2} &  &  \\
                否则 \circled{3} &  &  \\ \hline
            \end{tabular}
        \end{table}
            \qn 请设计若干信号量,给出每一个信号量的作用和初值。
            \qn 请将信号量对应的 PV 操作填写在代码中适当位置。注:每一标号处可以不填入语句(请标记为 X)或多条语句。
    \end{problems}