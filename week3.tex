\chapter{体系结构初步}
    \begin{summary}
        \begin{compactitem}
            \item 了解体系结构的简要发展史,熟悉 RISC、CISC 的区别和应用。了解基本的门电路、算术运算器、选择器、触发器和寄存器的原理和实现。
            \item 熟悉 Y86 “体系结构”中,各种指令的编码规则(包括操作数、指令子类型等),会将汇编代码和指令编码相互转换。简单了解 MIPS 体系结构。
        \end{compactitem}
    \end{summary}

    \begin{problems}
        \pro 下列描述更符合(早期)RISC 还是 CISC?
            \qn 指令机器码长度固定。
            \qn 指令类型多、功能丰富。
            \qn 不采用条件码。
            \qn 实现同一功能,需要的汇编代码较多。
            \qn 译码电路复杂。
            \qn 访存模式多样。
            \qn 参数、返回地址都使用寄存器进行保存。
            \qn x86-64。
            \qn MIPS。
            \qn 广泛用于嵌入式系统。
            \qn 已知某个体系结构使用 \verb|add R1, R2, R3| 来完成加法运算。当要将数据从寄存器 \verb|S| 移动至寄存器 \verb|D| 时,需要使用 \verb|add S, #ZR, D| 进行操作(\verb|#ZR| 是一个恒为 0 的寄存器)。
            \qn 已知某个体系结构提供了 \verb|xlat| 指令,它以一个固定的寄存器 \verb|A| 为基地址,以另一个固定的寄存器 \verb|B| 为偏移量,在 \verb|A| 对应的数组中取出下标为 \verb|B| 的项的内容,放回寄存器 \verb|A| 中。
    \end{problems}