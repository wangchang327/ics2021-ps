\chapter{位级表示}
    \begin{problems}
        \pro 下面关于 IEEE 浮点数标准说法正确的是哪个?
		\begin{choices}
			\item 在位数一定的情况下,不论怎么分配 exponent bits 和 fraction bits,所能表示的数的个数是不变的。
			\item 若甲类浮点数有 10 位,乙类浮点数有 11 位,那么甲所能表示的最大数一定比乙小。
			\item 若甲类浮点数有 10 位,乙类浮点数有 11 位,那么甲所能表示的最小正数一定比乙小。
		    \item “\texttt{0111000}”可能是 7 位浮点数的 \texttt{NaN} 表示。
		\end{choices}
		\sol A 错误,因为 \texttt{unsigned} 确实一样多,但是 \texttt{NaN} 的数目不一样,小数位越多 \texttt{NaN} 越多。 依照前述,最大数,最大正数,最小正数只和指数有关($\epsilon>0$ 不影响比较),所以 B、C 都错误。D 正确,比如“\texttt{0\_1\_11000}”就是一个表示方法。
        \pro 对于 IEEE 浮点数,如果减少 1 位指数位,将其用于小数部分,下列叙述正确的是哪个?
		\begin{choices}
			\item 能表示更多数量的实数值,但实数值取值范围比原来小了。
			\item 能表示的实数数量没有变化,但数值的精度更高了。
			\item 能表示的最大实数变小,最小的实数变大,但数值的精度更高。
			\item 以上说法都不正确。
        \end{choices}
		\sol 按照前一题的分析:减少指数位,导致范围变窄;增加小数位,提高精度,\texttt{NaN}变多,表示数量变少。选 C。
        \pro 对 $x=1 \frac{1}{8}$ 和 $y=1 \frac{3}{8}$ 进行小数点后两位取整(rounding to nearest even),结果是什么?
    \end{problems}