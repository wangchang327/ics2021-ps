\chapter{位级表示}
	\begin{summary}
		\begin{compactitem}
			\item 知道整数、浮点数在系统的存储方式。
			\item 熟练掌握整数、浮点数的位级表示规则,快速完成其同十进制数的转换。
			\item 理解整数、浮点数规范中的各类特殊数的计算方法及其性质。
			\item 运用浮点数舍入的规则进行运算,知道类型转换的基本规则,能针对整数、浮点数的一些常见“反常情况”进行判断。
			\item *了解位级操作中常见的技巧和套路,会用位运算处理非常规的编程要求。
		\end{compactitem}
	\end{summary}

    \begin{problems}
		\pro 在 x86-64 机器上,定义 \texttt{unsigned int A = 0x123456}。请画出 \texttt{A} 在内存中的存储方式:
		\begin{table}[H]
			\centering
			\begin{tabular}{|c|c|c|c|c|c|c|c|}
				\hline
				... & 低地址 & \multicolumn{4}{c|}{\texttt{A}} & 高地址 & ... \\ \hline
				\multicolumn{2}{|c|}{...} & {\qquad \qquad} & {\qquad \qquad} & {\qquad \qquad} & {\qquad \qquad} & \multicolumn{2}{c|}{...} \\ \hline
			\end{tabular}
		\end{table}
		定义 \texttt{unsigned short B[2] = \{0x1234, 0x5678\}}。请画出 \texttt{B} 在内存中的存储方式:
		\begin{table}[H]
			\centering
			\begin{tabular}{|c|c|c|c|c|c|c|c|}
				\hline
				... & 低地址 & \multicolumn{4}{c|}{\texttt{B}} & 高地址 & ... \\ \hline
				\multicolumn{2}{|c|}{...} & {\qquad \qquad} & {\qquad \qquad} & {\qquad \qquad} & {\qquad \qquad} & \multicolumn{2}{c|}{...} \\ \hline
			\end{tabular}
		\end{table}
		\pro 下列代码的目的是将字符串 \texttt{A} 的内容复制到字符串 \texttt{B},覆盖 \texttt{B} 原有的内容,并输出“Hello World”;但实际运行输出是“Buggy Codes”。尝试找到代码中的错误。
		\begin{minted}[frame=single, fontsize=\small]{c}
    int main() {
        char A[12] = "Hello World";
        char B[12] = "Buggy Codes";
        int pos;
        for (pos = 0; pos - sizeof(B) < 0; pos++)
            B[pos] = A[pos];
        printf("%s\n", B);
    }
		\end{minted}
		\pro 在 x86-64 机器上,有下列 C 代码:
		\begin{minted}[frame=single, fontsize=\small]{c}
    int main() {
        unsigned int A = 0x11112222;
        unsigned int B = 0x33336666;
        void *x = (void *)&A;
        void *y = 2 + (void *)&B;
        unsigned short P = *(unsigned short *)x;
        unsigned short Q = *(unsigned short *)y;
        printf("0x%04x", P + Q);
        return 0;
    }
		\end{minted}
		运行该代码,结果是什么?
		\pro 在 x86-64 机器上,有下列 C 代码:
		\begin{minted}[frame=single, fontsize=\small]{c}
    int main() {
        char A[12] = "11224455";
        char B[12] = "11445577";
        void *x = (void *)&A;
        void *y = 2 + (void *)&B;
        unsigned short P = *(unsigned short *)x;
        unsigned short Q = *(unsigned short *)y;
        printf("0x%04x", Q - P);
        return 0;
    }
		\end{minted}
		运行该代码,结果是什么?
		\pro 在 x86-64 机器上,有如下的定义:
		\begin{minted}[frame=single, fontsize=\small]{c}
    int x = ...; // 表达式 A
    int y = ...; // 表达式 B
    unsigned int ux = x;
    unsigned int uy = y;
		\end{minted}
		判断下表中的表达式是否等价:
		\begin{table}[H]
			\centering
			\begin{tabular}{|c|c|c|}
				\hline
				序号 & 表达式 \verb|A| & 表达式 \verb|B| \\ \hline
				1 & \verb|x > y| & \verb|ux > uy| \\ \hline
				2 & \verb+(x > 0) || (x < ux)+ & \verb|1| \\ \hline
				3 & \verb|x ^ y ^ x ^ y ^ x| & \verb|x| \\ \hline
				4 & \verb|((x >> 1) << 1) <= x| & \verb|1| \\ \hline
				5 & \verb|((x / 2) * 2) <= x| & \verb|1| \\ \hline
				6 & \verb|x ^ y ^ (~x) - y| & \verb|y ^ x ^ (~y) - x| \\ \hline
				7 & \verb|(x == 1) && (ux - 2 < 2)| & \verb|(x == 1) && ((!!ux) - 2 < 2)| \\ \hline
			\end{tabular}
		\end{table}
		\begin{hint}
			减法的运算优先级比按位异或高。布尔运算的结果都是有符号数。
		\end{hint}
		\pro 现有一个二进制浮点的表示规则,其中 $E$ 为指数部分(3 比特),bias为 3;$M$ 为小数部分(5 比特),采用二进制补码表示形式,且取值 $0.5 \leq |M|<1$,$s$ 是浮点的符号位。该形式包含一个值为 1 的隐藏位。问 $+5_{10}$ 在该表示下的值是下列哪一个?
		\begin{choices}
			\item \texttt{010001100}
			\item \texttt{010100100}
			\item \texttt{011011010}
			\item \texttt{011110101}
		\end{choices}
        \pro 下面关于 IEEE 浮点数标准说法正确的是哪个?
		\begin{choices}
			\item 在位数一定的情况下,不论怎么分配 exponent bits 和 fraction bits,所能表示的数的个数是不变的。
			\item 若甲类浮点数有 10 位,乙类浮点数有 11 位,那么甲所能表示的最大数一定比乙小。
			\item 若甲类浮点数有 10 位,乙类浮点数有 11 位,那么甲所能表示的最小正数一定比乙小。
		    \item “\texttt{0111000}”可能是 7 位浮点数的 \texttt{NaN} 表示。
		\end{choices}
        \pro 对于 IEEE 浮点数,如果减少 1 位指数位,将其用于小数部分,下列叙述正确的是哪个?
		\begin{choices}
			\item 能表示更多数量的实数值,但实数值取值范围比原来小了。
			\item 能表示的实数数量没有变化,但数值的精度更高了。
			\item 能表示的最大实数变小,最小的实数变大,但数值的精度更高。
			\item 以上说法都不正确。
        \end{choices}
		\pro 假设某浮点数格式为 1 位符号、3 位阶码、4 位小数。下表给出了用该格式表达的浮点数 $(-1)^SM \cdot 2^E$ 与其二进制表示的关系。完成下表。
		\begin{table}[H]
			\centering
			\begin{tabular}{|c|c|c|c|c|}
				\hline
				描述 & 二进制表示 & $M$(写成分数) & $E$ & $f$ \\ \hline
				负零 &  & / & / & \texttt{-0.0} \\ \hline
				/ & \verb|01000101| &  &  &  \\ \hline
				最小的非规格化负数 &  &  &  &  \\ \hline
				最大的规格化正数 &  &  &  &  \\ \hline
				一 &  &  &  & \texttt{1.0} \\ \hline
				/ &  &  &  & \texttt{5.5} \\ \hline
				$+\infty$ &  & / & / & / \\ \hline
			\end{tabular}
		\end{table}
		\pro 假设浮点数格式 A 为 1 位符号、3 位阶码、4 位小数,浮点数格式 B 为 1 位符号、4 位阶码、3 位小数。回答下列问题。
			\qn 格式 A 中有多少个二进制表示对应于正无穷大?
			\qn 考虑能精确表示的实数的最大绝对值。A 比 B 大还是比 B 小,还是两者一样?
			\qn 考虑能精确表示的实数的最小非零绝对值。A 比 B 大还是比 B 小,还是两者一样?
			\qn 考虑能精确表示的实数的个数。A 比 B 多还是比 B 少,还是两者一样?
		\pro 判断下列说法的正确性。
			\qn 对于任意的单精度浮点数 \texttt{a} 和 \texttt{b},如果 \texttt{a > b},那么 \texttt{a + 1 > b}。
			\qn 对于任意的单精度浮点数 \texttt{a} 和 \texttt{b},如果 \texttt{a > b},那么 \texttt{a + b > b + b}。
			\qn 对于任意的单精度浮点数 \texttt{a} 和 \texttt{b},如果 \texttt{a} > \texttt{b},那么 \texttt{a + 1 > b + 1}。
			\qn 对于任意的双精度浮点数 \texttt{d},如果 \texttt{d < 0},那么 \texttt{d * d > 0}。
			\qn 对于任意的双精度浮点数 \texttt{d},如果 \texttt{d < 0},那么 \texttt{d * 2 < 0}。
			\qn 对于任意的双精度浮点数 \texttt{d},\texttt{d == d}。
			\qn 将 \texttt{float} 转换成 \texttt{int} 时,既有可能造成舍入,又有可能造成溢出。
		\pro 遵循 IEEE 754 浮点数标准,考虑下列代码:
		\begin{minted}[frame=single, fontsize=\small]{c}
    for (int x = 0; ; x++) {
        float f = x;
        if (x != (int)f) {
            printf("%d", x);
            break;
        }
    }
		\end{minted}
		试问代码的运行结果是什么?或者死循环?
		\pro 遵循 IEEE 754 浮点数标准,考虑下列代码:
		\begin{minted}[frame=single, fontsize=\small]{c}
    int x = 33554466; // 2^25 + 34
    int y = x + 8;
    for ( ; x < y; x++) {
        float f = x;
        printf("%d ", x - (int)f);
    }
		\end{minted}
		写出程序的运行结果。
    \end{problems}