\chapter{讲座课}
    \begin{problems}
        \pro 下列有关微处理器及其设计自动化的叙述不妥当的是:
        \begin{choices}
            \item 芯片设计自动化英文全称为 Electronic Design Automation,是一个涉及物理学、计算机科学、应用数学的跨领域研究。
            \item 微处理器的最早的先驱一般认为是 Intel 公司 1971 年出产的 4004 型,它有大约 2300 个晶体管,位宽为 4。
            \item 硬件描述语言有 \verb|SystemC, Verilog, VHDL, HCL| 等;将这些语言翻译为实际生产中的描述的工具称为“硬件编译器”,翻译过程一般称为 synthesis。
            \item EDA 过程中面临许多问题,包括 placement、routing 等,这些实际问题的求解中的困难主要是问题规模太大。
        \end{choices}
        \pro 摩尔定律(Moore's law)是由英特尔创始人之一戈登·摩尔提出的。根据摩尔定律,在过去几十年中,单块集成电路的集成度大约每 \rule{1cm}{0.25mm} 个月翻一番,性能达到原来的 \rule{1cm}{0.25mm} 倍。
        \begin{choices}
            \item 24、2
            \item 24、1.5
            \item 18、2
            \item 18、1.5
        \end{choices}
        \pro 下列有关 placement 问题的叙述,错误的是:
        \begin{choices}
            \item 问题的数学模型是在一个二维平面上放置若干几何块,使得它们互不重叠且相互之间的距离之和(wirelength)尽量小。
            \item 历史上出现过很多针对该问题的算法,例如划分、模拟退火、最小割、优化方法等,其中模拟退火结果不好,划分法效率较低。
            \item 使用优化方法的一个主要困难是优化问题的目标和约束光滑性不足,或者是非凸的,解决方案之一是采用一些光滑函数近似。
            \item 若用优化方法求解,则可以引入势函数、梯度下降、深度神经网络等方法启发式地求解。
        \end{choices}
        \pro 下列有关 routing 问题的叙述,错误的是:
        \begin{choices}
            \item Routing 一般是在各 cell 的 placement 后决定的,问题的数学模型是决定各个 cell 之间的最短路,有多项式时间的算法。
            \item 问题常见的约束包括:关键 cell 网络中延迟界、总 wirelength 长度、芯片面积和其他几何约束、和光刻有关的约束等等。
            \item 最短路问题可以用大家熟知的迷宫问题建模,多项式时间算法可以是 Lee 算法(即 BFS)。在问题规模较大时可以用 Aker 的数据结构等进行优化。
            \item Routing 问题在引脚较多时需要引入斯坦纳树等结构,具体而言问题是 $\mathsf{NP}$-完全的,不过可以用深度强化学习来启发式求解。
        \end{choices}
        \pro 北京大学计划筹建新校区,请你帮忙规划新校区的建筑分布。假设新校区在 $xy$ 平面上,其边界是多边形曲线,每段都平行于 $x$ 或 $y$ 轴。已知宿舍楼 50 幢、食堂 10 个、教学楼 10 幢、绿地 20 块,均为边界平行于 $x, y$ 轴的已知大小的矩形。要求宿舍楼、教学楼和食堂尽可能靠近,且不同宿舍楼、教学楼尽量靠近不同的食堂,不同建筑不能重叠。试为问题建立基于优化的数学模型。
    \end{problems}