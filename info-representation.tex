\chapter{位级表示}\thispagestyle{empty}
	\begin{summary}
		\begin{compactitem}
			\item 知道整数、浮点数在系统的存储方式(字节序)。
			\item 熟练掌握整数、浮点数的位级表示规则,快速完成其同十进制数的相互转换。
			\item 理解整数、浮点数规范中的各类特殊数的计算方法及其性质。
			\item 运用浮点数舍入的规则进行运算,知道类型转换的基本规则,能针对整数、浮点数的一些常见“反常情况”进行判断。
		\end{compactitem}
	\end{summary}

	\begin{problems}
		\pro 在 x86-64 机器上,定义 \texttt{unsigned int A = 0x123456}。请画出 \texttt{A} 在内存中的存储方式:
		\begin{table}[H]
			\centering
			\begin{tabular}{|c|c|c|c|c|c|c|c|}
				\hline
				... & 低地址 & \multicolumn{4}{c|}{\texttt{A}} & 高地址 & ... \\ \hline
				\multicolumn{2}{|c|}{...} & {\qquad \qquad} & {\qquad \qquad} & {\qquad \qquad} & {\qquad \qquad} & \multicolumn{2}{c|}{...} \\ \hline
			\end{tabular}
		\end{table}
		定义 \texttt{unsigned short B[2] = \{0x1234, 0x5678\}}。请画出 \texttt{B} 在内存中的存储方式:
		\begin{table}[H]
			\centering
			\begin{tabular}{|c|c|c|c|c|c|c|c|}
				\hline
				... & 低地址 & \multicolumn{4}{c|}{\texttt{B}} & 高地址 & ... \\ \hline
				\multicolumn{2}{|c|}{...} & {\qquad \qquad} & {\qquad \qquad} & {\qquad \qquad} & {\qquad \qquad} & \multicolumn{2}{c|}{...} \\ \hline
			\end{tabular}
		\end{table}
		\sol 根据小端法可知分别为 \verb|0x56 0x34 0x12 0x00|、\verb|0x34 0x12 0x78 0x56|。
		\pro 在 x86-64 机器上,有下列 C 代码:
		\begin{minted}[frame=single, fontsize=\small, linenos]{c}
    int main() {
        unsigned int A = 0x11112222;
        unsigned int B = 0x33336666;
        void *x = (void *)&A;
        void *y = 2 + (void *)&B;
        unsigned short P = *(unsigned short *)x;
        unsigned short Q = *(unsigned short *)y;
        printf("0x%04x", P + Q);
        return 0;
    }
		\end{minted}
		运行该代码,结果是什么?
		\sol 根据小端法,\verb|A| 在内存中从高地址到低地址分别是 \verb|11 11 22 22|,得到 \verb|P = 0x2222|,同理 \verb|Q = 0x3333|,结果为 \verb|0x5555|。
		\pro 在 x86-64 机器上,有下列 C 代码:
		\begin{minted}[frame=single, fontsize=\small, linenos]{c}
    int main() {
        char A[12] = "11224455";
        char B[12] = "11445577";
        void *x = (void *)&A;
        void *y = 2 + (void *)&B;
        unsigned short P = *(unsigned short *)x;
        unsigned short Q = *(unsigned short *)y;
        printf("0x%04x", Q - P);
        return 0;
    }
		\end{minted}
		运行该代码,结果是什么?
		\sol 答案为 \verb|0x0303|,分析方法和前一题一样。
		\pro 在 x86-64 机器上,有如下的定义:
		\begin{minted}[frame=single, fontsize=\small, linenos]{c}
    int x = ...; // 表达式 A
    int y = ...; // 表达式 B
    unsigned int ux = x;
    unsigned int uy = y;
		\end{minted}
		判断下表中的表达式是否等价:
		\begin{table}[H]
			\centering
			\begin{tabular}{|c|c|c|}
				\hline
				序号 & 表达式 \verb|A| & 表达式 \verb|B| \\ \hline
				1 & \verb|x > y| & \verb|ux > uy| \\ \hline
				2 & \verb+(x > 0) || (x < ux)+ & \verb|1| \\ \hline
				3 & \verb|x ^ y ^ x ^ y ^ x| & \verb|x| \\ \hline
				4 & \verb|((x >> 1) << 1) <= x| & \verb|1| \\ \hline
				5 & \verb|((x / 2) * 2) <= x| & \verb|1| \\ \hline
				6 & \verb|x ^ y ^ (~x) - y| & \verb|y ^ x ^ (~y) - x| \\ \hline
				7 & \verb|(x == 1) && (ux - 2 < 2)| & \verb|(x == 1) && ((!!ux) - 2 < 2)| \\ \hline
			\end{tabular}
		\end{table}
		\begin{hint}
			减法的运算优先级比按位异或高。布尔运算的结果都是有符号数。
		\end{hint}
		\sol 1 取 \verb|x = 1, y = -1| 即不正确;2 取 \verb|x = -1| 即不正确;3 正确,利用异或的交换律、结合律,以及 \verb|x ^ x == 0|(视为模 2 加法即可);4 正确,即使是对负数;5 不正确,负奇数该运算向 0 舍入;6 正确,\verb|(~x) – y| 也就是 \verb|(~x) + (~y) + 1|,注意运算优先级;7 不正确,\verb|!!ux| 是有符号数。
		\pro 下列代码的目的是将字符串 \texttt{A} 的内容复制到字符串 \texttt{B},覆盖 \texttt{B} 原有的内容,并输出“Hello World”;但实际运行输出是“Buggy Codes”。尝试找到代码中的错误。
		\begin{minted}[frame=single, fontsize=\small, linenos]{c}
    int main() {
        char A[12] = "Hello World";
        char B[12] = "Buggy Codes";
        int pos;
        for (pos = 0; pos - sizeof(B) < 0; pos++)
            B[pos] = A[pos];
        printf("%s\n", B);
    }
		\end{minted}
		\sol \verb|sizeof| 返回的结果是无符号数,因此循环条件恒不成立,不会执行循环体。
		\pro 假设某浮点数格式为 1 位符号、3 位阶码、4 位小数。下表给出了用该格式表达的浮点数 $(-1)^SM \cdot 2^E$ 与其二进制表示的关系。完成下表。
		\begin{table}[H]
			\centering
			\begin{tabular}{|c|c|c|c|c|}
				\hline
				描述 & 二进制表示 & $M$(写成分数) & $E$ & $f$ \\ \hline
				负零 &  & / & / & \texttt{-0.0} \\ \hline
				/ & \verb|01000101| &  &  &  \\ \hline
				最小的非规格化负数 &  &  &  &  \\ \hline
				最大的规格化正数 &  &  &  &  \\ \hline
				一 &  &  &  & \texttt{1.0} \\ \hline
				/ &  &  &  & \texttt{5.5} \\ \hline
				$+\infty$ &  & / & / & / \\ \hline
			\end{tabular}
		\end{table}
		\sol 这是基本练习,填完的表格如下:
		\begin{table}[H]
			\centering
			\begin{tabular}{|c|c|c|c|c|}
				\hline
				描述 & 二进制表示 & $M$(写成分数) & $E$ & $f$ \\ \hline
				负零 & \verb|10000000| & / & / & \texttt{-0.0} \\ \hline
				/ & \verb|01000101| & \verb|21/16| & \verb|1| & \verb|1/8| \\ \hline
				最小的非规格化负数 & \verb|10001111| & \verb|15/16| & \verb|-2| & \verb|-15/64| \\ \hline
				最大的规格化正数 & \verb|01101111| & \verb|31/16| & \verb|3| & \verb|31/2| \\ \hline
				一 & \verb|00110000| & \verb|1| & \verb|0| & \texttt{1.0} \\ \hline
				/ & \verb|01010110| & \verb|11/8| & \verb|2| & \texttt{5.5} \\ \hline
				$+\infty$ & \verb|01110000| & / & / & / \\ \hline
			\end{tabular}
		\end{table}
		\pro 假设浮点数格式 A 为 1 位符号、3 位阶码、4 位小数,浮点数格式 B 为 1 位符号、4 位阶码、3 位小数。回答下列问题。
			\qn 格式 A 中有多少个二进制表示对应于正无穷大?
			\qn 考虑能精确表示的实数的最大绝对值。A 比 B 大还是比 B 小,还是两者一样?
			\qn 考虑能精确表示的实数的最小非零绝对值。A 比 B 大还是比 B 小,还是两者一样?
			\qn 考虑能精确表示的实数的个数。A 比 B 多还是比 B 少,还是两者一样?
		\sol 根据无穷大的定义,类 IEEE 754 的浮点数都只有 1 个无穷大。一般阶码越多,则最大绝对值越大,所以 B 大;A 大,理由同前;小数长度越多,则 \verb|NaN| 越多,而总的可能表示的模式数一样,所以 B 更多。
		\pro 判断下列说法的正确性。
			\qn 对于任意的单精度浮点数 \texttt{a} 和 \texttt{b},如果 \texttt{a > b},那么 \texttt{a + 1 > b}。
			\qn 对于任意的单精度浮点数 \texttt{a} 和 \texttt{b},如果 \texttt{a > b},那么 \texttt{a + b > b + b}。
			\qn 对于任意的单精度浮点数 \texttt{a} 和 \texttt{b},如果 \texttt{a > b},那么 \texttt{a + 1 > b + 1}。
			\qn 对于任意的双精度浮点数 \texttt{d},如果 \texttt{d < 0},那么 \texttt{d * d > 0}。
			\qn 对于任意的双精度浮点数 \texttt{d},如果 \texttt{d < 0},那么 \texttt{d * 2 < 0}。
			\qn 对于任意的双精度浮点数 \texttt{d},\texttt{d == d}。
			\qn 将 \texttt{float} 转换成 \texttt{int} 时,既有可能造成舍入,又有可能造成溢出。
		\sol 1 正确;2 取 \verb|a = INF, b = FLT_MAX|;3 取 \verb|a = 16777220, b = 16777218|;4 取最大的非规格化负数;5 正确;6 \verb|nan != nan|;7 正确。
		\pro 遵循 IEEE 754 浮点数标准,考虑下列代码:
		\begin{minted}[frame=single, fontsize=\small, linenos]{c}
    for (int x = 0; ; x++) {
        float f = x;
        if (x != (int)f) {
            printf("%d", x);
            break;
        }
    }
		\end{minted}
		试问代码的运行结果是什么?或者死循环?
		\sol $16777217=2^{24}+1$,这时写成科学计数法小数点后面要 24 位,超出了单精度浮点的表示范围,会发生舍入。
		\pro 遵循 IEEE 754 浮点数标准,考虑下列代码:
		\begin{minted}[frame=single, fontsize=\small, linenos]{c}
    int x = 33554466; // 2^25 + 34
    int y = x + 8;
    for ( ; x < y; x++) {
        float f = x;
        printf("%d ", x - (int)f);
    }
		\end{minted}
		写出程序的运行结果。
		\sol 写出 $2^{25}+34$ 的科学计数法表示后按要求舍入即可,这些数舍入到单精度浮点数的结果计算如下。我们用下划线表示单精度浮点数小数表示的极限位置,后面的需要进行舍入,根据四舍六入五成双规则即可得到答案(注意最后一位相当于 1,倒数第二位相当于 2)。
		\begin{compactenum}
			\item \ \verb|1.00000000000000000001000_10 * 2^25|,差 2。
			\item \ \verb|1.00000000000000000001000_11 * 2^25|,差 $-1$。
			\item \ \verb|1.00000000000000000001001_00 * 2^25|,差 0。
			\item \ \verb|1.00000000000000000001001_01 * 2^25|,差 1。
			\item \ \verb|1.00000000000000000001001_10 * 2^25|,差 $-2$
			\item \ \verb|1.00000000000000000001011_11 * 2^25|,差 $-1$。
			\item \ \verb|1.00000000000000000001010_00 * 2^25|,差 0。
			\item \ \verb|1.00000000000000000001010_01 * 2^25|,差 1。
		\end{compactenum}
	\end{problems}

	\section*{附:有关位运算的技巧}
	\begin{summary}
        \begin{compactitem}
            \item 了解位级操作中常见的小模块,包括 de Morgan 律、加法器、减法器、popcount 等,会运用常见策略处理非常规的位运算编程要求,如掩码设计、模拟、分治、位级表示展开等。
            \item 了解各种运算符的优先顺序。知道 ANSI C 与 C99 等标准的不同。初步了解整数在发生强制转换时的基本规则。
            \item 熟练运用 IEEE 754 标准,会使用该标准直接构造浮点数。
            \item 推荐的课外参考书:\textit{Hacker's Delight} 和 \textit{Matters Computational: Ideas, Algorithms, Source Code},以及一个链接 \url{https://graphics.stanford.edu/~seander/bithacks.html},里面介绍了一些位运算魔法。Datalab 和往年有一些比较过分的考题大量出自前一本书。因为 datalab 已经 due,故列出来供大家参考。
        \end{compactitem}
    \end{summary}

    我们下面通过几个新例子来回顾 datalab 中用到的重要技巧,供大家回顾、反思和练习,以便于确保从 datalab 中学到了东西。当然,解法不唯一。

    首先是大家已经熟练使用的掩码。它是指一串二进制数字,通过与目标数字的按位操作,达到屏蔽指定位而抽取信息的需求。例如,我们要取得某个二进制数 \texttt{x} 的最高位,可以使用 \verb|(x >> 31) & 1|,这里 \texttt{1} 就是掩码。又如,获得 \texttt{x} 的所有偶数位,可以使用 \verb|x & 0xcccccccc|,这里 \texttt{0xcccccccc} 就是掩码,等等。这段话中我们回顾了 \texttt{allOddBits} 的做法。

    在第一个正式例子中,我们回顾 \texttt{bitNot, bitXor} 两个题的基本思路。
    \begin{example}[de Morgan 律]
        我们知道 \verb+~, &, |+ 可以表达“所有”的逻辑表达式,比方说异或 \verb+x ^ y = (x & ~y) | (~x & y)+。而实际上 \verb+~, |+ 就足以做到这件事,因为我们有 \verb+x & y = ~(~x | ~y)+。这样的情况称为\emph{连接词的完备集},以后大家还会反复学到。

        当然,其实单纯一个与非或者一个或非也够了。假设\emph{或非} 用 $\downarrow$ 表示(即 $\texttt{x} \downarrow \texttt{y} = \verb+~(x | y)+$),请你完成以下代码:唯一允许的位运算操作符是 $\downarrow$。
        \begin{minted}[frame=single, fontsize=\small, linenos]{c}
    unsigned xor_with_nor(unsigned x, unsigned y) {
        return _________________________; // return x ^ y with only NOR
    }
        \end{minted}
    \end{example}

    在第二个例子中,我们总结和 \texttt{isLessOrEqual, sm2tc, counter1To5} 几个题有关的重要技巧。
    \begin{example}[选择器]
        试用位级运算的技术计算表达式 \texttt{cond ? t : f}。根据注释中的提示尝试补全下面的代码,假设 \texttt{cond} 的输入总是 1 或者 0。
        \begin{minted}[frame=single, fontsize=\small, linenos]{c}
    int conditional(int cond, int t, int f) {
        /* Compute a mask that equals 0x00000000
         * or 0xFFFFFFFF depending on the value of cond */
        int mask = ____________________;
        /* Use the mask to toggle between returning t or returning f */
        return ____________________;
    }
        \end{minted}
        这个例子的意义是,如果你一定想要用 \texttt{if},可以用这个办法避开 datalab 的限制。更显然地,减法、常数乘法都是事实上可以用的,此外你也可以回顾你在实验题中是怎样表达 \texttt{==} 的。
        \begin{minted}[frame=single, fontsize=\small, linenos]{c}
    int equal(int x, int y) {
        return _________________________; // return x == y
    }

    int minus(int x, int y) {
        return _________________________; // return x - y
    }
        \end{minted}
    \end{example}

    下面的例子则和 \texttt{fullSub, satAdd} 以及 \texttt{trueFiveEighths} 有关。
    \begin{example}[加法器与减法器]
        你可能以前听说过,如果要计算两个无符号数的平均 $\left \lfloor \frac{x+y}{2} \right \rfloor$ 而不发生溢出可以用:\verb|(x & y) + ((x ^ y) >> 1)|。不难看出其原理:\verb|x ^ y| 是半加(不考虑进位)的部分,不会发生溢出,右移即可;另一部分则包括进位,当且仅当两位均为 1 时发生,而如果将这个进位保留在原地,恰好就是除以 2 的效果。这样的思想来自于计算机的加法器。
        \begin{figure}[H]
            \centering
            \tikzstyle{branch}=[fill, shape=circle, minimum size=3pt, inner sep=0pt]
            \begin{tikzpicture}[label distance=2mm]
                \node (x) at (-1,6) {$x$};
                \node (y) at ($(x) + (0,-1.2)$) {$y$};
                \node[not gate US, draw] at ($(x)+(0.5,-0.8)$) (notx) {};
                \node[not gate US, draw] at ($(y)+(0.5,-0.8)$) (noty) {};
                \node[and gate US, draw, rotate=-90, logic gate inputs=nn] at (1,3) (A) {};
                \node[and gate US, draw, rotate=-90, logic gate inputs=nn] at ($(A)+(2,0)$) (B) {};
                \node[and gate US, draw, rotate=-90, logic gate inputs=nn] at ($(B)+(2,0)$) (C) {};
                \node[or gate US, draw, rotate=-90, logic gate inputs=nn] at ($(A)+(1,-1.5)$) (D) {};
                \foreach \i in {x,y} {
                    \path (\i) -- coordinate (punt\i) (\i |- not\i.input);
                    \draw (\i) |- (punt\i) node[branch] {} |- (not\i.input);
                }
                \draw (puntx) -| (C.input 1);
                \draw (punty) -| (C.input 2);
                \draw (puntx) -| (B.input 1);
                \draw (punty) -| (A.input 2);
                \draw (notx) -| (A.input 1);
                \draw (noty) -| (B.input 2);
                \draw (A.output) -- ([yshift=-0.2cm] A.output) -| (D.input 2);
                \draw (B.output) -- ([yshift=-0.2cm] B.output) -| (D.input 1);
                \draw (C) -- ($(C) + (0, -1.8)$) -- node[right] {$C$} ($(C) + (0, -2.5)$);
                \draw (D.output) -- node[right] {$S$} ($(D) + (0, -1)$);
            \end{tikzpicture}
            \caption{半加器的结构,它计算两个一位整数的和 $S$ 及其进位 $C$。可以看出 $S = x \oplus y, C=x \& y$。}
        \end{figure}
        假如我们要运算两个两位数相加,不难看出,只需要将第一位的进位 $C$ 连接到下一位的加法中即可,即下一位是 $x \oplus y \oplus C$,以此类推。

        如果要将加法改成减法,大家都很清楚怎么办,因为 \verb|x - y = x + (~y + 1)|。

        一般来说,有符号数的处理比无符号数稍微需要一些讨论。根据上面的方法,请你尝试给出两个有符号数的平均 $\left \lfloor \frac{x+y}{2} \right \rfloor$ 而不发生溢出的算法;对 $\left \lceil \frac{x+y}{2} \right \rceil$ 做同样的事。
        
        与此相关,请问如何用位运算检查 $x+y$ 是否溢出了?\begin{hint} 检查符号位。 \end{hint}
        \begin{minted}[frame=single, fontsize=\small, linenos]{c}
    int addOK(int x, int y) {
        ________________________________;
        ________________________________;
        // You can add more lines
        return ____; // Determine if we can compute x + y without overflow
    }
        \end{minted}
    \end{example}

    剩下的若干例子中,我们仔细回顾 \texttt{countConsecutive1} 和 \texttt{palindrome} 这两个比较难的问题需要用到的分治技巧。
    \begin{example}[计数]
        让我们来决定一个给定数的二进制表示中,1 的个数是偶数还是奇数;奇数返回 1,否则返回 0。允许使用 datalab 整数部分规定的所有运算符。
        
        这里我们采用\emph{分治}的策略,首先考察比较短的数。例如,当所处理的数只有两位时,采用的代码十分显然:
        \begin{minted}[frame=single, fontsize=\small, linenos]{c}
    int bitParity2bit(int x) {
        int bit1 = 0b01 & x;
        int bit2 = 0b01 & (x >> 1);
        return bit1 ^ bit2;
    }
        \end{minted}
        将异或理解为 $\mathbb F_2$ 上的加法是非常有益的。

        对于四位整数,我们两位两位操作,并让操作的过程某种意义上“并行”进行。首先分别确定 1、2 和 3、4 位的奇偶性,所得结果再设法异或一次。(思考:\texttt{mask2} 可以改成其他数吗?)
        \begin{minted}[frame=single, fontsize=\small, linenos]{c}
    int bitParity4bit(int x) {
        int mask = 0b0101;
        int halfParity = (mask & x) ^ (mask & (x >> 1));
        int mask2 = 0b0011;
        return (mask2 & halfParity) ^ (mask2 & (halfParity >> 2));
    }
        \end{minted}

        现在,请你尝试根据上面的提示,补全下面针对八位整数的算法(要求使用操作符数目不超过 12 个)。
        \begin{minted}[frame=single, fontsize=\small, linenos]{c}
    int bitParity8bit(int x) {
        int mask = ____________________;
        int quarterParity = ____________________;
        int mask2 = ____________________;
        int halfParity = ____________________;
        int mask3 = ____________________;
        return ____________________;
    }
        \end{minted}
        最后,试完成针对十六位或更长整数的算法(注意,根据 datalab 要求,立即数有效长度不能超过 8 位)。
        
        作为模仿练习,请你对问题“决定一个给定数的二进制表示中有多少个 1”做同样的事。
        \begin{minted}[frame=single, fontsize=\small, linenos]{c}
    /* Let's count how many bits are set in a number */
    int bitCount8bit(int x) {
        int mask = ____________________;
        int quarterSum = ____________________;
        int mask2 = ____________________;
        int halfSum = ____________________;
        int mask3 = ____________________;
        return ____________________ + ____________________;
    }
        \end{minted}
    \end{example}

    \begin{example}[模拟操作 2]
        对于一个无符号整数,我们希望将它的位反过来(记为 $\operatorname{rev}(\cdot)$),例如 \texttt{0x01234567} 倒过来就是 \texttt{0xE6A2C480}。

        这个某种意义上是一个广为流传的\sout{面试题}。我们沿用\emph{分治}的思想。假设二进制数 $x$ 有 $2m$ 位 $(\underbracket{b_{2m-1} \dotsm b_m}_{x_h} \underbracket{b_{m-1} \dotsm b_0}_{x_l})_2$,那么显然有
        \[ \operatorname{rev}(x) = \operatorname{rev}(x_l) \operatorname{rev}(x_h). \]
        根据这个思路,请你补全以下代码:
        \begin{minted}[frame=single, fontsize=\small, linenos]{c}
    unsigned reverseBits(unsigned n) {
        n = (n >> 16) | (n << 16);
        n = ((n & __________) >> 8) | ((n & __________) << 8);
        n = ________________________ | _______________________;
        n = __________________________________________________;
        n = __________________________________________________;
        return n;
    }
        \end{minted}

        读过两个分治的操作之后,请思考:给一个整数,如何计算它有多少个前导零?允许的操作符包括所有的位级运算,以及加法和 \texttt{!}。\begin{hint} 二分查找。 \end{hint}
    \end{example}

    \begin{example}[算术运算]
        考虑这样一个问题:对一个无符号整数 $x$, 计算 $x \bmod 5$。除了加法之外不能用其他算术操作。

        方法上这很标准,假设 $x=(b_{31} \dotsm b_1b_0)_2$,通过计算 $2^i \pmod 5$ 的周期,我们知道
        \[ x \equiv \sum_{i=0}^{31} b_i2^i \equiv 1b_0+2b_1+4b_2+3b_3+1b_4 + \dotsb + 3b_{31} \pmod 5. \]
        技术上,对一个四位无符号整数 $(b_3b_2b_1b_0)_2$ 计算 $b_0+2b_1+4b_2+3b_3$ 可以非常直接:比如首先取得各位,然后用左移配合加法计算结果。你也可以思考有什么更省操作符数目的方法。

        注意,当你需要求取 $x \bmod 2^i$ 时,总是可以简单使用 \verb|x & ((1 << i) - 1)|。
    \end{example}

\chapter{位级表示{---}往年考题}\thispagestyle{empty}
	\begin{problems}
		\proy{2018} 下列哪种类型转换既可能导致溢出,又可能导致舍入?
		\begin{choices}
			\item \texttt{int} 转 \texttt{float}
			\item \texttt{float} 转 \texttt{int}
			\item \texttt{int} 转 \texttt{double}
			\item \texttt{float} 转 \texttt{double}
		\end{choices}
		\sol B。
		\proy{2018} 在采用小端法存储机器上运行下面的代码,输出的结果是?(\texttt{int}、\texttt{unsigned} 为 32 位长,\texttt{short} 为 16 位长,0\textasciitilde9 的 ASCII 码是 \texttt{0x30}\textasciitilde\texttt{0x39})
		\begin{minted}[frame=single, fontsize=\small, linenos]{c}
    char *s = "2018";
    int *p1 = (int *)s;
    short s1 = (*p1) >> 12;
    unsigned u1 = (unsigned) s1;
    printf("0x%x\n", u1);
		\end{minted}
		\begin{choices}
			\item \texttt{0x00002303}
			\item \texttt{0x00032303}
			\item \texttt{0xffff8313}
			\item \texttt{0x00008313}
		\end{choices}
		\proy{2018} 考虑如下函数
		\begin{minted}[frame=single, fontsize=\small, linenos]{c}
    void XOR(int x, int y) {
        y = x ^ y;
        x = x ^ y;
        y = x ^ y;
        printf(x, y);
    }
		\end{minted}
		则 $\texttt{XOR}(a, b)$ 的输出结果是什么?
		\begin{choices}
			\item $a, b$
			\item $b, a$
			\item $b, 0$
			\item $b, a\ \verb|^|\ b$
		\end{choices}
		\proy{2017} 假定一个特殊设计的计算机,将 \texttt{int} 型数据的长度从 4 字节 扩展为 $4N$ 字节,采用大端法。现将该 \texttt{int} 型所能表示的最小负数写入内存中,如下图所示。其中每个小矩形代表一个字节,请问 X 位置这个字节中的值是多少?
		\begin{figure}[H]
			\centering
			\MemoryLayout{%
				8/white/\relax,
				24/gray!25/{$4N$ 字节},
				32/white/\relax
			}{%
				\draw[thick] (0,0) -- ++(0,3) node[above] {低地址};
				\draw[thick] (-32,0) -- ++(0,3) node[above] {高地址};
				\draw[-{Stealth[angle'=45]}, thick] (-8.5,1) -- ++(0,3) node[above] {X};
			}
		\end{figure}
		\begin{choices}
			\item \texttt{00000000}
			\item \texttt{01111111}
			\item \texttt{10000000}
			\item \texttt{11111111}
		\end{choices}
		\proy{2017} 以下说法正确的是:
		\begin{choices}
			\item 负数加上负数结果都为负数
			\item 正数加上正数结果都为正数
			\item 用 \verb|&| 和 \verb|~| 可以表示所有的逻辑与或非操作
			\item 用 \verb|&| 和 \verb+|+ 可以表示所有的逻辑与或非操作
		\end{choices}
		\sol C,连接词的完备集。
		\proy{2017, 2016} 若我们采用基于 IEEE 浮点格式的浮点数表示方法,阶码字段(exp)占据 $k$ 位,小数字段(frac)占据 $n$ 位,则最小的规格化正数是:
		\begin{choices}
			\item $(1-2^{-n}) \cdot 2^{-2^{k-1}+2}$
			\item $2^{-2^{k-1}+2}$
			\item $2^{-n} \cdot 2^{-2^{k-1}+2}$
			\item $(1-2^{-n}) \cdot 2^{-2^k+1}$
		\end{choices}
		\proy{2016} 假定编译器规定 \texttt{int} 和 \texttt{short} 型长度分别为 32 位和 16 位,执行下列语句:\texttt{unsigned short x = 65530; unsigned int y = x;},得到 \texttt{y} 的机器数是 \rule{2.5cm}{0.25mm}。(用 16 进制表示,勿省略前导的 0)
		\proy{2016} 一个 C 语言程序在一台 32 位机器上运行。程序中定义了三个变量 $x, y, z$,其中 $x$ 和 $z$ 为 \texttt{int} 型,$y$ 为 \texttt{short} 型。当 $x=127, y=-9$ 时,执行赋值语句 $z=x+y$ 后,$z$ 的值是 \rule{2.5cm}{0.25mm}。(用 16 进制表示,勿省略前导的 0)
		\proy{2016} 若按 IEEE 浮点标准的单精度浮点数(符号位 1 位,阶码字段 exp 占据 8 位,小数字段 frac 占据 23 位)表示 $-8.25$,结果是 \rule{2.5cm}{0.25mm}。(用 16 进制表示)
		\proy{2015} 给定一个实数,会因为该实数表示成单精度浮点数而发生误差。不考虑 \texttt{NaN} 和 \texttt{Inf} 的情况,该绝对误差的最大值为:
		\begin{choices}
			\item $2^{103}$
			\item $2^{104}$
			\item $2^{230}$
			\item $2^{231}$
		\end{choices}
		\sol A。计算过程:浮点数阶码 8 位,即最大指数为 127,尾数为 23 位,即最低位表示的数为 $2^{-23}2^{127}=2^{104}$。那么误差在刚好落在最低位的一半的时候最大,即 $2^{103}$。
		\proy{2016} 现有一个二进制浮点的表示规则,其中 $E$ 为指数部分(3 比特),bias为 3;$M$ 为小数部分(5 比特),采用二进制补码表示形式,且取值 $0.5 \leq |M|<1$,$s$ 是浮点的符号位。该形式包含一个值为 1 的隐藏位。问 $+5_{10}$ 在该表示下的值是下列哪一个?
		\begin{choices}
			\item \texttt{010001100}
			\item \texttt{010100100}
			\item \texttt{011011010}
			\item \texttt{011110101}
		\end{choices}
		\sol D。计算 $0.5 \leq \left|\frac{5}{2^e}-1 \right|<1 \implies e=4, \text{EXP}=7 = \texttt{111}_\texttt{2}$,注意此时 $M$ 为负数,$|M|=\texttt{0.01010}_\texttt{2}$, 用补码表示得$\texttt{1.10101}_\texttt{2}$。
		\proy{2015} 关于浮点数,以下说法正确的是:
		\begin{choices}
			\item 给定任意浮点数 $a, b, x$,如果 $a>b$ 成立(指求值为 1),则一定有 $a+x>b+x$ 成立
			\item 给定任意浮点数 $a, b, x$,如果 $a>b$ 不成立(指求值为 0),则一定有 $a+x>b+x$ 不成立
			\item 不考虑结果为 \texttt{NaN, Inf} 或运算过程发生溢出的情况,高精度浮点数一定得到比低精度浮点数更精确或相同的结果
			\item 不考虑结果为 \texttt{NaN, Inf} 的情况,高精度浮点数一定得到比低精度浮点数更精确或相同的结果
		\end{choices}
		\proy{2015} 在 32 位平台上,按 C90 标准判定以下语句中结果为假的是:
		\begin{choices}
			\item \verb|return INT_MIN < INT_MAX;|
			\item \verb|return -2147483648 < 2147483647;|
			\item \verb|int a = -2147483648; return a < 2147483647;|
			\item \verb|return -2147483647 - 1 < 2147483647;|
		\end{choices}
		\begin{hint}
			C90 中的类型转换顺序:\texttt{int -> long -> unsigned -> unsigned long},$2^{31}=2147483648$。
		\end{hint}
		\sol B。此题和 C90 标准有关,分析参看 \url{https://stackoverflow.com/questions/9941261/warning-this-decimal-constant-is-unsigned-only-in-iso-c90}。
		\proy{2014} 设整数均为 32 位,其中 \texttt{unsigned x = 0x00000001; int y = 0x80000000; int z = 0x80000001;}。以下表达式求值为 1 的是:
		\begin{choices}
			\item \verb|(-1) < x|
			\item \verb|(-y) > -1|
			\item \verb|~y + y == -1|
			\item \verb|(z << 4) > (z * 16)|
		\end{choices}
		\proy{2014} 下面说法正确的是:
		\begin{choices}
			\item 数 0 的反码表示是唯一的
			\item 数 0 的补码表示不是唯一的
			\item \verb|1000_1111_1110_1111_1100_0000_0000_0000| 表示的唯一整数是 \texttt{0x8FEFC000}
			\item \verb|1000_1111_1110_1111_1100_0000_0000_0000| 如果是单精度浮点数,则其值是 $-(1.110111111)_2 \cdot 2^{31-127}$
		\end{choices}
		\sol D,注意 C 可能是有符号整数。
		\proy{2016, 2014} 下面表达式中为真的是:
		\begin{choices}
			\item \verb|(unsigned)-1 < -2|
			\item \verb|2147483647 > (int)2147483648U|
			\item \verb|(0x80005942 >> 4) == 0x09005942|
			\item \verb|2147483647 + 1 != 2147483648|
		\end{choices}
        \proy{2014} 下面关于 IEEE 浮点数标准说法正确的是哪个?
		\begin{choices}
			\item 在位数一定的情况下,不论怎么分配 exponent bits 和 fraction bits,所能表示的数的个数是不变的。
			\item 若甲类浮点数有 10 位,乙类浮点数有 11 位,那么甲所能表示的最大数一定比乙小。
			\item 若甲类浮点数有 10 位,乙类浮点数有 11 位,那么甲所能表示的最小正数一定比乙小。
		    \item “\texttt{0111000}”可能是 7 位浮点数的 \texttt{NaN} 表示。
		\end{choices}
		\proy{2014} 假设有下面 $x$ 和 $y$ 的程序定义:
		\begin{minted}[frame=single, fontsize=\small, linenos]{c}
    int x = a >> 2;
    int y = (x + a) / 4;
		\end{minted}
		那么有 \rule{2.5cm}{0.25mm} 个位于闭区间 $[-8, 8]$ 内的整数 $a$ 能使得 $x$ 和 $y$ 相等。
		\sol B,可以直接计算,注意不同运算的舍入规则。
        \proy{2013} 对于 IEEE 浮点数,如果减少 1 位指数位,将其用于小数部分,下列叙述正确的是哪个?
		\begin{choices}
			\item 能表示更多数量的实数值,但实数值取值范围比原来小了。
			\item 能表示的实数数量没有变化,但数值的精度更高了。
			\item 能表示的最大实数变小,最小的实数变大,但数值的精度更高。
			\item 以上说法都不正确。
        \end{choices}
		\proy{2017} 考虑有一种基于 IEEE 浮点格式的 9 位浮点表示格式 A。格式 A 有 1 个符号位、$k$ 个阶码位、$n$ 个小数位。现在已知 $-9$ 的位模式可以表示为 \texttt{101100010}。回答以下问题。(注:阶码偏移量为 $2^{k-1}-1$)
			\qn 求 $k$ 和 $n$ 的值。
			\qn 基于格式 A,请填写下表。值的表示可以写成整数(如 16),或者写成分数(如 17/64)。
			\begin{table}[H]
				\centering
				\begin{tabular}{|c|c|c|}
					\hline
					描述 & 二进制表示 & 值 \\ \hline
					最大的非规格化数 & {\qquad \qquad} & {\qquad \qquad} \\ \hline
					最小的正规格化数 & {\qquad \qquad} & {\qquad \qquad} \\ \hline
					最大的规格化数 & {\qquad \qquad} & {\qquad \qquad} \\ \hline
				\end{tabular}
			\end{table}
			\qn 假设格式 A 变为 1 个符号位、$k+1$ 个阶码位、$n-1$ 个小数位,那么能表示的实数数量会怎样变化?数值的精度会怎样变化?(回答增加、降低或不变即可)
		\proy{2016} 在 64 位机器上,判断表达式是否对代码生成的变量恒成立(指求值为 1)。
		\begin{minted}[frame=single, fontsize=\small, linenos]{c}
    /* random_int() 函数返回一个随机的 int 类型值 */
    int x = random_int();
    int y = random_int();
    int z = random_int();
    unsigned ux = (unsigned)x;
    long lx = (long)x; /* long 为 64 位 */
    long ly = (long)y;
    double dx = (double)x;
    double dy = (double)y;
    double dz = (double)z;
		\end{minted}
			\qn \verb+(x >= 0) || (3 * x < 0)+
			\qn \verb+(x >= 0) || (x < ux)+
			\qn \verb|((x >> 1) << 1) <= x|
			\qn \verb|((x - y) << 3) + (x >> 1) - y == 8 * x - 9 * y + x / 2|
			\qn \verb|(x - y > 0) == ((y + ~x + 1) >> 31 == 1)|
			\qn \verb|dx + dy == (double) (y + x)|
			\qn \verb|dx + dy + dz == dz + dy + dx|
			\qn \verb|(int)((lx + ly) >> 1) == ((x & y) + ((x ^ y) >> 1))|
		\proy{2016} 假设 C 语言中新定义了一种数据类型 T, 该类型为 12-bit 长的浮点数,此浮点数遵循 IEEE 浮点数格式,其字段划分如下:符号位(s) 1-bit、阶码字段(exp) 6-bit、小数字段(frac) 5-bit。
			\qn 若将该格式下能表示的所有正规格数从小到大依次排列,则相邻两数之间差值的最小值为 \rule{2.5cm}{0.25mm},最大值为 \rule{2.5cm}{0.25mm}。
			\qn 现定义了如下变量:
			\begin{minted}[frame=single, fontsize=\small, linenos]{c}
    T a = -15.875;
    T b = (1 << 28) + (1 << 24) + (1 << 22);
    T c = a * b;
			\end{minted}
			给出各变量的二进制表示。
		\proy{2015} 对于下面的每一个表达式,请选择以下选项中的\emph{一个或多个},使得该表达式恒成立,如果没有满足条件的选项则填写 none:A. \texttt{<} \quad B. \texttt{>} \quad C. \texttt{==} \quad D. \texttt{!=} \quad E. none。
		
		题目中出现的变量定义如下(浮点数保证不是 \texttt{NaN} 或者 \texttt{Inf}):\texttt{int x, y; unsigned ux = x; double d;}。
		\qn 如果 \texttt{x > 0},则 \texttt{x + 1} \rule{2.5cm}{0.25mm} \texttt{0}
		\qn 如果 \texttt{x > y},则 \texttt{ux} \rule{2.5cm}{0.25mm} \texttt{y}
		\qn 如果 \verb|((x << 31) >> 31) < 0|,则 \verb|x & 1| \rule{2.5cm}{0.25mm} \texttt{0}
		\qn 如果 \verb|((unsigned char)x >> 1) < 64|,则 \texttt{(char)x} \rule{2.5cm}{0.25mm} \texttt{0}
		\qn 如果 \texttt{d < 0},则 \texttt{d * 2} \rule{2.5cm}{0.25mm} \texttt{0}
		\qn 如果 \texttt{d < 0},则 \texttt{d * d} \rule{2.5cm}{0.25mm} \texttt{0}
		\qn \verb|x ^ y ^ (~x) - y| \rule{2.5cm}{0.25mm} \verb|y ^ x ^ (~y) - x|
		\qn \verb|(((!!ux)) << 31) >> 31)| \rule{2.5cm}{0.25mm} \verb|(((!!x) << 31) >> 31)|
		\sol 分别为 D、D、BD、E、AD、E、C、C。
		\proy{2015} 考虑一种 12-bit 长的浮点数,此浮点数遵循 IEEE 浮点数格式,浮点数的字段划分如下:符号位(s) 1-bit、阶码字段(exp) 4-bit、小数字段(frac) 7-bit。回答下列问题。
			\qn 请写出在下列区间中包含多少个用上面规则精确表示的浮点数:$[1, 2); [2, 3)$。
			\qn 填写下表。
			\begin{table}[H]
				\centering
				\begin{tabular}{|c|c|}
					\hline
					描述 & 二进制表示 \\ \hline
					最大的非规格化数 & {\qquad \qquad \qquad \qquad} \\ \hline
					最小的正规格化数 & {\qquad \qquad \qquad \qquad} \\ \hline
					$17 \frac{1}{16}$ & {\qquad \qquad \qquad \qquad} \\ \hline
					$-\frac{1}{8192}$ & {\qquad \qquad \qquad \qquad} \\ \hline
					$20 \frac{3}{8}$ & {\qquad \qquad \qquad \qquad} \\ \hline
					$-\infty$ & {\qquad \qquad \qquad \qquad} \\ \hline
				\end{tabular}
			\end{table}
		\proy{2014} 回答下列问题。
			\qn 假设下列 \texttt{unsigned} 和 \texttt{int} 数均为 5 位(有符号整型用补码运算表示):\texttt{int y = - 7; unsigned z = y;}。确定 \texttt{y, z} 和最小有符号整数的十进制表示和二进制表示。
			\qn 按照 IEEE 浮点数标准,首先将下列两个数表示 $(-1)^sM \cdot 2^E$ 的形式,然后写出其二进制表示:$0.375, -12.5$。
	\end{problems}