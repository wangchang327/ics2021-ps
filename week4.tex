\chapter{存储器层次结构}
    \begin{summary}
        \begin{compactitem}
            \item 了解 SRAM、DRAM 等易失性存储器及其变种(如 DDR SDARM 等)的功能和历史;了解 PROM、EPROM 等非易失性存储器及其变种的功能和历史。
            \item 知道传统旋转磁盘的结构相关概念和工作原理,会计算其存储容量和使用磁盘访问数据的延迟。了解固态硬盘的工作原理。
            \item 了解计算机体系结构中总线的作用以及含总线的计算机基本结构。
            \item 熟悉时间局部性和空间局部性的概念,会阅读程序判断局部性的概念。
        \end{compactitem}
    \end{summary}

    \begin{problems}
        \pro 对于下列描述,是 SRAM 更符合还是 DRAM 更符合,还是均符合?
            \qn 访问速度更快
            \qn 每比特需要的晶体管数目少
            \qn 单位容量造价更便宜
            \qn 常用作主存
            \qn 需要定期刷新
            \qn 断电后失去存储的信息
            \qn 支持随机访问
            \qn 一种变种是 SDRAM
        \pro 下列存储器中,属于易失性存储介质的有:
            \begin{choices}
                \item DRAM
                \item SRAM
                \item ROM
                \item 软盘
                \item SSD
                \item U 盘
            \end{choices}
        \pro 已知一个双面磁盘有 2 个盘片、10000 个柱面,每条磁道有 400 个扇区,每个扇区容量为 512 字节,则它的存储容量是 \rule{2.5cm}{0.25mm} GB。
        \pro 已知一个磁盘的平均寻道时间为 \SI{6}{ms},旋转速度为 \SI{7500}{RPM},那么它的平均访问时间大约为 \rule{2.5cm}{0.25mm} ms。
        \pro 已知一个磁盘每条磁道平均有 400 个扇区,旋转速度为 \SI{6000}{RPM},那么它的平均传送时间大约为 \rule{2.5cm}{0.25mm} ms。
        \pro 考虑如下程序:
        \begin{minted}[frame=single, fontsize=\small]{c}
    for (int i = 0; i < n; i++) {
        B[i] = 0;
        for (int j = 0; j < m; j++)
            B[i] += A[i][j];
    }
        \end{minted}
        判断下列说法的正确性。
            \qn 对于数组 \verb|A| 的访问体现了时间局部性。
            \qn 对于数组 \verb|A| 的访问体现了空间局部性。
            \qn 对于数组 \verb|B| 的访问体现了时间局部性。
            \qn 对于数组 \verb|B| 的访问体现了空间局部性。
    \end{problems}

\chapter{存储器层次结构{---}往年考题}
    \begin{problems}
        \proy{2018} 以下关于存储的叙述中,正确的是:
        \begin{choices}
            \item 由于基于 SRAM 的内存性能与 CPU 的性能有很大差距,因此现代计算机使用更快的基于 DRAM 的高速缓存,试图弥补 CPU 和内存间性能的差距。
            \item SSD 相对于旋转磁盘而言具有更好的读性能,但是 SSD 写的速度通常比读的速度慢得多,而且 SSD 比旋转磁盘单位容量的价格更贵,此外 SSD 底层基于 EEPROM 的闪存会磨损。
            \item 一个有 2 个盘片,10000 个柱面,每条磁道平均有 400 个扇区,每个扇区有 512 个字节的双面磁盘的容量为 \SI{8}{GB}。
            \item 访问一个磁盘扇区的平均时间主要取决于寻道时间和旋转延迟,因此一个旋转速率为 \SI{6000}{RPM}、平均寻道时间为 \SI{9}{ms} 的磁盘的平均访问时间大约为 \SI{19}{ms}。
        \end{choices}
        \proy{2017} 以下计算机部件中,通常不用于存储器层次结构(Memory Hierarchy)的是:
        \begin{choices}
            \item 高速缓存
            \item 内存
            \item 硬盘
            \item 优盘(U 盘)
        \end{choices}
        \proy{2017} 关于局部性(locality)的描述,正确的是:
        \begin{choices}
            \item 数据的时间局部性或数据空间局部性,在任何有意义的程序中都能体现。
            \item 指令的时间局部性或数据空间局部性,在任何有意义的程序中都能体现。
            \item 数据的时间局部性,在任何循环操作中都能体现。
            \item 数据的空间局部性,在任何数组操作中都能体现。
        \end{choices}
        \proy{2016} 以下关于存储结构的讨论,哪个是正确的?
        \begin{choices}
            \item 增加额外一级存储,数据存取的延时一定不会下降。
            \item 增加存储的容量,数据存取的延时一定不会下降。
            \item 增加额外一级存储,数据存取的延时一定不会增加。
            \item 以上选项都不正确。
        \end{choices}
        \proy{2016} 关于局部性(locality)的描述,不正确的是:
        \begin{choices}
            \item 循环通常具有很好的时间局部性。
            \item 循环通常具有很好的空间局部性。
            \item 数组通常具有很好的时间局部性。
            \item 数组通常具有很好的空间局部性。
        \end{choices}
        \proy{2015} 下面关于存储器的说法,错误的是:
        \begin{choices}
            \item SDRAM 的速度比 FPM DRAM 快。
            \item SDRAM 的 RAS 和 CAS 请求共享相同的地址引脚。
            \item 磁盘的寻道时间和旋转延迟大致在一个数量级。
            \item 固态硬盘的随机读写性能基本相当。
        \end{choices}
        \proy{2015} 某磁盘的旋转速率为 \SI{7200}{RPM},每条磁道平均有 400 扇区,则一个扇区的平均传送时间为 \rule{2.5cm}{0.25mm} ms。
    \end{problems}